\section{Diskussion}
\label{sec:Diskussion}
Die errechneten Strömungsgeschwindigkeiten verhalten sich wie erwartet. Bei jeder der fünf unterschiedlichen
Antriebseinstellungen sollte die Geschwindigkeit im größten Rohr am geringsten und im kleinsten Rohr am größten sein.
Dieser Zusammenhang ist auch deutlich erkennbar. Zudem decken sich die gemessenen Frequenzdifferenzen auch in etwa mit den
dazu vergleichbaren Frequenzdifferenzen aus der zweiten Messung. Dass die Geschwindigkeiten daher vollständig den realen Werten entsprechen
ist jedoch nicht gesichert, da bei der Rechnung idealisierte Annahmen wie z.B. eine vollständig laminare Dopplerflüssigkeit gemacht werden.
Zudem ist teilweise das Ablesen der Messwerte von dem Rechner durch häufige Schwankungen erschwert.
Allerdings sollte eine ungefähre Übereinstimmung zur Realität trotzdem gegeben sein.

Die Momentangeschwindigkeit beschreibt näherungsweise den zu erwarteten Verlauf im Rohr. Sie ist in der Mitte des Rohres größer als am Rand. Die Abweichung der ersten drei und letzten
Messwerte sind dadurch zu erklären, dass die Messtiefe außerhalb des Schlauches war, weshalb diese nicht relevant sind.
Auch das Verhalten der Streuintensitäten war wie zu erwarten, da am Rand größere Turbulenzen, wegen der Wand des Schlauches, auftreten. Die Messwerte am linken und rechten Rand liegen
wieder außerhalb des Schlauches, weshalb diese nicht relevant sind.
