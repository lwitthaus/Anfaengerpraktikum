\section{Diskussion}
\label{sec:Diskussion}

Die Momentangeschwindigkeit beschreibt näherungsweise den zu erwarteten Verlauf im Rohr. Sie ist in der Mitte des Rohres größer als am Rand. Die Abweichung der ersten drei und letzten
Messwerte sind dadurch zu erklären, dass die Messtiefe außerhalb des Schlauches war, weshalb diese nicht relevant sind.
Auch das Verhalten der Streuintensitäten war wie zu erwarten, da am Rand größere Turbulenzen, wegen der Wand des Schlauches, auftreten. Die Messwerte am linken und rechten Rand liegen
wieder außerhalb des Schlauches, weshalb diese nicht relevant sind.
