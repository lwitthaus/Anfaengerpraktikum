\section{Auswertung}
\label{sec:Auswertung}

\subsection{Bestimmung der Strömungsgeschwindigkeit}

In Tabelle 1 sind die gemessenen Frequenzen in Abhängigkeit von der Strömungsgeschwindigkeit $v$ und des Winkels $\alpha$ der einzelnen Rohre.
Zusätzlich werden die mit diesen Werten aus Gleichung 3 berechneten Geschwindigkeiten der Dopplerflüssigkeit aufgeführt.

\begin{table}
\centering
\caption{Frequenzen und Geschwindigkeiten in Abhängigkeit von der Dicke des Rohres, des Winkels und der Strömungsgeschwindigkeit.}
\sisetup{table-format=2.1}
\begin{tabular}{S [table-format=3.1] | S S S | S S S | S S S}
\toprule
& \multicolumn{3}{c}{$7$mm Rohr} & \multicolumn{3}{c}{$10$mm Rohr} & \multicolumn{3}{c}{$16$mm Rohr} \\
\cmidrule(lr){2-4}\cmidrule(lr){5-7}\cmidrule(lr){8-10}
{$v/$rpm}
& {$\alpha$/°}& {$\nu/Hz$} & {$v/$ m/s} & {$\alpha$/°} & {$\nu/$Hz} & {$v/$ m/s} & {$\alpha$/°} & {$\nu/$Hz} & {$v/$ m/s} \\
\midrule
2000 & 15 & -73 & -0,191 & 15 & -49  & -0,128 & 15 & -61 & -0,160 \\
2000 & 30 & 122  & 0,164 & 30 &  73  & 0,098 & 30 &  49 & 0,066  \\
2000 & 60 & -159 & -0,124 & 60 & -98  & -0,076 & 60 & -61 & -0,048 \\
\midrule
3000 & 15 & -146 & -0,382 & 15 & -85  & -0,222 & 15 & -49 & -0,128  \\
3000 & 30 & 232 & 0,313 & 30 & 110  & 0,148 & 30 & 73 & 0,098 \\
3000 & 60 & -415 & -0,323 & 60 &-159  & -0,124 & 60 &-122 & -0,095  \\
\midrule
4000 & 15 & -244 & -0,639 & 15 & -146 & -0,382 & 15 & -73 & -0,191 \\
4000 & 30 &  439 & 0,592 & 30 &  232 & 0,313 & 30 & 110 & 0,148 \\
4000 & 60 & -842 & -0,656 & 60 & -415 & 0,323 & 60 &-183 & -0,143 \\
\midrule
5000 & 15 & -415  & -1,086 & 15 & -183 & -0,479 & 15 & -98 & -0,257 \\
5000 & 30 &  732  & 0,987 & 30 &  354 & 0,477 & 30 & 171 & 0,231 \\
5000 & 60 & -1440 & -1,121 & 60 & -696 & -0,542 & 60 &-281 & -0,219 \\
\midrule
6000 & 15 & -610 & -1,597 & 15 & -269 & -0,704 & 15 & -134 & -0,351 \\
6000 & 30 & 1160 & 1,564 & 30 &  549 & 0,740 & 30 &  232 & 0,313 \\
6000 & 60 &-2222 & -1,730 & 60 &-1062 & -0,827 & 60 & -391 & -0,304 \\
\bottomrule
\end{tabular}
\end{table}

Daraus ergeben sich die in Tabelle \ref{tab:mittel} dargestellten Mittelwerte $v_m$ für die einzelnen Geschwindigkeiten.

\begin{table}
\label{tab:mittel}
\centering
\caption{Mittelwerte der errechneten Geschwindigkeiten.}
\sisetup{table-format=2.1}
\begin{tabular}{S [table-format=3.1] S S S}
\toprule
& \multicolumn{1}{c}{$7$mm Rohr} & \multicolumn{1}{c}{$10$mm Rohr} & \multicolumn{1}{c}{$16$mm Rohr} \\
\cmidrule(lr){2-2}\cmidrule(lr){3-3}\cmidrule(lr){4-4}
{$v/$rpm}
& {$v_{mk}$/ m/s}& {$v_{mm}$/ m/s} & {$v_{mg}$/ m/s} \\
\midrule
2000 & 0,160 \pm 0,028  & 0,101 \pm 0,021 & 0,091 \pm 0,049  \\
\midrule
3000 & 0,339 \pm 0,030 & 0,165 \pm 0,042 & 0,107 \pm 0,015  \\
\midrule
4000 & 0,629 \pm 0,027 & 0,339 \pm 0,030 & 0,161 \pm 0,022  \\
\midrule
5000 & 1,065 \pm 0,057 & 0,499 \pm 0,030  & 0,236 \pm 0,016  \\
\midrule
6000 & 1,630 \pm 0,072 & 0,757 \pm 0,052 & 0,323 \pm 0,020  \\
\bottomrule
\end{tabular}
\end{table}

Trägt man $\frac{\nu}{cos(\alpha)}$ gegen die jeweils errechnete Strömungsgeschwindigkeit auf, so ergibt sich
das folgende Diagramm (hier exemplarisch für das dünne Rohr).

\begin{figure}
  \centering
  \includegraphics{plot5.pdf}
  \caption{Streuintensität bei einer Strömungsgeschwindigkeit von 3000rpm}
  \label{fig:plot}
\end{figure}

Es ergibt sich eine Gerade. Bei den anderen Rohrdicken verhalten sich die Diagramme analog.

\subsection{Bestimmung des Strömungsprofils}

Die Frequenz, die Tiefe und die Standardabweichung (Std.) werden in Tabelle (?) für die Geschwindigkeiten $v_1 = 3000$rpm und $v_2 = 5000$rpm dargestellt.

\begin{table}
\centering
\caption{Frequenzen und Tiefen bei verschiedenen Strömungsgeschwindigkeiten.}
\setlength{\tabcolsep}{14pt}
\sisetup{table-format=2.1}
\begin{tabular}{S [table-format=3.1] |S S S | S S S}
\toprule
& \multicolumn{3}{c}{$v_1 = 3000$rpm} & \multicolumn{3}{c}{$v_2=5000$rpm} \\
\cmidrule(lr){2-4}\cmidrule(lr){5-7}
{Tiefe$/\symup{\mu s}$}
& {$\nu/$Hz}& {Std/\%} & {$v/\symup{\frac{m}{s}}$}& {$\nu/$Hz} & {Std/\%} & {$v/\symup{\frac{m}{s}}$}  \\
\midrule
10,0 & -98 & 9,0& -0.26  & -232 & 7,0 & -0.61      \\
10,5 & -98 & 6,5& -0.26  & -232 & 7,0 & -0.61      \\
11,0 & -98 & 9,0& -0.26  & -232 & 7,0 & -0.61     \\
11,5 & -98 & 9,0& -0.26  & -183 &12,5 & -0.48       \\
12,0 & -73 &13,0& -0.19  & -134 &10,0 & -0.35       \\
12,5 & -73 & 9,5& -0.19  & -159 & 8,0 & -0.42       \\
13,0 & -85 & 6,5& -0.22  & -183 & 5,0 & -0.48     \\
13,5 & -85 & 7,0& -0.22  & -232 & 6,0 & -0.61     \\
14,0 & -98 & 5,5& -0.26  & -269 & 4,0 & -0.70     \\
14,5 &-110 & 5,0& -0.29  & -305 & 3,0 & -0.80     \\
15,0 &-110 & 4,0& -0.29  & -305 & 3,5 & -0.80     \\
15,5 & -98 & 6,0& -0.26  & -293 & 3,2 & -0.77     \\
16,0 & -85 & 7,0& -0.22  & -256 & 3,5 & -0.67     \\
16,5 & -73 & 8,0& -0.19  & -195 & 5,0 & -0.51      \\
17,0 & -73 &10,0& -0.19  & -146 & 7,2 & -0.38      \\
17,5 & -85 &10,2& -0.22  & -140 &13,0 & -0.37    \\
18,0 & -85 &10,0& -0.22  & -183 &10,0 & -0.48    \\
18,5 & -85 &10,0& -0.22  & -208 & 8,0 & -0.54    \\
19,0 & -85 & 9,5& -0.22  & -195 & 8,0 & -0.51    \\
\bottomrule
\end{tabular}
\end{table}

Der Dopplerwinkel beträgt dabei $\alpha=15°$.

Für beide Geschwindigkeiten werden die Streuintensitäten und die Momentangeschwindigkeiten in Abhängigkeit von der Tiefe dargestellt. Dies ist in den Diagrammen \ref{fig:plot1}
\ref{fig:plot2} \ref{fig:plot3} \ref{fig:plot4} dargetellt.


\begin{figure}
  \centering
  \includegraphics{plot.pdf}
  \caption{Streuintensität bei einer Strömungsgeschwindigkeit von 3000rpm}
  \label{fig:plot1}
\end{figure}

\begin{figure}
  \centering
  \includegraphics{plot3.pdf}
  \caption{Momentangeschwindigkeit bei einer Strömungsgeschwindigkeit von 3000rpm}
  \label{fig:plot2}
\end{figure}


\begin{figure}
  \centering
  \includegraphics{plot2.pdf}
  \caption{Streuintensität bei einer Strömungsgeschwindigkeit von 5000rpm}
  \label{fig:plot3}
\end{figure}

\begin{figure}
  \centering
  \includegraphics{plot4.pdf}
  \caption{Momentangeschwindigkeit bei einer Strömungsgeschwindigkeit von 5000rpm}
  \label{fig:plot4}
\end{figure}
