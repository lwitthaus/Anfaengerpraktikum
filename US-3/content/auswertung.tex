\section{Auswertung}
\label{sec:Auswertung}

\subsection{Bestimmung der Strömungsgeschwindigkeit}

In Tabelle 1 sind die gemessenen Frequenzen in Abhängigkeit von der Strömungsgeschwindigkeit $v$ und des Winkels $\alpha$ der einzelnen Rohre.

\begin{table}
\centering
\caption{Frequenzen in Abhängigkeit von der Dicke des Rohres, des Winkels und der Strömungsgeschwindigkeit.}
\sisetup{table-format=2.1}
\begin{tabular}{S [table-format=3.1] S S S S S S}
\toprule
& \multicolumn{2}{c}{$7$mm Rohr} & \multicolumn{2}{c}{$10$mm Rohr} & \multicolumn{2}{c}{$16$mm Rohr} \\
\cmidrule(lr){2-3}\cmidrule(lr){4-5}\cmidrule(lr){6-7}
{$v/$rpm}
& {$\alpha$/°}& {$\nu/Hz$} & {$\alpha$/°} & {$\nu/$Hz} & {$\alpha$/°} & {$\nu/$Hz} \\
\midrule
2000 & 15 & -73  & 15 & -49  & 15 & -61  \\
2000 & 30 & 122  & 30 &  73  & 30 &  49  \\
2000 & 60 & -159 & 60 & -98  & 60 & -61 \\
\midrule
3000 & 15 & -146 & 15 & -85  & 15 & -49  \\
3000 & 30 &  232 & 30 & 110  & 30 &  73  \\
3000 & 60 & -415 & 60 &-159  & 60 &-122  \\
\midrule
4000 & 15 & -244 & 15 & -146 & 15 & -73 \\
4000 & 30 &  439 & 30 &  232 & 30 & 110 \\
4000 & 60 & -842 & 60 & -415 & 60 &-183 \\
\midrule
5000 & 15 & -415  & 15 & -183 & 15 & -98 \\
5000 & 30 &  732  & 30 &  354 & 30 & 171 \\
5000 & 60 & -1440 & 60 & -696 & 60 &-281 \\
\midrule
6000 & 15 & -610 & 15 & -269 & 15 & -134 \\
6000 & 30 & 1160 & 30 &  549 & 30 &  232 \\
6000 & 60 &-2222 & 60 &-1062 & 60 & -391 \\
\bottomrule
\end{tabular}
\end{table}









\subsection{Bestimmung des Strömungsprofils}

Die Frequenz, die Tiefe und die Standardabweichung (Std.) werden in Tabelle (?) für die Geschwindigkeiten $v_1 = 3000$rpm und $v_2 = 5000$rpm dargestellt.

\begin{table}
\centering
\caption{Frequenzen und Tiefen bei verschiedenen Strömungsgeschwindigkeiten.}
\setlength{\tabcolsep}{14pt}
\sisetup{table-format=2.1}
\begin{tabular}{S [table-format=3.1] |S S S | S S S}
\toprule
& \multicolumn{3}{c}{$v_1 = 3000$rpm} & \multicolumn{3}{c}{$v_2=5000$rpm} \\
\cmidrule(lr){2-4}\cmidrule(lr){5-7}
{Tiefe$/\symup{\mu s}$}
& {$\nu/$Hz}& {Std/\%} & {$v/\symup{\frac{m}{s}}$}& {$\nu/$Hz} & {Std/\%} & {$v/\symup{\frac{m}{s}}$}  \\
\midrule
10,0 & -98 & 9,0& -0.26  & -232 & 7,0 & -0.61      \\
10,5 & -98 & 6,5& -0.26  & -232 & 7,0 & -0.61      \\
11,0 & -98 & 9,0& -0.26  & -232 & 7,0 & -0.61     \\
11,5 & -98 & 9,0& -0.26  & -183 &12,5 & -0.48       \\
12,0 & -73 &13,0& -0.19  & -134 &10,0 & -0.35       \\
12,5 & -73 & 9,5& -0.19  & -159 & 8,0 & -0.42       \\
13,0 & -85 & 6,5& -0.22  & -183 & 5,0 & -0.48     \\
13,5 & -85 & 7,0& -0.22  & -232 & 6,0 & -0.61     \\
14,0 & -98 & 5,5& -0.26  & -269 & 4,0 & -0.70     \\
14,5 &-110 & 5,0& -0.29  & -305 & 3,0 & -0.80     \\
15,0 &-110 & 4,0& -0.29  & -305 & 3,5 & -0.80     \\
15,5 & -98 & 6,0& -0.26  & -293 & 3,2 & -0.77     \\
16,0 & -85 & 7,0& -0.22  & -256 & 3,5 & -0.67     \\
16,5 & -73 & 8,0& -0.19  & -195 & 5,0 & -0.51      \\
17,0 & -73 &10,0& -0.19  & -146 & 7,2 & -0.38      \\
17,5 & -85 &10,2& -0.22  & -140 &13,0 & -0.37    \\
18,0 & -85 &10,0& -0.22  & -183 &10,0 & -0.48    \\
18,5 & -85 &10,0& -0.22  & -208 & 8,0 & -0.54    \\
19,0 & -85 & 9,5& -0.22  & -195 & 8,0 & -0.51    \\
\bottomrule
\end{tabular}
\end{table}

Der Dopplerwinkel beträgt dabei $\alpha=15°$.

Für beide Geschwindigkeiten werden die Streuintensitäten und die Momentangeschwindigkeiten in Abhängigkeit von der Tiefe dargestellt. Dies ist in den Diagrammen \ref{fig:plot1}
\ref{fig:plot2} \ref{fig:plot3} \ref{fig:plot4} dargetellt.


\begin{figure}
  \centering
  \includegraphics{plot.pdf}
  \caption{Streuintensität bei einer Strömungsgeschwindigkeit von 3000rpm}
  \label{fig:plot1}
\end{figure}

\begin{figure}
  \centering
  \includegraphics{plot3.pdf}
  \caption{Momentangeschwindigkeit bei einer Strömungsgeschwindigkeit von 3000rpm}
  \label{fig:plot2}
\end{figure}


\begin{figure}
  \centering
  \includegraphics{plot2.pdf}
  \caption{Streuintensität bei einer Strömungsgeschwindigkeit von 5000rpm}
  \label{fig:plot3}
\end{figure}

\begin{figure}
  \centering
  \includegraphics{plot4.pdf}
  \caption{Momentangeschwindigkeit bei einer Strömungsgeschwindigkeit von 5000rpm}
  \label{fig:plot4}
\end{figure}
