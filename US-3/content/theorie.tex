\section{Theorie}
\label{sec:Theorie}

Ultraschall ist eine Schallwelle mit einem Frequenzbereich von $20\,$kHz bis $1\,$GHz und somit nicht mehr für den
Menschen hörbar.

Diese Wellen können unter anderem mit einem piezo-elektrischen Kristall, meist Quarze, erzeugt werden. In einem elektrischen Wechselfeld werden
diese Kristalle zu Schwingungen angeregt und strahlen dabei Ultraschallwellen ab.

Mithilfe von Ultraschall kann die Geschwindigkeit von Strömungen berechnet werden. Die Frequenz der Ultraschallwelle wird beim Auftreffen auf ein bewegtes Objekt
gemäß dem Doppler-Effekt verschoben. Dieser Effekt beschreibt die Vergrößerung der Frequenz einer Schallquelle, falls sich diese einem Empfänger nähert, oder der
Empfänger sich der Quelle nähert. Dementsprechend verringert sich die Frequenz, falls sich die Quelle von dem Empfänger entfernt, oder sich
der Empfänger von der Quelle entfernt.

\begin{align}
  &\text{Für bewegte Quelle:} \: &\nu = \frac{\nu_0}{1 \mp \frac{v_S}{c}} \\
  &\text{Für bewegten Empfänger:} \: &\nu = \nu_0 \left(1 \pm \frac{v_E}{c}  \right)
\end{align}

Mit der Geschwindigkeit des Senders $v_S$ und der Geschwindigkeit des Empfängers $v_E$ und der Schallgeschwindigkeit des jeweiligen Mediums $c$.

Für die Frequenzverschiebung $\Delta \nu$ bei dem Auftreffen des Ultraschalls auf die Strömung unter einem Winkel gilt:
\begin{align}
  \Delta \nu = 2\nu \frac{v_E}{c}\cos{\alpha}
\end{align}

In Abbildung 1 ist die Beziehung des Winkels $\alpha$ mit $v_E$ dargestellt.

\begin{figure}[H]
  \centering
  \includegraphics[height=5cm]{winkel.PNG}
  \caption{Verfahren zur Bestimmung der Geschwindigkeit von Strömungen. \cite{sample}}
  \label{fig:Linienspektrum}
\end{figure}
