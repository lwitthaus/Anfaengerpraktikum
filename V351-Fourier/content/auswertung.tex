\section{Auswertung}
\label{sec:Auswertung}

\subsection{Fourier-Analyse}

Die gemessenen Amplituden der drei Spannungen werden in den Tabellen 1,2 und 3 dargestellt.
Die Quotienten $\frac{U_n}{U_1}$ werden ebenfalls in den tabellen dargestellt, da diese für
spätere Rechnungen nötig sind.
 Die Frequenz des Funktionsgenerators beträgt bei allen Messungen $1$kHz.

\begin{table}[H]
  \centering
  \caption{Amplituden und Frequenzen der Rechteckspannung}
  \label{tab:Rechteckspannung}
  \begin{tabular}{c c c c c c}
    \toprule
    $U_R/V$ & $\frac{U_n}{U_1}$ & $U_S/V$ & $\frac{U_n}{U_1}$ & $U_D/V$ & $\frac{U_n}{U_1}$ \\
    \midrule
    4.44 & 1.00 & 2.18 & 1.00 & 2.80 &  1.00 \\
    1.36 & 0.31 & 1.04 & 0.48 & 0.30 &  0.11\\
    0.72 & 0.16 & 0.64 & 0.29 & 0.10 &  0.04\\
    0.56 & 0.13 & 0.44 & 0.20 & 0.05 &  0.02\\
    0.48 & 0.11 & 0.30 & 0.14 & 0.03 &  0.01\\
    0.40 & 0.09 & 0.26 & 0.12 & 0.02 &  0.01\\
    0.32 & 0.07 & 0.24 & 0.11 & 0.02 &  0.01\\
    0.24 & 0.05 & 0.20 & 0.09 & 0.01 &  0.00\\
    0.24 & 0.05 & 0.18 & 0.08 & 0.01 &  0.00\\
    \bottomrule
  \end{tabular}
\end{table}

\begin{figure}
  \centering
  \includegraphics{plot.pdf}
  \caption{Plot.}
  \label{fig:plot}
\end{figure}
