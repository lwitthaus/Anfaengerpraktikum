\section{Durchführung}
\label{sec:Durchführung}

\subsection{Fourierreihen}
Zur Vorbereitung werden die Fourierkomponenten von drei Funktionen bestimmt, wobei
die Rechteckspannung, die Sägezahnspannung und die Dreiecksspannung gewählt werden. Die Funktion werden als ungerade oder
gerade definiert um die Fourierkomponenten möglichst einfach darstellen zu können.
Es ergeben sich folgende Fourierkomponenten:
\begin{align}
  a_n &= 0 \:\: &b_n = \frac{4A}{\pi n}  \:\:\:\:\: &\text{Rechteckspannung} \\
  a_n &= 0 \:\: &b_n = \frac{m_S}{\pi n}   \:\:\:\:\: &\text{Sägezahnspannung} \\
  b_n &= 0 \:\: &a_n = \frac{m_D}{\pi^2 n^2} \:   &\text{Dreiecksspannung}
\end{align}
$A$ ist die Amplitude der Rechteckspannung und $m_S$ die Steigung der Flanken der Sägezahnspannung.
$m_D$ ist die Steigung der Flanke der Dreieckspannung.


\subsection{Fourier-Analyse}
Ein Funktionsgenerator wird an das Oszilloskop angeschlossen, welches eine Fourier-Transformation
durchführt. Der Funktionsgenerator generiert die Rechteckspannung, Sägezahnspannung und die
Dreiecksspannung. Die auf dem Bildschirm angezeigten Peaks werden gemessen, sowie die
Frequenzen.

\subsection{Fourier-Synthese}
Ein Signalgenerator wird an das Oszilloskop angeschlossen, so dass die Grundschwingung und eine
beliebige Oberschwingung auf dem Bildschirm angezeigt wird. Abhängig von der
darzustellenden Funktion werden die einzelnen Oberschwingungen auf die Grundschwingung addiert.
Bei der Rechteckspannung ist jede Fourierkomponente $b_n$ mit geradem $n$ Null, weshalb nur jede zweite
Oberschwingung auf die Grundschwinung addiert wird. Dies gilt ebenso für die Dreiecksspannung.
Die Frequenzen der Oberschwingungen werden so eingestellt, dass die auf dem Bildschirm angezeigte
Funktion die theoretisch zu erwartende Funktion bestmöglich approximiert. Die
Amplituden der nten-Oberschwingung der Rechteckspannung und der Sägezahnspannung ist dabei die Amplitude der Grundschwingung dividiert durch $n$.
Bei der Dreieckspannung wird mit $n^2$ dividiert.
