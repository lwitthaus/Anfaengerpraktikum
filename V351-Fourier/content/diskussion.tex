\section{Diskussion}
\label{sec:Diskussion}

Bei der Fourier-Analyse weisen die Amplituden der einzelnen Oberschwingungen
einen in etwa der Theorie entsprechende Abfall auf. Die Abweichung treten
möglicherweise durch das nicht störfreie Signal auf. Zudem kann die Fourier-Transformation
von dem Oszilloskop bei einem endlichen Zeitintervall nicht genau durchgeführt werden.
Hinzu kommt bei der Dreieckspannung, dass die schnell abfallenden Amplituden nicht mehr
genau gemessen werden können.

Die Rechteck- und Sägezahnspannung konnten nicht genau synthetisiert werden.
Grund dafür ist, dass die Regler nicht genau kalibriert sind und die Phasen nicht
optimal eingestellt werden können. Somit ist auch das Gibbs'sche Phänomen an den
Sprungstellen nicht genau zu beobachten. Die Dreieckspannung hingegen kann
gut approximiert werden, da die Oberwellen mit $\frac{1}{n^2}$ abfallen und
dadurch nur die ersten Oberwellen dominieren. Die endliche Anzahl an Oberwellen
hat also keinen so großen Einfluss.
