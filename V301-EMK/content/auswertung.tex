\section{Auswertung}
\label{sec:Auswertung}

Die gemessene Leerlaufspannung, der Monozelle beträgt $\SI{1.7}{\volt}$. Der Innenwiderstand dieser, beträgt $\SI{10}{\mega\ohm}$.

\subsection{Bestimmung der leerlaufspannung und des Innenwiderstandes einer Monozelle}

In Tabelle 1 werden die gemessenen Spannungen $U_K$ und die zugehörigen Stromstärken $I$ dargestellt.

\begin{table}[H]
  \centering
  \caption{Gemessene Spannungen und Stromstärken}
  \label{tab:spannung1}
  \begin{tabular}{c c}
    \toprule
    $I/$mA & $U/$V \\
    \midrule
    80.8  &  0.05 \\
    75.0  &  0.10 \\
    70.0  &  0.20 \\
    65.0  &  0.26 \\
    60.0  &  0.35 \\
    50.0  &  0.53 \\
    40.0  &  0.70 \\
    30.0  &  0.85 \\
    20.0  &  1.01 \\
    15.0  &  1.10 \\
    \bottomrule
  \end{tabular}
\end{table}

Die Messwerte werden in einem Diagramm aufgetragen und eine lineare Regression wird durch geführt.


\begin{figure}[H]
  \centering
  \includegraphics{plot.pdf}
  \caption{$U_k$ in Abhängigkeit von der Stromstärke}
  \label{fig:plot}
\end{figure}

Die Gerade kann durch die Gleichung $y = ax + b$ beschrieben werden. Die Parameter $a$ und $b$ betragen:
\begin{align*}
  -a &= R_i = \SI{16.3(2)}{\ohm} \\
  b &= U_0 = \SI{1.34(1)}{\volt}
\end{align*}

Der Fehler der Parameter wird mit Python berechnet.


\subsection{Bestimmung der Spannung und des Innenwiderstandes mit einer Gegenspannung}

In Tabelle 2 werden die Messwerte $U_k$ und $I$ bei angelegter Gegenspannung dargestellt.

\begin{table}[H]
  \centering
  \caption{Gemessene Spannungen und Stromstärken bei angelgter Gegenspannung}
  \label{tab:gegenspannung}
  \begin{tabular}{c c}
    \toprule
    $I/$mA & $U/$V \\
    \midrule
    49.0  2.60 \\
    46.0  2.55 \\
    29.0  2.35 \\
    23.0  2.25 \\
    30.0  2.30 \\
    35.0  2.40 \\
    40.0  2.45 \\
    55.0  2.70 \\
    70.0  2.95 \\
    80.0  3.13 \\
    65.0  2.90 \\
    \bottomrule
  \end{tabular}
\end{table}

Die Messwerte werden in einem Diagramm aufgetragen und eine lineare Regression wird durch geführt.

\begin{figure}[H]
  \centering
  \includegraphics{plot2.pdf}
  \caption{$U_k$ in Abhängigkeit von der Stromstärke bei angelegter Gegenspannung }
  \label{fig:plot2}
\end{figure}

Die Gerade wird durch die Gleichung $y = cx + d$ beschrieben. Die Parameter betragen:
\begin{align*}
  c &= R_i = \SI{15.7(4)}{\ohm} \\
  d &= U_0 = \SI{1.85(2)}{\volt}
\end{align*}

Der Fehler der Parameter wird mit Python berechnet.
