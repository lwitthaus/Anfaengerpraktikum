\section{Auswertung}
\label{sec:Auswertung}

Die gemessene Leerlaufspannung, der Monozelle beträgt $\SI{1.7}{\volt}$. Der Innenwiderstand dieser, beträgt $\SI{10}{\mega\ohm}$.

\subsection{Bestimmung der Leerlaufspannung und des Innenwiderstandes einer Monozelle}

In Tabelle 1 werden die gemessenen Spannungen $U_K$ und die zugehörigen Stromstärken $I$ dargestellt.

\begin{table}[H]
  \centering
  \caption{Gemessene Spannungen und Stromstärken}
  \label{tab:spannung1}
  \begin{tabular}{c c}
    \toprule
    $I/$mA & $U/$V \\
    \midrule
    80.8  &  0.05 \\
    75.0  &  0.10 \\
    70.0  &  0.20 \\
    65.0  &  0.26 \\
    60.0  &  0.35 \\
    50.0  &  0.53 \\
    40.0  &  0.70 \\
    30.0  &  0.85 \\
    20.0  &  1.01 \\
    15.0  &  1.10 \\
    \bottomrule
  \end{tabular}
\end{table}

Die Messwerte werden in einem Diagramm aufgetragen und eine lineare Regression wird durch geführt.


\begin{figure}[H]
  \centering
  \includegraphics{plot.pdf}
  \caption{Lineare Regression von $U_k$ in Abhängigkeit von der Stromstärke}
  \label{fig:plot}
\end{figure}

Die Gerade kann durch die Gleichung $y = ax + b$ beschrieben werden. Die Parameter $a$ und $b$ betragen:
\begin{align*}
  -a &= R_i = \SI{16.3(2)}{\ohm} \\
  b &= U_0 = \SI{1.34(1)}{\volt}
\end{align*}

Der Fehler der Parameter wird mit Python berechnet.


\subsection{Bestimmung der Spannung und des Innenwiderstandes mit einer Gegenspannung}

In Tabelle 2 werden die Messwerte $U_k$ und $I$ bei angelegter Gegenspannung dargestellt.

\begin{table}[H]
  \centering
  \caption{Gemessene Spannungen und Stromstärken bei angelegter Gegenspannung}
  \label{tab:gegenspannung}
  \begin{tabular}{c c}
    \toprule
    $I/$mA & $U/$V \\
    \midrule
    23.0  &  2.25 \\
    29.0  &  2.35 \\
    30.0  &  2.30 \\
    35.0  &  2.40 \\
    40.0  &  2.45 \\
    46.0  &  2.55 \\
    49.0  &  2.60 \\
    55.0  &  2.70 \\
    65.0  &  2.90 \\
    70.0  &  2.95 \\
    80.0  &  3.13 \\
    \bottomrule
  \end{tabular}
\end{table}

Die Messwerte werden in einem Diagramm aufgetragen und eine lineare Regression wird durch geführt.

\begin{figure}[H]
  \centering
  \includegraphics{plot2.pdf}
  \caption{Lineare Regression von $U_k$ in Abhängigkeit von der Stromstärke bei angelegter Gegenspannung }
  \label{fig:plot2}
\end{figure}

Die Gerade wird durch die Gleichung $y = cx + d$ beschrieben. Die Parameter betragen:
\begin{align*}
  c &= R_i = \SI{15.7(4)}{\ohm} \\
  d &= U_0 = \SI{1.85(2)}{\volt}
\end{align*}

Der Fehler der Parameter wird mit Python berechnet.


\subsection{Bestimmung der Klemmspannung und des Innenwiderstandes bei angelegter Wechselspannung}
Die gemessenen Spannungen und Stromstärken der Sinusspannungsquelle und der Rechteckspannungsquelle werden
in Tabelle 3 dargestellt.

\begin{table}[H]
  \centering
  \caption{Gemessene Spannungen und Stromstärken bei angelegter Wechselspannung}
  \label{tab:wechselspannung}
  \begin{tabular}{c c c c}
    \toprule
    \multicolumn{2}{c}{Rechteckspannung} & \multicolumn{2}{c}{Sinusspannung} \\
    \cmidrule(lr){1-2}\cmidrule(lr){3-4}
    $I/$mA & $U/$V & $I/$mA & $U/$V \\
    \midrule
    8.5  &  0.21  &  1.00  &   0.34 \\
    8.0  &  0.22  &  0.90  &   0.41 \\
    7.0  &  0.23  &  0.80  &   0.48 \\
    6.0  &  0.24  &  0.70  &   0.55 \\
    5.5  &  0.25  &  0.60  &   0.62 \\
    5.0  &  0.26  &  0.50  &   0.69 \\
    4.0  &  0.27  &  0.40  &   0.76 \\
    3.0  &  0.29  &  0.30  &   0.83 \\
    2.0  &  0.30  &  0.20  &   0.90 \\
    1.0  &  0.32  &  0.16  &   0.92 \\
    \bottomrule
  \end{tabular}
\end{table}

Die Messwerte werden jeweils in ein Diagramm eingetragen und es wird eine lineare Regression durchgeführt.

\begin{figure}[H]
  \centering
  \includegraphics{plot3.pdf}
  \caption{$U_k$ in Abhängigkeit von der Stromstärke bei angelegter Rechteckspannung}
  \label{fig:plot3}
\end{figure}

Die Gerade wird durch die Gleichung $y = ex + f$ beschrieben. Die Parameter $e$ und $f$ betragen:
\begin{align*}
  -e &= R_i = \SI{14.3(4)}{\ohm} \\
  f &= U_0 = \SI{0.331(2)}{\volt}
\end{align*}

Für die Rechteckspannung wird analog vorgegangen.

\begin{figure}[H]
  \centering
  \includegraphics{plot4.pdf}
  \caption{$U_k$ in Abhängigkeit von der Stromstärke bei angelegter Sinusspannung}
  \label{fig:plot4}
\end{figure}

Die Gerade wird durch die Gleichung $y = gx + h$ beschrieben. Die Parameter $g$ und $h$ betragen:
\begin{align*}
  g &= -R_i = \SI{696(3)}{\ohm}   \\
  h &= U_0 = \SI{1.037(2)}{\volt}
\end{align*}

\subsection{Systematische Fehler bei der einfachen Messung der Leerlaufspannung}
Zur genauen Messung der Leerlaufspannung muss der Eingangswiderstand $R_V$ möglichst groß sein.
Da dieser jedoch nicht unendlich groß sein kann, wird ein systematischer Fehler gemacht.
Nach Gleichung (2) ergibt sich für die tatsächliche Leerlaufspannung:
\begin{equation*}
  U_0 = U_k + \frac{U_kR_i}{R_V}
\end{equation*}
$U_k$ ist dabei nun die eigentlich gemessene Leerlaufspannung und für $R_i$ wird
der zuvor berechnete Innenwiderstand bei Gleichspannung eingesetzt.
Der Fehler entspricht dann der Differenz $U_0 - U_k$ und ergibt sich zu:
\begin{align*}
  \Delta U_0 &= \frac{U_kR_i}{R_V} \\
  \Delta U_0 &= (2,771 \pm 0,034)\cdot 10^{-6} \: \symup{V}
\end{align*}

Mit dem Voltmeter sollte die Spannung nicht hinter dem Amperemeter (Punkt H, siehe Abbildung \ref{fig:Schaltung}) abgegriffen
werden, da dessen Widerstand die gemessene Spannung ändert. Es würde also nicht mehr nur die
Klemmspannung gemessen werden.




\subsection{Bestimmung des Belastungswiderstandes und der Leistung}
In Tabelle 4 werden die Leistung und der Belastungswiderstand dargestellt.

\begin{table}[H]
  \centering
  \caption{Leistung und Belastungswiderstand}
  \label{tab:Leistung}
  \begin{tabular}{c c}
    \toprule
    $P/$mW & $R/ \symup{\Omega}$ \\
    \midrule
    4.0   &  0.6        \\
    7.5   &  1.3        \\
    14.0  &  2.9        \\
    16.9  &  4.0        \\
    21.0  &  5.8        \\
    26.5  &  10.6        \\
    28    &  17.5        \\
    25.5  &  28.3        \\
    20.2  &  50.5        \\
    16.5  &  73.3        \\
    \bottomrule
  \end{tabular}
\end{table}

Die Leistung $P=U_k \cdot I$ wird gegen den Belastungswiderstand $R_a = \frac{U_k}{I}$ aufgetragen. Zudem wird in dem
Diagramm die TheorieKruve für $P = \frac{U_0^2 \cdot R_a}{(R_a + R_i)^2}$ dargestellt.

\begin{figure}[H]
  \centering
  \includegraphics{plot5.pdf}
  \caption{Berechnete Werte und Theoriekurve der Leistung der Monozelle}
  \label{fig:plot5}
\end{figure}
