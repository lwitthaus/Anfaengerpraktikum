\section{Diskussion}
\label{sec:Diskussion}

Die ermittelte Leerlaufspannung der Monozelle weicht mit $21.2 \%$ signifikant von der anfangs bestimmten Leerlaufspannung ohne
Belastungswiderstand ab. Angesichts der Anzahl an Messwerten, sind statistische Fehler als primäre Fehlerquelle ausgeschlossen.
Dies kann an einer ungenauen Eichung des Amperemeters liegen. Ein Vergleichswert für
den Innenwiderstand gibt es, da die Messwerte präzise auf einer Geraden liegen, könnte der Widerstand
eine kleine Abweichung zu dem realen Wert betragen. Das $U_0$ aus der zweiten Messreihe weicht um $8.8\%$
von der anfänglich bestimmten Leerlaufspannung ab, wobei eine größere Abweichung der Messpunkte von der Regressionsgerade
erkennnbar ist. Als systematischer Fehler ist die gegenspannung zu nennen, welche nicht auf $U_0 +2\symup{V}$
einstellen kann.

Die Messpunkte bei der Rechteckspannung verhalten sich scheinbar nicht ganz linear, was in der theorei schon erklärt ist. Bei der
Sinusspannung ist nach der Theorie nicht zwingend ein linearer Zusammenhang gegeben, jedoch weisen die Messpunkte, welche
sehr präzise auf der Regressionsgeraden liegen, daraufhin.

Der berechnete systematische Fehler ist sehr klein, daraus lääst sich schließen, dass der Eingangswiderstand des
Voltmeter sehr groß ist, dass damit die Leerlaufspannung mit zuverlässiger Genauigkeit gemessen werden kann.

Die Messwerte der Leistung liegen im Rahmen der Messungenauigkeit auf der Theoriekurve. Systematische Fehler
sind bei der Rechnung unbeutend klein.
