\section{Auswertung}
\label{sec:Auswertung}
\subsection{Ermittlung der herausgefilterten Frequenz des Selektivfilters}
Um die Frequenz zu ermitteln, die der Selektivfilter herausfilter wird die Frequenz $\nu$ gegen
die zugehörige Spannung $U$ aufgetragen. Die Messwerte sind in Tabelle 1 dargestellt.

\begin{table}[H]
  \centering
  \caption{Gemessene Spannungen in Abhängigkeit von der Frequenz.}
  \label{tab:Rechteckspannung}
  \begin{tabular}{c c}
    \toprule
    $\nu/$kHz & $U/$mV \\
    \midrule
    20,0 &       4,3 \\
    22,0 &       5,4 \\
    24,0 &       6,8 \\
    26,0 &       8,9 \\
    28,0 &      10,5 \\
    30,0 &      16,0 \\
    31,0 &      20,7 \\
    32,0 &      28,5 \\
    33,0 &      37,5 \\
    33,5 &      52,0 \\
    34,0 &      73,0 \\
    34,5 &       101 \\
    35,0 &       305 \\
    35,5 &       315 \\
    36,0 &       124 \\
    36,5 &      81,0 \\
    37,0 &      55,0 \\
    37,5 &      42,0 \\
    38,0 &      33,0 \\
    38,5 &      26,0 \\
    39,0 &      22,0 \\
    40,0 &      16,0 \\
    \bottomrule
  \end{tabular}
\end{table}

In Abbildung 4 ist die Spannung in Abhängigkeit von der Frequenz dargestellt.

\begin{figure}[H]
  \centering
  \includegraphics{plot.pdf}
  \caption{Emissionsspektrum der Kupferröntgenröhre.}
  \label{fig:plot}
\end{figure}

Das Maximum liegt bei einer Frequenz von $35,5$kHz.


\subsection{Berechnung der magnetischen Suszeptibiliät anhand der gemessenen Daten}
Die in der Brückenschaltung eingebaute Spule hat die folgenden Abmessungen:

Länge $l = 135 \: \symup{mm}$

Querschnitt $F = 86,7 \: \symup{mm^2}$

Windungen $n = 250$

Widerstand $R = 0,7 \: \symup{\Omega}$

Ohne eine Probe in der Spule ergibt sich eine minimale Brückenspannung von
\begin{equation*}
  U_{br} = 13,5 \: \symup{mV}
\end{equation*}
bei einem Widerstand
\begin{equation*}
  R_3 = 4150 \: \symup{m\Omega} \: .
\end{equation*}

Die Widerstände $R_a$, bei denen die Brückenschaltung mit den entsprechenden Proben
in der Spule abgeglichen ist, und die daraus resultierenden Widerstandsänderungen $\Delta R$ können
Tabelle \ref{tab:Widerstand} entnommen werden. Zudem werden die Ausgangsspannungen der Brücke aufgeführt,
die sich ergeben, wenn die Proben bei vorher abgeglichener Brücke in Spule geschoben werden.
\begin{table}[H]
  \centering
  \caption{Messwerte zur Bestimmung der Suszeptibiliäten.}
  \label{tab:Widerstand}
  \begin{tabular}{c c c c c}
    \toprule
    Probe & $U_{neu} / \symup{mV}$ & $\Delta U / \symup{mV}$ &  $R_a / \symup{m\Omega}$   & $\Delta R / \symup{m\Omega}$  \\
    \midrule
    Dy & 21,0 & 7,5 & 2550 & 1600 \\
    Nd & 13,5 & 0,0 & 4000 & 150 \\
    Pr & 13,5 & 0,0 & 3850 & 300 \\
    Gd & 15,9 & 2,4 & 3300 & 850 \\
    \bottomrule
  \end{tabular}
\end{table}

Die Abmessungen der Proben können Tabelle \ref{tab:Dy} entnommen werden.
Dabei ist M die Masse, $\rho$ die Dichte, und L die Länge der Probe.

\begin{table}[H]
  \centering
  \caption{Abmessungen der Proben.}
  \label{tab:Dy}
  \begin{tabular}{c c c c}
    \toprule
    Probe & $M/ \symup{g}$ & $\rho / \symup{\frac{g}{cm^3}}$   & $L / \symup{mm}$  \\
    \midrule
    Dy & 15,1 & 7,8 & 135\\
    Nd & 9,0 & 7,24 & 135\\
    Gd & 14,8 & 7,40 & 135\\
    \bottomrule
  \end{tabular}
\end{table}



\subsubsection{Dysprosium}
Es ergibt sich nach
\begin{equation*}
  Q_{real} = \frac{M}{L \cdot \rho}
\end{equation*}
ein realer Querschnitt von
\begin{equation*}
  Q_{Dy} = 14,33 \: \symup{mm^2}
\end{equation*}

Mit Gleichung (19) ergibt sich für die Suszeptibilität:
\begin{equation*}
  \chi_{Dy_U} = 0,010146
\end{equation*}
Wobei beachtet werden muss, dass die Brückenspannung um einen Faktor $100$ verstärkt gemessen wird. Für die anderen
Proben gilt dies auch.

Mit Gleichung (20) ergibt sich eine Suszeptibiliät von
\begin{equation*}
  \chi_{Dy_R} = ? .
\end{equation*}

\subsubsection{Neodym}
Es ergibt sich ein realer Querschnitt von
\begin{equation*}
  Q_{Nd} = 9,21 \: \symup{mm^2}
\end{equation*}
Mit Gleichung (19) ergibt sich für die Suszeptibilität:
\begin{equation*}
  \chi_{Nd_U} = 0,010157
\end{equation*}

Mit Gleichung (20) ergibt sich eine Suszeptibiliät von
\begin{equation*}
  \chi_{Nd_R} = 0 .
\end{equation*}

\subsubsection{Gadolinium}
Es ergibt sich ein realer Querschnitt von
\begin{equation*}
  Q_{Gd} = 14,81 \: \symup{mm^2}
\end{equation*}
Mit Gleichung (19) ergibt sich für die Suszeptibilität:
\begin{equation*}
  \chi_{Gd_U} = 0,007817
\end{equation*}

Mit Gleichung (20) ergibt sich eine Suszeptibiliät von
\begin{equation*}
  \chi_{Gd_R} = ? .
\end{equation*}


\subsection{Theoretische Werte und Vergleich mit den berechneten Werten}

Um den theoretischen Wert der Suszeptibilität mit Gleichung (18) zu bestimmen wird der Land$\acute{\text{e}}$-Faktor $g_J$, sowie $\vec{J}$ und
die Anzahl $N= \frac{\rho}{M}$ der Momente pro Volumen benötigt. Dabei ist $M$ die molare Masse. Als Raumtemperatur wird bei der
Rechnung $296,15$K angenommen. Die Elemente Dy, Nd und Gd besitzen jeweils $9$, $3$ und $7$ Elektronen in der 4f-Schale. In Tabelle §
sind die aus den Hundschen Regeln folgende Werte dargestellt.

\begin{table}[H]
  \centering
  \caption{Gesamtspin, Bahndrehimpuls, Gesamtdrehimpuls und Land$\acute{\text{e}}$-Faktor der Proben}
  \label{tab:Dy}
  \begin{tabular}{c c c c c}
    \toprule
    Probe & $L$ & $S$ & $J$  & $g_J$  \\
    \midrule
    Dy & 5 & 2,5 & 7,5 & 1,33\\
    Nd & 6 & 1,5 & 4,5 & 0,73\\
    Gd & 0 & 3,5 & 3,5 & 2,00\\
    \bottomrule
  \end{tabular}
\end{table}

Für $N$ der drei Proben ergibt sich:
\begin{align*}
  &N_{Dy} = \SI{2.89e28}{\per\meter^3} \\
  &N_{Nd} = \SI{3.02e28}{\per\meter^3} \\
  &N_{Gd} = \SI{2.83e28}{\per\meter^3}
\end{align*}

Die theoretischen Werte der Suszeptibilitäten $\chi_T$, sowie die vorher berechneten Suszeptibilitäten
der Proben werden in Tabelle 5 dargestellt. Eine relative Abweichung der Werte wird ebenfalls angegeben.

\begin{table}[H]
  \centering
  \caption{Theoretische Werte und Vergleiche der Suszeptibilitäten}
  \label{tab:Dy}
  \begin{tabular}{c c c c c c}
    \toprule
    Probe & $\chi_T$ & $\chi_U$ & $\chi_R$ & $\frac{\chi_T - \chi_U}{\chi_T}/$\%  & $\frac{\chi_T - \chi_R}{\chi_T}/$\% \\
    \midrule
    Dy & 0.028705 & 0,010146& ? & -64,7 & ?\\
    Nd & 0.003508 & 0,010157& ? & 189,5 & ?\\
    Gd & 0.015704 & 0,007817& ? & -50,2 & ?\\
    \bottomrule
  \end{tabular}
\end{table}
