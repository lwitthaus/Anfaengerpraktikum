\section{Auswertung}
\label{sec:Auswertung}
\subsection{Ermittlung der herausgefilterten Frequenz des Selektivfilters}
Um die Frequenz zu ermitteln, die der Selektivfilter herausfilter wird die Frequenz $\nu$ gegen
die zugehörige Spannung $U$ aufgetragen. Die Messwerte werden in Tabelle 1 dargestellt.

\begin{table}[H]
  \centering
  \caption{Gemessene Spannungen in Abhängigkeit von der Frequenz.}
  \label{tab:Rechteckspannung}
  \begin{tabular}{c c}
    \toprule
    $\nu/$kHz & $U/$mV \\
    \midrule
    20,0 &       4,3 \\
    22,0 &       5,4 \\
    24,0 &       6,8 \\
    26,0 &       8,9 \\
    28,0 &      10,5 \\
    30,0 &      16,0 \\
    31,0 &      20,7 \\
    32,0 &      28,5 \\
    33,0 &      37,5 \\
    33,5 &      52,0 \\
    34,0 &      73,0 \\
    34,5 &       101 \\
    35,0 &       305 \\
    35,5 &       315 \\
    36,0 &       124 \\
    36,5 &      81,0 \\
    37,0 &      55,0 \\
    37,5 &      42,0 \\
    38,0 &      33,0 \\
    38,5 &      26,0 \\
    39,0 &      22,0 \\
    40,0 &      16,0 \\
    \bottomrule
  \end{tabular}
\end{table}

Abbildung 4 stellt die Spannung in Abhängigkeit von der Frequenz dar.
\includegraphics{plot.pdf}
\caption{Plot.}
\label{fig:plot}
\end{figure}

Das Maximum liegt bei einer Frequenz von $35,5$kHz.


\subsection{Berechnung der magnetischen Suszeptibiliät anhand der gemessenen Daten}
Die in der Brückenschaltung eingebaute Spule hat die folgenden Abmessungen:

Länge $l = 135 \: \symup{mm}$

Querschnitt $F = 86,7 \: \symup{mm^2}$

Windungen $n = 250$

Widerstand $R = 0,7 \: \symup{\Omega}$

Ohne eine Probe in der Spule ergibt sich eine minimale Brückenspannung von
\begin{equation*}
  U_{br} = 13,5 \: \symup{mV}
\end{equation*}
bei einem Widerstand
\begin{equation*}
  R_3 = 4150 \: \symup{m\Omega} \: .
\end{equation*}

Die Widerstände $R_a$, bei denen die Brückenschaltung mit den entsprechenden Proben
in der Spule abgeglichen ist, und die daraus resultierenden Widerstandsänderungen $\Delta R$ können
Tabelle \ref{tab:Widerstand} entnommen werden. Zudem werden die Ausgangsspannungen der Brücke aufgeführt,
die sich ergeben, wenn die Proben bei vorher abgeglichener Brücke in Spule geschoben werden.
\begin{table}[H]
  \centering
  \caption{Messwerte zur Bestimmung der Suszeptibiliäten.}
  \label{tab:Widerstand}
  \begin{tabular}{c c c c c}
    \toprule
    Probe & $U_{neu} / \symup{mV}$ & $\Delta U / \symup{mV}$ &  $R_a / \symup{m\Omega}$   & $\Delta R / \symup{m\Omega}$  \\
    \midrule
    Dy & 21,0 & 7,5 & 2550 & 1600 \\
    Nd & 13,5 & 0,0 & 4000 & 150 \\
    Pr & 13,5 & 0,0 & 3850 & 300 \\
    Gd & 15,9 & 2,4 & 3300 & 850 \\
    \bottomrule
  \end{tabular}
\end{table}

\subsubsection{Dysprosium}
Die Abmessungen der Probe können Tabelle \ref{tab:Dy} entnommen werden.
Dabei ist M die Masse, $\rho$ die Dichte, und L die Länge der Probe.
\begin{table}[H]
  \centering
  \caption{Abmessungen der Dysprosium-Probe.}
  \label{tab:Dy}
  \begin{tabular}{c c c}
    \toprule
    $M_{Dy} / \symup{g}$ & $\rho_{Dy} / \symup{\frac{g}{cm^3}}$   & $L_{Dy} / \symup{cm}$  \\
    \midrule
    15,1 & 7,8 & 13,5\\
    \bottomrule
  \end{tabular}
\end{table}
Daraus ergibt sich nach
\begin{equation*}
  Q_{real} = \frac{M}{L \cdot \rho}
\end{equation*}
ein realer Querschnitt von
\begin{equation*}
  Q_{Dy} = 14,33 \: \symup{mm^2}
\end{equation*}
Daraus ergibt sich nach Gleichung (20) eine Suszeptibiliät von
\begin{equation*}
  \chi_{Dy} = 0 \: \symup{cm^2}
\end{equation*}
