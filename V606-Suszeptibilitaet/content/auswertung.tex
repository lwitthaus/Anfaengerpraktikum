\section{Auswertung}
\label{sec:Auswertung}
\subsection{Ermittlung der herausgefilterten Frequenz des Selektivfilters}
Um die Frequenz zu ermitteln, die der Selektivfilter herausfilter wird die Frequenz $\nu$ gegen
die zugehörige Spannung $U$ aufgetragen. Die Messwerte werden in Tabelle 1 dargestellt.

\begin{table}[H]
  \centering
  \caption{Gemessene Spannungen in Abhängigkeit von der Frequenz.}
  \label{tab:Rechteckspannung}
  \begin{tabular}{c c}
    \toprule
    $\nu/$kHz & $U/$mV \\
    \midrule
    20,0 &       4,3 \\
    22,0 &       5,4 \\
    24,0 &       6,8 \\
    26,0 &       8,9 \\
    28,0 &      10,5 \\
    30,0 &      16,0 \\
    31,0 &      20,7 \\
    32,0 &      28,5 \\
    33,0 &      37,5 \\
    33,5 &      52,0 \\
    34,0 &      73,0 \\
    34,5 &       101 \\
    35,0 &       305 \\
    35,5 &       315 \\
    36,0 &       124 \\
    36,5 &      81,0 \\
    37,0 &      55,0 \\
    37,5 &      42,0 \\
    38,0 &      33,0 \\
    38,5 &      26,0 \\
    39,0 &      22,0 \\
    40,0 &      16,0 \\
    \bottomrule
  \end{tabular}
\end{table}

Abbildung 4 stellt die Spannung in Abhängigkeit von der Frequenz dar.

\begin{figure}
  \centering
  \includegraphics{plot.pdf}
  \caption{Plot.}
  \label{fig:plot}
\end{figure}

Das Maximum liegt bei einer Frequenz von $35,5$kHz.

\subsection{Berechnung der Suszeptibilität verschiedener Selten-Erd-Metalle}


Die Brückenspannung
