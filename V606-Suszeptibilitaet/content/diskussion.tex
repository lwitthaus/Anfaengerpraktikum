\section{Diskussion}
\label{sec:Diskussion}
Die berechneten Werte der Suseptibilität durch die beiden Methoden weichen von der
theoretischen Suszeptibilität deutlich ab. Dies kann durch die verwendeten Näherungen hervorgerufen werden.
Die Längen der Proben wurden auf die Länge der Spule genähert und die Raumtemperatur auf $23$°C geschätzt.
Zusätzlich war die Brückenschaltung nicht richtig kalibriert was sich auf die gemessenen Brückenspannungen
und Widerstände auswirkt.

Die Messkurve des Selektivfilters entspricht der zu erwarteten Kurve. Die berechnete Güte des Seletkitvverstärkers weicht deutlich von dem
zu erwartenden Wert ab. Dies lässt sich durch die geringe Anzahl an Messpunkten um das Maximum herum erklären, wodurch die genaue Frequenz des
Maximums nicht bestimmt werden kann.
