\section{Durchführung}
\label{sec:Durchführung}

Es wird die Schaltung einer Wheatstoneschen Brücke aufgebaut (siehe Abbildung \ref{fig:Wheat}).
Um die Brückenspannung zu messen, wird sie auf einem Oszilloskop ausgegeben.
Das Widerstandsverhältnis des Potentiometers wird solange variiert, bis eine minimale
Brückenspannung gemessen wird. Es werden die Beträge aller Widerstände in
dieser Verhältnislage notiert. Diese Messung wird zur späteren Fehlerbestimmung bei gleichem zu
vermessenden Widerstand $R_x$ mit drei unterschiedlichen Widerständen für $R_2$ durchgeführt.
Es werden mit diesem Verfahren insgesamt zwei Widerstände ($R_x$) vermessen.

Anschließend wird eine Schaltung entsprechend der Kapazitätsmessbrücke (siehe Abbildung \ref{fig:kapazität})
aufgebaut. Auch hier wird die Brückenspannung auf einem Oszilloskop ausgegeben.
Das Widerstandsverhältnis des Potentiometers und des Widerstands $R_2$ wird solange variiert, bis eine minimale
Brückenspannung gemessen wird. Dabei wird festgestellt, dass bei keiner der
vermessenen Kapazitäten (Wert 1 und Wert 3) der Einbau eines Widerstands $R_2$
überhaupt von Nöten ist, da dieser zur Minimierung der Brückenspannung ohnehin auf
den Wert Null eingestellt werden muss. Lediglich bei der RC-Kombination (Wert 8)
ist er unverzichtbar. Die Beträge aller Widerstände in den jeweiligen Verhältnislagen
werden notiert. Zur Fehlerbestimmung wird jede Messung (Wert 1, Wert 3, Wert 8)
mit drei unterschiedlichen Kapazitäten $C_2$ durchgeführt.

Nun wird die Schaltung einer Induktivitätsmessbrücke (siehe Abbildung \ref{fig:induktivitaet}) aufgebaut.
Als zu untersuchende Induktivität wird Wert 17 gewählt.
Die Brückenspannung wird wiederum auf einem Oszilloskop ausgegeben.
Das Widerstandsverhältnis von Potentiometer und Widerstand $R_2$ wird wieder
variiert um die diese zu minimieren. Die Widerstände der entsprechenden
Verhältnislage werden notiert. Hier wird die Messung zur Fehlerbestimmung mit
drei unterschiedlichen Induktivitäten $L_2$ durchgeführt.
Da bei der Verwendung bestimmter Induktivitäten $L_2$ keine verschwindende
Brückenspannung eingestellt werden konnte, wurden die Widerständsverhältnisse
bei ihrer höchstens zu erreichenden Minimallage notiert.

Dieselbe Induktivität wird noch einmal mit einer Maxwell-Brücke (siehe Abbildung \ref{fig:maxwell}) untersucht.
Hier werden zur Minimierung der erneut auf dem Oszilloskop ausgegebenen Brückenspannung
die Widerstände $R_3$ und $R_4$ variiert. Zur Fehlerbestimmung werden unterschiedliche
Widerstände $R_2$ verwendet.

Zuletzt wird dann die Frequenzabhängigkeit einer Wien-Robinson Brücke (siehe Abbildung \ref{fig:wien})
überprüft. Es wird eine entsprechende Schaltung aufgebaut. Die Frequenz der
Speisespannung $U_s$ wird über einen Frequenzbereich von 20 - 30000 Hz variiert.
Dabei werden bei 30 unterschiedlichen Frequenzwerten jeweils die Amplitude eben dieser
Speisespannung sowie die der Brückenspannung gemessen. Aus dieser Messung können
auch sämtliche zur Bestimmung des Klirrfaktors wichtigen Daten entnommen werden.
