\section{Diskussion}
\label{sec:Diskussion}

Die Zeitkonstanten die mit der frequenzabhängigen Amplitude und Phasenverschiebung berechnet werden, weichen mit $81,3\%$
deutlich von einander ab.
Die mit dem Entladevorgang ermittelte Zeitkonstante ist ungefähr so groß wie die Zeitkonstante die mit der
Phasenverschiebung ermittelt wird. Daraus wird geschlossen, dass die Zeitkonstante, welche mit der Amplitude
berechnet wird, großen systematischen Fehlern unterliegt.
Die Messwerte des Polarplots weichen deutlich von der Theoriekurve ab. Daraus wird geschlossen, da auch hier die
Amplituden verwendet werden, weist dies ebenfalls auf Fehler in der Messung der Spannungsamplituden. Für so
große Abweichungen, sind anhand der Anzahl an Messwerten, statistische Fehler als primäre Fehlerquelle auszuschließen.
Der Generatorinnenwiderstand ist eine systematische Fehlerquelle, welche die gemessenen Amplituden verfälscht.

Bei der Nutzung des RC-Kreises als Integrator ergeben sich die Spannungsformen
wie erwartet. Sie entsprechen den theoretisch errechneten Funktionen.
Aus einer Rechteckspannung ergibt sich eine Dreieckspannung, aus einer
Sinusspannung eine Cosinusspannung und aus einer Dreieckspannung eine
quadratische Spannung. Dies ist auch auf den erstellten Thermodrucken zu erkennen.
Dabei ist die Amplitude der Kondensatorspannung aufgrund der hohen Frequenz wesentlich
geringer als die der Eingangsspannung. Der RC-Kreis ist also gut als Integrator
geeignet.
