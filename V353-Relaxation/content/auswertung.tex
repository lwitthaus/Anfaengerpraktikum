\section{Auswertung}
\label{sec:Auswertung}

\subsection{Fehlerrechnung}
Der Mittelwert eines Datensatzes mit $N$ Werten ist definiert durch:
\begin{equation}
  \bar{x} = \frac{1}{N} \sum_{i=1}^N x_i
\end{equation}
Die Standardabweichung eines Datensatzes von seinem Mittelwert durch:
\begin{equation}
  \sigma = \sqrt{\frac{1}{N(N-1)} \sum_{i=1}^N (x_i - \bar{x})}
\end{equation}
Pflanzen sich Unsicherheiten fort, wird der Fehler mit der gaußschen
Fehlerfortpflanzung berechnet:
\begin{equation}
  \sigma_f = \sqrt{
      \sum\limits_{i = 1}^N
       \left( \frac{\partial f}{\partial x_i} \sigma_i \right)^{\!\! 2}
     }
\end{equation}
Der Fehler der Steigung $m$ und des Achsenabschnittes $b$ einer linearen Regression
wird wie folgt berechnet:
\begin{align}
  \sigma_m &= \frac{\overline{xy}-\bar{x}\cdot\bar{y}}{\bar{x^2}-\bar{x}^2} \\
  \sigma_b &= \frac{\bar{x^2}\bar{y}-\overline{xy}\bar{x}}{\bar{x^2}-\bar{x}^2}
\end{align}


\subsection{Entladevorgang des Kondensators}
In Tabelle 1 sind die gemessenen Spannungen, bei dem Entladevorgang des Kondensators, zu den jeweiligen Zeitpunkten dargestellt. Zudem
ist der Bruch $\frac{U}{U_0}$ tabelliert, welcher für weitere Rechnungen nötig ist. Die Generatorpannung beträgt
dabei $U_0 = 3.30$V.

\begin{table}[H]
  \centering
  \caption{Gemessene Spannungen bei dem Entladen eines Kondensators}
  \label{tab:Rechteckspannung}
  \begin{tabular}{c c c}
    \toprule
    $t/\left(10^{-3}\symup{s}\right)$ & $U_C/$V & $\ln{\frac{U_C}{U_0}}$ \\
    \midrule
    0.5 & 2.36 & -0.33 \\
    1.0 & 1.62 & -0.71 \\
    1.5 & 1.08 & -1.11 \\
    2.0 & 0.72 & -1.51 \\
    2.5 & 0.48 & -1.90 \\
    3.0 & 0.32 & -2.30 \\
    3.5 & 0.18 & -3.00 \\
    4.0 & 0.12 & -3.22 \\
    4.5 & 0.06 & -3.91 \\
    5.0 & 0.04 & -4.61 \\
    \bottomrule
  \end{tabular}
\end{table}

Aus Gleichnung \eqref{eqn:Entladung} folgt:
\begin{equation}
  \ln{\frac{U}{U_0}} = -\frac{1}{RC}t.
\end{equation}
Der Logarithmus von $\frac{U}{U_0}$ wird gegen die Zeit aufgetragen und es wird eine lineare Regression durchgeführt.



\begin{figure}[H]
  \centering
  \includegraphics{plot1.pdf}
  \caption{Logarithmus der Kondensatorspannung zu dem Zeitpunkt t}
  \label{fig:entladung}
\end{figure}

Die Gerade kann durch die Gleichung $y = -\frac{1}{m}x + b$ beschrieben werden. Für die Parameter ergeben sich:
\begin{align*}
  -\frac{1}{m} = &RC = (1.08 \pm 0.04) \, \symup{s} \\
  &b = (0.28 \pm 0.11)
\end{align*}

Der Fehler der Parameter wird mit Gleichung (18) und (19) berechnet.

\subsection{Bestimmung der Zeitkonstante mit frequenzabhängiger Amplitude und Phasenverschiebung}

In Tabelle 2 ist die Amplitude $A$ und Phasenverschiebung für verschiedene Frequenzen dargestellt. Zudem ist der
Abstand der Nulldurchgänge $a$ tabelliert. Die Generatorspannung beträgt $U_0 = 3.60$V.

\begin{table}[H]
  \centering
  \caption{Gemessene Amplituden und Phasenverschiebungen der Kondensatorspannung}
  \label{tab:amplitude}
  \begin{tabular}{c c c c}
    \toprule
    $\omega/$Hz & $A/$V & $a/\left(10^{-3}\symup{s}\right)$ & $\varphi /°$ \\
    \midrule
    30   &  3.40 &    1.00 &  10.80 \\
    70   &  3.00 &    1.20 &  30.24 \\
    100  &  2.68 &    1.20 &  43.20 \\
    300  &  1.24 &    0.60 &  64.80 \\
    700  &  0.56 &    0.31 &  78.12 \\
    1000 &  0.40 &    0.22 &  79.20 \\
    2000 &  0.20 &    0.12 &  86.40 \\
    3000 &  0.13 &    0.08 &  86.40 \\
    4000 &  0.10 &    0.06 &  86.40 \\
    5000 &  0.08 &    0.05 &  90.00 \\
    \bottomrule
  \end{tabular}
\end{table}

Die Spannungsamplitude $\frac{A}{U_0}$ wird gegen $\ln{\omega}$ aufgetragen. Es wird eine lineare
Regression durchgeführt.

\begin{figure}[H]
  \centering
  \includegraphics{plot2.pdf}
  \caption{Amplitude der Kondensatorspannung in Abhängkeit der Frequenz}
  \label{fig:amplitude
\end{figure}


\subsection{Nutzung des RC-Kreises als Integrator}

\begin{figure}[H]
  \centering
  \includegraphics{MAP001.png}
  \caption{Eingang: Rechteckspannung}
  \label{fig:Rechteckspannung}
\end{figure}

Bei einer Rechteckspannung ergibt sich eine Dreieckspannung als integrierte Spannung.
Die steigende Flanke der Dreieckspannung ist dort, wo die Rechteckspannung einen positiven Wert hat.
Die fallende Flanke der Dreieckspannung ist dort, wo die Rechteckspannung einen negativen Wert hat.
Die Maxima und Minima der Dreieckspannung sind dort, wo die Rechteckspannung eine Sprungstelle hat.

\begin{figure}[H]
  \centering
  \includegraphics{MAP002.png}
  \caption{Eingang: Sinusspannung }
  \label{fig:Sinusspannung}
\end{figure}

Bei einer Sinusspannung ergibt sich eine Cosinusspannung als integrierte Spannung.

\begin{figure}[H]
  \centering
  \includegraphics{MAP003.png}
  \caption{Eingang: Dreieckspannung}
  \label{fig:Dreieckspannung}
\end{figure}

Bei einer Dreieckspannung ergibt sich eine Cosinusspannung als integrierte Spannung.
