\section{Auswertung}
\label{sec:Auswertung}


\subsection{Messung von verrauschten Signalen in Abhängigkeit von der Phase}

Die gemessenen Spannungen mit und ohne Rauschen, in Abhängigkeit von der Phase $\phi$, werden in Tabelle 1 dargestellt.

\begin{table}[H]
  \centering
  \caption{Spannungen in Abhängigkeit von der Phase}
  \label{tab:Phase}
  \begin{tabular}{c c c}
    \toprule
    $\phi/$grad  &  $U_{out}/$mV &  $U_v/$mV \\
    \midrule
    0     &  -21,8   &  -22,0 \\
    30    &  -29,7   &  -30,0 \\
    60    &  -30,2   &  -30,5 \\
    90    &  -23,4   &  -23,4 \\
    120   &  -11,9   &  -11,4 \\
    150   &   11,4   &   12,2 \\
    180   &   23,3   &   23,5 \\
    210   &   31,1   &   31,7 \\
    240   &   31,4   &   32,1 \\
    270   &   24,8   &   25,0 \\
    300   &   13,1   &   12,9 \\
    \bottomrule
  \end{tabular}
\end{table}

Die Messwerte der nichtverrauschten Spannungen werden gegen die Phase aufgetragen. Eine Ausgleichsfunktion mit der Form
$A \cos{\phi + B} + C$ mit den Parametern $A, B ,C$ wird erstellt und ist in Abbildung \ref{fig:plot} zu sehen.

(Alle Plots werden mit Python erstellt und alle Fehler mit Python berechnet.)



\begin{figure}[H]
  \centering
  \includegraphics{plot.pdf}
  \caption{Nicht verrauschte Spannungen in Abhängigkeit von der Phase}
  \label{fig:plot}
\end{figure}

Die Parameter betragen:
\begin{align*}
  A &= \SI{-32.6(8)}{\milli\volt} \\
  B &= \SI{33.72(2)}{} \\
  C &= \SI{1.0(6)}{\milli\volt}
\end{align*}


Mit dem verrauschtem Signal wird analog vorgegangen. Die Messwerte und die Ausgleichsfunktion ist sind in Abbildung \ref{fig:plot2} dargestellt.

\begin{figure}[H]
  \centering
  \includegraphics{plot2.pdf}
  \caption{Nicht verrauschte Spannungen in Abhängigkeit von der Phase}
  \label{fig:plot2}
\end{figure}


Die Parameter der Funktion $D \cos{\phi + E} + F$ betragen:
\begin{align*}
  D &= \SI{32.9(8)}{\milli\volt} \\
  E &= \SI{33.73(3)}{} \\
  F &= \SI{1.2(6)}{\milli\volt}
\end{align*}



\subsection{Messung eines verrauschten Lichtsignals}
