\section{Diskussion}
\label{sec:Diskussion}

In Tabelle 3 werden die wichtigsten Ergebnisse dargestellt.
\begin{table}[H]
  \centering
  \caption{Wichtige Ergebnisse}
  \label{tab:Phase}
  \begin{tabular}{c c}
    \toprule
    Parameter & Werte \\
    \midrule
    A & $\SI{-32.6(8)}{\milli\volt}$ \\
    D & $\SI{-32.9(8)}{\milli\volt}$ \\
    B & $-1,67 \pm 0.07$ \\
    \bottomrule
  \end{tabular}
\end{table}



Anhand der erstellten Bilder ist zu erkennen, dass die Modulation der Signalspannung mit der Referenzspannung gut funktioniert. Dies bestätigen
die Werte $A$ und $B$, welche innerhalb ihrer Standardabweichung liegen.
Die auf jeweils eine Seite umgeklappten Halbwellen sind problemlos erkennbar, weshalb sich diese mithilfe des
Tiefpasses zu einer entsprechenden Ausgangsspannung mitteln lassen sollten. Diese Annahme wird auch durch die erstellten
Graphen unterstützt.
Die Graphen der Spannung mit und ohne Rauschen zeigen nämlich einen nahezu unveränderten Verlauf. Daraus wird geschlossen, dass der
Lock-In-Verstärker das Rauschsignal effektiv unterdrücken kann.

Bei der Messung der Lichtintensität weicht der berechnete Wert $B=1,2$ deutlich von dem erwarteten Wert ab. Zusätzlich ist die Abweichung
größer als der Wert, was den Wert sehr ungenau werden lässt.
Auffällig sind die ersten 6 Messwerte, da ohne diese, sich für $B$ der Wert $B= \SI{1.6(1)}{}$ ergibt. Statistische Fehler sind
als primäre Fehlerquelle auszuschließen, da dafür der Fehler zu groß ist. Das Oszilloskop zeigt eine schwankende Spannung, welche
ein optimales Ablesen der Spannung nicht ermöglicht. Zudem wird bei Schaltung der Preamplifier überbrückt, da die Schaltung sonst nicht
funktioniert. Da dieser das Signal verstärken und somit die Messung der Spannung verbessern soll, ist das Weglassen des
Preamplifiers ein möglicher Grund für die große Abweichung.
