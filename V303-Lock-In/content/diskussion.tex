\section{Diskussion}
\label{sec:Diskussion}




Die Graphen der Spannung mit und ohne Rauschen, zeigt ein nahezu unveränderten Verlauf. Daraus wird geschlossen, dass der
Lock-In-Verstärker das Rauschsignal effektiv unterdrücken kann.

Bei der Messung der Lichtintensität weicht der Berechnete Wert $B=1,2$ deutlich von dem erwarteten Wert ab. Auffällig sind dabei
die ersten 6 Messwerte, da ohne diese, sich für $B$ der Wert $B= \SI{1.6(1)}{}$ ergibt. Statistische Fehler sind
als primäre Fehlerquelle auszuschließen, da dafür der Fehler zu groß ist. Das Oszilloskop zeigt eine schwankende Spannung, welche
eine optimales ablesen der Spannung nicht ermöglicht.
