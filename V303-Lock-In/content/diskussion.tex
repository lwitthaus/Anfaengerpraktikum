\section{Diskussion}
\label{sec:Diskussion}




Die Graphen der Spannung mit und ohne Rauschen, zeigt ein nahezu unveränderten Verlauf. Daraus wird geschlossen, dass der
Lock-In-Verstärker das Rauschsignal effektiv unterdrücken kann.

Bei der Messung der Lichtintensität weicht der Berechnete Wert $B=1,2$ deutlich von dem erwarteten Wert ab. Auffällig sind dabei
die ersten 6 Messwerte, da ohne diese, sich für $B$ der Wert $B= \SI{1.6(1)}{}$ ergibt. Statistische Fehler sind
als primäre Fehlerquelle auszuschließen, da dafür der Fehler zu groß ist. Das Oszilloskop zeigt eine schwankende Spannung, welche
ein optimales Ablesen der Spannung nicht ermöglicht. Zudem wird bei Schaltung der Pre-amplifier überbrückt, da die Schaltung sonst nicht
funktioniert. Da dieser das Signal verstärken und somit die Messung der Spannung verbessern soll, ist das Weglassen des 
Pre-amplifiers ein möglicher Grund für die Große Abweichung.
