\section{Diskussion}
\label{sec:Diskussion}

Die geschätzte Totzeit weicht von der gemessenen Totzeit deutlich ab. Dies lässt sich durch die qualitative
Bestimmung der Totzeit durch das Oszilloskop erklären, da nur eine ungenaue Schätzung möglich ist. Somit ist
davon auszugehen, dass die berechnete Totzeit näher an dem tatsächlichen Wert liet als die abgeschätzte.


Aus Abbildung 7 lässt sich ein proportionaler zusammenhang zwischen freigesetzter Ladung und Spannung erkennen. Der Größenbrereich entspricht
dem Bereich IV in Abbildung 2, was zu erwarten ist, da in diesem Bereich das Geiger-Müller-Zählrohr hauptsächlich arbeitet.
