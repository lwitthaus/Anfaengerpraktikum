\section{Diskussion}
\label{sec:Diskussion}

In der gemessenen Charakteristik des Zählrohrs ist deutlich eine Plateau-Steigung zu erkennen. Dies war auch zu
erwarten, da es sich natürlich nicht um ein ideales Geiger-Müller-Zählrohr handeln kann. Trotzdem ist das Plateau
noch als ein solches erkennbar. Die Steigung ist also nicht unerwartet groß, was darauf schließen lässt, dass die
restlichen Versuchsteile bei einer entsprechenden Spannung in dessen Bereich gut durchführbar sind.

Die geschätzte Totzeit weicht von der gemessenen Totzeit deutlich ab. Dies lässt sich durch die qualitative
Bestimmung der Totzeit durch das Oszilloskop erklären, da nur eine ungenaue Schätzung möglich ist. Somit ist
davon auszugehen, dass die berechnete Totzeit näher an dem tatsächlichen Wert liegt als die abgeschätzte.


Aus Abbildung 7 lässt sich ein proportionaler Zusammenhang zwischen freigesetzter Ladung und Spannung erkennen. Der Größenbereich entspricht
dem Bereich IV in Abbildung 2, was zu erwarten ist, da in diesem Bereich das Geiger-Müller-Zählrohr hauptsächlich arbeitet.
