\section{Auswertung}
\label{sec:Auswertung}

\subsection{Charakteristik}
In Tabelle 1 ist die Anzahl an gemessenen Teilchen $N$ in Abhängigkeit von der Spannung  $U$ dargestellt. Zusätzlich
ist dort auch die gemessene Stromstärke $I$ angegeben.

\begin{table}[H]
  \centering
  \caption{Stromstärke und Anzahl an gemessenen Teilchen in Abhängigkeit von der Spannung.}
  \label{tab:Rechteckspannung}
  \begin{tabular}{c c c}
    \toprule
    $U/$V & $N$ & $I/\symup{\mu}$A \\
    \midrule
    300 &	0	    & 0  \\
    310 &	16104 &	0  \\
    320 &	16789 &	0,5 \\
    330 &	16869 &	1,0 \\
    340 &	16774 &	1,0 \\
    350 &	17052 &	1,2 \\
    360 &	16952 &	1,4 \\
    370 &	17137 &	1,6 \\
    380 &	17248 &	1,9 \\
    390 &	17137 &	2,0 \\
    400 &	17237 &	2,1 \\
    410 &	17092 &	2,3 \\
    420 &	17377 &	2,4 \\
    430 &	17272 &	2,6 \\
    440 &	17092 &	2,9 \\
    450 &	17510 &	3,0 \\
    460 &	17521 &	3,0 \\
    470 &	17576 &	3,2 \\
    480 &	17418 &	3,5 \\
    490 &	17401 &	3,7 \\
    500 &	17547 &	4,0 \\
    510 &	17465 &	4,0 \\
    520 &	17400 &	4,2 \\
    530 &	17347 &	4,4 \\
    540 &	17485 &	4,5 \\
    550 &	17601 &	4,7 \\
    560 &	17659 &	5,0 \\
    570 &	17465 &	5,2 \\
    580 &	17750 &	5,4 \\
    590 &	17782 &	5,6 \\
    600 &	17732 &	5,9 \\
    610 &	17823 &	6,0 \\
    620 &	18170 &	6,2 \\
    630 &	18016 &	6,5 \\
    640 &	17992 &	6,7 \\
    650 &	18008 &	7,0 \\
    660 &	18123 &	7,0 \\
    670 &	18441 &	7,2 \\
    680 &	18425 &	7,3 \\
    690 &	18241 &	7,5 \\
    700 &	18646 &	7,7 \\
    \bottomrule
  \end{tabular}
\end{table}

Aus diesen Messwerten ergibt sich das folgende Diagramm

\begin{figure}
  \centering
  \includegraphics{plot.pdf}
  \caption{Zählraten in Abhängigkeit der Spannung mit linearer Regression im Plateau-Bereich.}
  \label{fig:plot}
\end{figure}

Dabei wurde die lineare Regrssion nur für den Bereich des Plateaus durchgeführt. Die ersten beiden
Messwerte werden nicht berücksichtigt.
Es ergibt sich eine Steigung von von:
\begin{equation*}
  m = (641,6 \pm 37,0) \: \symup{\frac{\%}{100V}}
\end{equation*}








\subsection{Bestimmung der Totzeit}

In Tabelle 2 wird Zählrate $N$ für 2 Beta-Strahler, sowie die Zählrate, wenn beide Strahler gleichzeitig
eingesetzt werden.

\begin{table}[H]
  \centering
  \caption{Anzahl an gemessenen Teilchen pro Sekunde in Abhängigkeit von dem Strahler}
  \label{tab:Rechteckspannung}
  \begin{tabular}{c c}
    \toprule
    Strahler & $N$  \\
    \midrule
    1. Strahler & 310,5 \\
    2. Strahler & 238,1\\
    Beide Strahler & 535,7 \\
    \bottomrule
  \end{tabular}
\end{table}

Mit Gleichung (1) lässt sich aus diesen Werten die ungefähre Totzeit bestimmen.
\begin{align*}
  T_1 \approx \SI{8.72e-5}{\second} = \SI{87,2}{\micro\second}
\end{align*}

Die mit dem Oszilloskop abgeschätzte Totzeit beträgt $T_2 = 50 \symup{\mu}$s.
Die Abweichung der beiden Totzeiten beträgt $42,7 \%$.


\subsection{Bestimmung der freigesetzten Ladungsmenge pro Teilchen}
