\section{Auswertung}
\label{sec:Auswertung}

\subsection{Charakteristik}
In Tabelle 1 ist die Anzahl an gemessenen Teilchen $N$ in Abhängigkeit von der Spannung  $U$ dargestellt. Zusätzlich
ist dort auch die gemessene Stromstärke $I$ angegeben. Die Fehler von $N$ betragen dabei $\sqrt{N}$, da diese nach der
Poissonverteilung verteilt sind.

\begin{table}[H]
  \centering
  \caption{Stromstärke und Anzahl an gemessenen Teilchen in Abhängigkeit von der Spannung.}
  \label{tab:Rechteckspannung}
  \begin{tabular}{c c c}
    \toprule
    $U/$V & $N$ & $I/\symup{\mu}$A \\
    \midrule
    300 &	0	    & 0  \\
    310 &	16104 $\pm$ 130 &	0  \\
    320 &	16789 $\pm$ 130 &	0,5 \\
    330 &	16869 $\pm$ 130 &	1,0 \\
    340 &	16774 $\pm$ 130 &	1,0 \\
    350 &	17052 $\pm$ 140 &	1,2 \\
    360 &	16952 $\pm$ 140 &	1,4 \\
    370 &	17137 $\pm$ 140 &	1,6 \\
    380 &	17248 $\pm$ 140 &	1,9 \\
    390 &	17137 $\pm$ 140 &	2,0 \\
    400 &	17237 $\pm$ 140 &	2,1 \\
    410 &	17092 $\pm$ 140 &	2,3 \\
    420 &	17377 $\pm$ 140 &	2,4 \\
    430 &	17272 $\pm$ 140 &	2,6 \\
    440 &	17092 $\pm$ 140 &	2,9 \\
    450 &	17510 $\pm$ 140 &	3,0 \\
    460 &	17521 $\pm$ 140 &	3,0 \\
    470 &	17576 $\pm$ 140 &	3,2 \\
    480 &	17418 $\pm$ 140 &	3,5 \\
    490 &	17401 $\pm$ 140 &	3,7 \\
    500 &	17547 $\pm$ 140 &	4,0 \\
    510 &	17465 $\pm$ 140 &	4,0 \\
    520 &	17400 $\pm$ 140 &	4,2 \\
    530 &	17347 $\pm$ 140 &	4,4 \\
    540 &	17485 $\pm$ 140 &	4,5 \\
    550 &	17601 $\pm$ 140 &	4,7 \\
    560 &	17659 $\pm$ 140 &	5,0 \\
    570 &	17465 $\pm$ 140 &	5,2 \\
    580 &	17750 $\pm$ 140 &	5,4 \\
    590 &	17782 $\pm$ 140 &	5,6 \\
    600 &	17732 $\pm$ 140 &	5,9 \\
    610 &	17823 $\pm$ 140 &	6,0 \\
    620 &	18170 $\pm$ 140 &	6,2 \\
    630 &	18016 $\pm$ 140 &	6,5 \\
    640 &	17992 $\pm$ 140 &	6,7 \\
    650 &	18008 $\pm$ 140 &	7,0 \\
    660 &	18123 $\pm$ 140 &	7,0 \\
    670 &	18441 $\pm$ 140 &	7,2 \\
    680 &	18425 $\pm$ 140 &	7,3 \\
    690 &	18241 $\pm$ 140 &	7,5 \\
    700 &	18646 $\pm$ 140 &	7,7 \\
    \bottomrule
  \end{tabular}
\end{table}

Aus diesen Messwerten ergibt sich das folgende Diagramm. Der erste Messwert wird
dabei nicht aufgetragen, da er für die Analyse des Plateaus irrelevant ist.

\begin{figure}
  \centering
  \includegraphics{plot.pdf}
  \caption{Zählraten in Abhängigkeit von der Spannung mit linearer Regression im Plateau-Bereich.}
  \label{fig:plot}
\end{figure}

Dabei wird die lineare Regression nur für den Bereich des Plateaus durchgeführt. Die ersten beiden
Messwerte werden nicht berücksichtigt. Das Plateau verläuft über einen Bereich von 380V (320V - 700V).
Es ergibt sich eine Steigung von:
\begin{equation*}
  m = (2,9 \pm 0,3) \: \symup{\frac{\%}{100V}}
\end{equation*}




\subsection{Bestimmung der Totzeit}

In Tabelle 2 wird Zählrate $N$ für 2 Beta-Strahler, sowie die Zählrate, wenn beide Strahler gleichzeitig
eingesetzt werden. Die Fehler von $N$ betragen hierbei wieder $\sqrt{N}$

\begin{table}[H]
  \centering
  \caption{Anzahl an gemessenen Teilchen pro Sekunde in Abhängigkeit von dem Strahler}
  \label{tab:Rechteckspannung}
  \begin{tabular}{c c}
    \toprule
    Strahler & $N$  \\
    \midrule
    1. Strahler & 311 $\pm$ 2 \\
    2. Strahler & 238 $\pm$ 2\\
    Beide Strahler & 536 $\pm$ 3\\
    \bottomrule
  \end{tabular}
\end{table}

Mit Gleichung (1) lässt sich aus diesen Werten die ungefähre Totzeit bestimmen.
\begin{align*}
  T_1 \approx \SI{9(3)e-5}{\second} = \SI{90(30)}{\micro\second}
\end{align*}

Der Fehler von $T$ wird mit der Gauß'schen Fehlerfortpflanzung bestimmt.
\begin{align*}
  \sigma_T = \sqrt{
      \sum\limits_{i = 1}^N
       \left( \frac{\partial f}{\partial x_i} \sigma_i \right)^{\!\! 2}
     }
    = \sqrt{\sigma_{a}^{2} \left(\frac{1}{2 a b} - \frac{a + b - c}{2 a^{2} b}\right)^{2}
  + \sigma_{b}^{2} \left(\frac{1}{2 a b} - \frac{a + b - c}{2 a b^{2}}\right)^{2} + \frac{\sigma_{c}^{2}}{4 a^{2} b^{2}}}
\end{align*}

Hierbei ist $a=N_1$, $b=N_2$ und $c=N_{1+2}$.
Die mit dem Oszilloskop abgeschätzte Totzeit beträgt $T_2 = 50 \, \symup{\mu} $s.
Die Abweichung der beiden Totzeiten voneinander beträgt $44,4 \, \%$.

Die geschätzte Erholungszeit beträgt $0,8 \,$ms.


\subsection{Bestimmung der freigesetzten Ladungsmenge pro Teilchen}

Die Ladung $\Delta Q$ wird mit Gleichung (2) bestimmt. In Tabelle 3 werden die Werte für Q und die zugehörigen Spannungen dargestellt.

\begin{table}[H]
  \centering
  \caption{Ladung in Abhängigkeit von der Spannung}
  \label{tab:Rechteckspannung}
  \begin{tabular}{c c}
    \toprule
    $U/$V & $\Delta Q/e_0 \cdot \symup{10^{10}}$ \\
    \midrule
      300 &                  0   $\pm 0$ \\
      310 &                  0   $\pm 0$ \\
      320 &        1,12    $\pm$0,01 \\
      330 &        2,22    $\pm$0,02 \\
      340 &        2,24    $\pm$0,02 \\
      350 &        2,64    $\pm$0,02 \\
      360 &        3,10    $\pm$0,03 \\
      370 &        3,50    $\pm$0,03 \\
      380 &        4,13    $\pm$0,04 \\
      390 &        4,38    $\pm$0,04 \\
      400 &        4,57    $\pm$0,04 \\
      410 &        5,05    $\pm$0,04 \\
      420 &        5,18    $\pm$0,04 \\
      430 &        5,64    $\pm$0,05 \\
      440 &        6,36    $\pm$0,05 \\
      450 &        6,42    $\pm$0,05 \\
      460 &        6,42    $\pm$0,05 \\
      470 &        6,83    $\pm$0,06 \\
      480 &        7,54    $\pm$0,06 \\
      490 &        7,97    $\pm$0,06 \\
      500 &        8,55    $\pm$0,07 \\
      510 &        8,59    $\pm$0,07 \\
      520 &        9,05    $\pm$0,07 \\
      530 &        9,51    $\pm$0,08 \\
      540 &        9,65    $\pm$0,08 \\
      550 &        10,01   $\pm$0,08 \\
      560 &        10,62   $\pm$0,08 \\
      570 &        11,17   $\pm$0,09 \\
      580 &        11,41   $\pm$0,09 \\
      590 &        11,81   $\pm$0,09 \\
      600 &        12,48   $\pm$0,10 \\
      610 &        12,62   $\pm$0,10 \\
      620 &        12,80   $\pm$0,10 \\
      630 &        13,53   $\pm$0,10 \\
      640 &        13,96   $\pm$0,11 \\
      650 &        14,58   $\pm$0,11 \\
      660 &        14,48   $\pm$0,11 \\
      670 &        14,64   $\pm$0,11 \\
      680 &        14,86   $\pm$0,11 \\
      690 &        15,42   $\pm$0,12 \\
      700 &        15,49   $\pm$0,12 \\
    \bottomrule
  \end{tabular}
\end{table}

Der Fehler von $\Delta Q/e_0$ wird dabei durch mit der Fehlerfortpflanzung berechnet. Die Größe $Z$ hat einen Fehler von $\sqrt{Z}$.
Die Fehlerfortpflanzung sieht dabei wie folgt aus:
\begin{equation*}
  \sigma_{\Delta Q/e_0} =\sqrt{ \frac{\overline{I}^{2} \sigma_{Z}^{2}}{Z^{4}} \Delta t^{2}}
\end{equation*}

Die Ladung wird in Einheiten der Elementarladung $e$ gegen die Spannung aufgetragen.

\begin{figure}
  \centering
  \includegraphics{rechnung.pdf}
  \caption{Freigesetzte Ladung aufgetragen gegen die Spannung}
  \label{fig:plot}
\end{figure}
