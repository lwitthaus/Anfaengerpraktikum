\section{Diskussion}
\label{sec:Diskussion}

Die Theoriekurve des Einzelspalts stimmt mit den experimentell gemessenen Werten überein. Die Abweichung der berechneten Spaltbreite weicht lediglich
$6,8\,$\% von dem Literaturwert ab. Diese Abweichung lässt sich durch systematische Fehler bei der Messung erklären. Ein Skalenwechsel des
Strommessgerätes verändert die Messwerte. Der Dunkelstrom ist über die Zeit nicht konstant, jedoch ist dieser im Vergleich zu
den Messwerte sehr klein und nimmt deswegen keinen signifikanten Einfluss auf die Messwerte. Es können noch kleine
statistische Abweichungen auftreten, welche Einfluss auf den Fit haben, diese Fehler spielen mit hoher Wahrscheinlichkeit nur eine untergeordnete
Rolle, da das Beugungsbild des Einzelspalts deutlich aus den Messwerten hervorgeht.

Die Messwerte für den ersten Doppelspalt nehmen ebenfalls die Form des bekannten Beugungsbildes eines Doppelspalts an. Der Fit konnte jedoch nicht
optimal an die Messwerte angepasst werden und beschreibt das Beugungsbild nur in der Umgebung des Hauptmaximums bis genau. Schon für
die ersten Nebenmaxima sind Abweichungen zu sehen. Für größere Abstände bis zum Hauptmaximum beschreibt der Fit keinesfalls mehr den zu erwartenden Verlauf, welcher
abklingen sollte. Es konnten keine Startwerte gefunden werden, welche die Funktion präzise an die Messwerte anpasst.
Dadurch sind die hohen Abweichungen für die Spaltbreite und den Spaltabstand zu erklären.

Für den zweiten Doppelspalt konnte ebenfalls keine Funktion präzise an die Messwerte angepasst werden. Die Abweichungen sind in diesem Fall noch größer,
was auf die Verteilung der Messwerte zurückzuführen ist, die Python nicht richtig mit der verwendeten Funktion und den
gegbenen Startwerten annähern kann.
Die Abweichungen der Parameter von den Literaturwerten ist dennoch ähnlich groß wie bei dem ersten Doppelspalt.


Da die Spaltbreite des ersten Doppelspalts und des Einzelspalts voneinander verschieden ist kann keine genaue Aussage aus diesem Vergleich gezogen werden.
Bei gleicher Spaltbreite wäre zu erwarten, dass das Beugungsbild des Einzelspalts als Einhüllende der Beugungsfigur des Doppelspalts dient.
Die geringere Spaltbreite des Einzelspalts sorgt für eine stärkere Beugung des Lichtes, wodurch das Maximum breiter und nicht mehr den
Verlauf der Einhüllenden einnimmt.
