\section{Auswertung}
\label{sec:Auswertung}

Die Messreihen der drei Beugungsbilder werden in Tabelle 1 dargestellt. Dabei bezeichnet der Index 1 den Einzelspalt,
der Index 2 den ersten Doppelspalt und der Index 3 den zweiten Doppelspalt.
%\begin{table}[htp]
        %\begin{center}
                %\begin{tabular}{cc}
                \begin{longtable}{S[table-format=3.2] S[table-format=3.2] S[table-format=3.2] S[table-format=3.2] S[table-format=3.2] S[table-format=3.2]}
                \caption{Gemessene Spannung in Abhängigkeit des Ortes von einem Einzelspalt und zwei Doppelspalte.}\\
                \label{tab:messwerte}
                %\toprule
                {$x_1/$mm} & {$I_1/$nA} & {$x_2/$mm} & {$I_2/$nA} & {$x_3/$mm} & {$I_3/$nA} \\
                \midrule
          13,2   &   0,018  &     19,1    &  0,28   & 21,5   &   0,75                                      \\
          13,7   &   0,018  &     19,6    &  0,16   & 21,6   &   0,74                                      \\
          14,2   &   0,016  &     19,9    &  0,08   & 21,7   &   0,71                                      \\
          14,7   &   0,014  &     20,2    &  0,04   & 21,8   &   0,70                                      \\
          15,7   &   0,008  &     20,5    &  0,05   & 21,9   &   0,71                                      \\
          16,2   &   0,006  &     20,8    &  0,10   & 22,0   &   0,84                                      \\
          16,7   &   0,006  &     21,1    &  0,22   & 22,1   &   1,00                                      \\
          17,2   &   0,008  &     21,3    &  0,59   & 22,2   &   1,20                                      \\
          17,4   &   0,010  &     21,5    &  0,94   & 22,3   &   1,50                                      \\
          17,6   &   0,012  &     21,7    &  1,30   & 22,4   &   1,80                                      \\
          17,8   &   0,015  &     21,9    &  1,60   & 22,5   &   2,15                                      \\
          18,0   &   0,019  &     22,1    &  1,75   & 22,6   &   2,35                                      \\
          18,2   &   0,024  &     22,3    &  1,65   & 22,7   &   2,45                                      \\
          18,4   &   0,028  &     22,5    &  1,35   & 22,8   &   2,40                                      \\
          18,6   &   0,035  &     22,7    &  0,97   & 22,9   &   3,25                                      \\
          18,8   &   0,042  &     22,9    &  0,64   & 23,0   &   2,05                                      \\
          19,0   &   0,050  &     23,1    &  0,72   & 23,1   &   1,90                                      \\
          19,2   &   0,059  &     23,3    &  1,20   & 23,2   &   1,80                                      \\
          19,4   &   0,067  &     23,4    &  1,75   & 23,3   &   1,85                                      \\
          19,6   &   0,078  &     23,5    &  2,30   & 23,4   &   2,00                                      \\
          20,0   &   0,090  &     23,6    &  2,90   & 23,5   &   2,30                                      \\
          20,4   &   0,125  &     23,7    &  3,70   & 23,6   &   2,65                                      \\
          20,8   &   0,150  &     23,8    &  4,50   & 23,7   &   2,80                                      \\
          21,2   &   0,180  &     23,9    &  5,20   & 23,8   &   3,10                                      \\
          21,6   &   0,205  &     24,0    &  5,90   & 23,9   &   3,30                                      \\
          22,0   &   0,285  &     24,1    &  6,20   & 24,0   &   3,20                                      \\
          22,4   &   0,260  &     24,2    &  6,60   & 24,1   &   3,00                                      \\
          22,8   &   0,280  &     24,3    &  6,60   & 24,2   &   2,80                                      \\
          23,2   &   0,280  &     24,4    &  6,60   & 24,3   &   2,40                                      \\
          23,6   &   0,300  &     24,5    &  6,40   & 24,4   &   2,10                                      \\
          24,0   &   0,300  &     24,6    &  5,90   & 24,5   &   1,80                                      \\
          24,4   &   0,310  &     24,7    &  5,30   & 24,6   &   1,85                                      \\
          24,8   &   0,300  &     24,8    &  4,60   & 24,7   &   1,90                                      \\
          25,2   &   0,290  &     24,9    &  3,80   & 24,8   &   2,05                                      \\
          25,6   &   0,280  &     25,0    &  3,00   & 24,9   &   2,25                                      \\
          26,0   &   0,260  &     25,1    &  2,30   & 25,0   &   2,40                                      \\
          26,4   &   0,240  &     25,2    &  1,60   & 25,1   &   2,50                                      \\
          26,8   &   0,220  &     25,4    &  0,90   & 25,2   &   2,50                                      \\
          27,2   &   0,200  &     25,6    &  0,60   & 25,3   &   2,40                                      \\
          27,6   &   0,170  &     25,8    &  0,71   & 25,4   &   2,10                                      \\
          28,0   &   0,140  &     26,0    &  1,05   & 25,5   &   1,80                                      \\
          28,4   &   0,120  &     26,2    &  1,45   & 25,6   &   1,50                                      \\
          28,8   &   0,100  &     26,4    &  1,75   & 25,7   &   1,20                                      \\
          29,2   &   0,070  &     26,6    &  1,75   & 25,8   &   0,95                                      \\
          29,6   &   0,050  &     26,8    &  1,55   & 25,9   &   0,88                                      \\
          30,0   &   0,050  &     27,0    &  1,20   & 26,0   &   0,81                                      \\
          30,4   &   0,035  &     27,2    &  0,85   & 26,1   &   0,80                                      \\
          30,8   &   0,020  &     27,4    &  0,60   & 26,2   &   0,81                                      \\
          31,2   &   0,016  &     27,6    &  0,34   & 26,3   &   0,82                                      \\
          31,6   &   0,010  &     27,8    &  0,18   & 26,4   &   0,82                                      \\
          32,0   &   0,006  &     28,0    &  0,10                                                          \\
          32,5   &   0,004  &     28,2    &  0,09                                                          \\
          33,0   &   0,004  &     28,4    &  0,088                                                          \\
          33,5   &   0,005  &     28,6    &  0,10                                     \\
          34,0   &   0,008  &     28,8    &  0,145                                     \\
          34,5   &   0,010  &     29,0    &  0,20                                     \\
          35,0   &   0,012  &     29,2    &  0,245                                     \\
          35,5   &   0,013  &     29,4    &  0,27                                     \\
          36,0   &   0,013  &                                     \\
          36,5   &   0,012  &                                     \\
                \bottomrule
                %\end{tabular}
        %\end{center}
\end{longtable}

Für den Einzelspalt wird eine Anpassungsfunktion an die Messwerte angepasst. Diese hat die Form wie
in Gleichung (3) dargestellt. Die Spaltbreite beträgt hierbei $b=0,075\,$mm, die Wellenlänge $\lambda = 635\,$nm
und $L$ beträgt $L=1,01\,$m.

%Hierbei ist $b$ die Spaltbreite, $\lambda$ die Wellenlänge, $L$ der Abstand zwischen Spaltöffnung und Detektor,
%$x_0$ die Stelle des Hauptmaximums, $x$ die Auslenkung und $A$ ein Parameter.

\begin{figure}
  \centering
  \includegraphics{einfachspalt.pdf}
  \caption{Beugungsbild des Einzelspalts}
  \label{fig:plot}
\end{figure}

Der Dunkelstrom beträgt dabei $I_{Du}=0,26\,$nm und wird in Abbildung 4 bereits berücksichtigt.


Für den Doppelspalt wird eine Anpassungsfunktion Gleichung (4) verwendet.
In Abbildung 5 und 6 sind die Beugungsbilder der beiden Doppelspalte dargestellt.

Für den Doppelspalt in Abbildung 5 beträgt die Spaltbreite $b=0,15\,$mm und die Spaltentfernung $g=0,25\,$ mm.
Für den Doppelspalt in Abbildung 6 beträgt die Spaltbreite $b=0,15\,$mm und die Spaltentfernung $g=0,5\,$ mm.

\begin{figure}
  \centering
  \includegraphics{doppelspalt_1.pdf}
  \caption{Beugungsbild des ersten Doppelspalts}
  \label{fig:plot}
\end{figure}

\begin{figure}
  \centering
  \includegraphics{doppelspalt_2.pdf}
  \caption{Beugungsbild des zweiten Doppelspalts}
  \label{fig:plot}
\end{figure}

Die Parameter werden mit Python berechnet und in Tabelle 2 dargestellt.
\begin{table}[H]
  \centering
  \caption{Ermittelte Werte der Spalte.}
  \label{tab:Parameter}
  \begin{tabular}{c c c c}
    \toprule
    Parameter & Einzelspalt & Doppelspalt $1$ & Doppelspalt $2$\\
    \midrule
     A & 6,90 \pm 0.05 & $(4,06 \pm 0,04)\cdot 10^{3}$ & $(6,1 \pm 0,4)\cdot 10^{3}$ \\
     $x_0/$mm & $24,31 \pm 0,03$ & $22,937 \pm 0,005$ & $23,993 \pm 0.007$ \\
     $b/$mm & $(0,0805 \pm 0,0007) $ & $0,1077 \pm 0,0006$ & $0,151 \pm 0,004$ \\
     $g/$mm & - & $0,1077 \pm 0,0006$ & $0,495 \pm 0,012$ \\
    \bottomrule
  \end{tabular}
\end{table}

Die Parameter werden mit Python berechnet und in Tabelle 2 dargestellt.
\begin{table}[H]
  \centering
  \caption{Literaturwerte \cite{sample} und Abweichung zu den experimentell bestimmten Werten.}
  \label{tab:Parameter}
  \begin{tabular}{c c c c}
    \toprule
    Parameter & Einzelspalt & Doppelspalt $1$ & Doppelspalt $2$ \\
    \midrule
     $b/$mm & $0,075$ & $0,15$ & $0,15$ \\
     $g/$mm & - & $0,25$ & $0,5$  \\
     \midrule
     Abweichung (b) & $6,8$ \% & $28,2$ \% & $0,006$\% \\
     Abweichung (g) & - & $56,92 $\% & $1,00\,$\% \\
    \bottomrule
  \end{tabular}
\end{table}


Der Plot des Einzelspalts und des ersten Doppespalts werden zusammen in einem Plot dargestellt. Dabei werden die gemessenen Werte des Einzelspalts um einen Faktor
20 vergrößert, damit die Beugungsfiguren in derselben Größenordnung liegen und die beiden Beugungsfiguren miteinander verglichen werden können.

\begin{figure}
  \centering
  \includegraphics{vergleich.pdf}
  \caption{Beugungsbild des Einzelspalts und des Doppelspalts.}
  \label{fig:plot}
\end{figure}

%Die Literaturwerte der Spalte werden aus der Anleitung entnommen \cite{sample}.
%Der Literaturwert der Spaltbreite des Einzelspalts beträgt $b=0,075\,$mm. Die Abweichung des berechneten Wertes davon beträgt $6,8$ \%.
%Die Spaltbreite des ersten Doppelspalts beträgt $b=0,15/,$mm. Der berechnete Wert weicht $28,2$ \% davon ab. Der
%Literaturwert des Spaltabstandes beträgt $g=0.25\,$mm. Die Abweichung zwischen dem errechneten Wert und dem Literaturwert beträgt $56,92 $\%.
%Für den zweiten Doppelspalt betragen die Literaturwerte für Spaltbreite und Spaltabstand $b=0,15\,$mm und $g=0,5\,$mm. Die zugehörigen Abweichungen
%zu den berechneten Werte betragen $0,0062$\% und $1,00\,$\%.
