\section{Diskussion}
\label{sec:Diskussion}

Deutliche Abweichungen der Güteziffern sind durch nicht ideale Bedingungen erklärbar.
Die idealen Güteziffern setzen einen reversiblen Prozess und eine  adiabatische
kompression des Gases voraus. Beides kann in der Realität nicht vollständig
umgesetzt werden.
Die Deckel liegen nicht vollständig auf den Eimern, was zu einer schwächeren
Wärmeisolierung führt. Die gemessenen Werte verändern sich
dementsprechend. Zudem sind die Anzeigen der Manometer nicht allzu präzise. Dies hat
ungenaue Werte für die Drücke zur Folge. Eine große Abweichung des Massendurchsatzes
und der Kompressorleistung im Vergleich zu einer idealen Wärmepumpe ist somit ebenfalls zu
erwarten.
