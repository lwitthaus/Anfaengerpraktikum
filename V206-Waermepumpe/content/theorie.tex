\section{Theorie}
\subsection{Güteziffer}
Unter Aufwand von Arbeit, wird entgegen dem natürlichen Wärmefluss, Wärme transportiert.
Das Verhältnis zwischen transportierter Wärmemenge und Arbeit wird als Güteziffer $\nu$
bezeichnet. Aus dem ersten Hauptsatz der Thermodynamik folgt:
\begin{equation}
  Q_1 = Q_2 + A
\end{equation}
$Q_1$ ist dabei die abgegbene Wärmemenge an das wärmere Reservoir und $Q_2$ die
entzogene Wärmemenge des kälteren Reservoirs. \\
Für die Güteziffer gilt dann
\begin{equation}
  \nu = \frac{Q_1}{A} .
\end{equation}
Ist die Änderung der Temperaturen der beiden Wärmereservoire nahezu Null, so folgt
aus dem zweiten Hauptsatz der Thermodynamik:
\begin{equation}
  \frac{Q_1}{T_1} - \frac{Q_2}{T_2} = 0
\end{equation}
Diese Gleichung gilt jedoch nur für reversible Prozesse, für die Wärmepumpe gilt somit:
\begin{equation}
  \frac{Q_1}{T_1} - \frac{Q_2}{T_2} > 0
\end{equation}
Mit den Gleichungen (1) und (3) folgt für die reale Wärmepumpe:
\begin{equation}
  Q_1 = A + \frac{T_1}{T_2}Q_1
\end{equation}
Zusaätzlich folgt mit Gleichung (2):
\begin{equation}
  \nu_{ideal} = \frac{T_1}{T_1 - T_2}
\end{equation}
Die Gleichung zeigt, dass $\nu$ möglichst groß wird, wenn die Temperaturdifferenz
der Wärmereservoire klein ist. Dies ist ein Vorteil gegenüber Methoden, welche
mechanische Arbeit direkt in Wärmenenergie umwandelt, da dort höchstens
\begin{equation}
  Q_1 \leq A
\end{equation}
gelten kann. \\
Um die reale Güteziffer zu bestimmen wird die Wärmekapazität des Wassers $c_1m_w$ in Reservoir 1 benötigt,
sowie die Wärmekapazität $c_km_k$ des Eimers und der Kupferschlange. Aus einer Messreihe mit der Temperatur
$T_1$ als Funktion der Zeit $t$ folgt für die gewonnene Wärmemenge pro Zeit:
\begin{equation}
  \frac{\Delta Q_1}{\Delta t} = (m_1 c_w+ c_k m_k) \frac{\Delta T_2}{\Delta t}
\end{equation}
Für die Güteziffer gilt:
\begin{equation}
  \nu = \frac{\Delta Q_1}{\Delta t N}
\end{equation}
N ist hierbei die über das Zeitinervall $\Delta t$ gemittelte Leistungsaufnahme des Kompressors.
\subsection{Massendurchsatz}
Der Massendurchsatz $\frac{\Delta m}{\Delta t}$ beschreibt eine Masse die pro Zeit einen bestimmten Querschnitt durchläuft.
Er ist definiert durch
\begin{equation}
  \frac{\Delta Q_2}{\Delta t} = L\frac{\Delta m}{\Delta t} ,
\end{equation}
da bei einer Wärmeentnahme $\frac{\Delta Q_2}{\Delta t}$ des Transportmediums die Verdampfungswärme L pro Masseneinheit und Zeiteinheit
verbraucht wird. Die entnommene Wärmemenge ist dabei
\begin{equation}
  \frac{\Delta Q_2}{\Delta t} = (m_2c_w + m_kc_k)\frac{\Delta T_2}{\Delta t} .
\end{equation}
\subsection{Kompressorleistung}
Der Kompressor verringert das Volumen des Gases von $V_a$ zu $V_b$. Dabei leistet
er die Arbeit $A_m$. Diese ist definiert durch
\begin{equation}
  A_m = - \int_{V_a}^{V_b} f(x) \, \mathrm{d}x
\end{equation}
Wird das Gas nahezu adiabatisch von dem Kompressor komprimiert gilt die Gleichung
\begin{equation}
  p_a V_a^\kappa = p_b V_b^\kappa = pV^\kappa.
\end{equation}
$\kappa$ ist dabei das Verhätlnis der Molwärme von $C_p$ und $C_V$.
Für $A_m$ folgt:
\begin{equation}
  A_m = \frac{1}{\kappa -1}(p_b\sqrt[\kappa]{\frac{p_a}{p_b}}-p_a)\frac{1}{\rho}\frac{\Delta m}{\Delta t}
\end{equation}
$\rho$ ist die Dichte des Transportmediums im Gasförmigen Zustand. Die mechanische Kompressorleistung
ist somit:
\begin{equation}
  N_{mech} = \frac{\Delta A_m}{\Delta t} = \frac{1}{\kappa -1} \left(p_b \sqrt[\kappa]{\frac{p_a}{p_b}}-p_a\right)\frac{1}{\rho}\frac{\Delta m}{\Delta t}
\end{equation}
\label{sec:Theorie}
