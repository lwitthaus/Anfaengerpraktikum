\section{Theorie}
\subsection{Güteziffer}
Unter Aufwand von Arbeit, wird entgegen dem natürlichen Wärmefluss, Wärmetransportiert.
Das Verhältnis zwischen transportierter Wärmemenge und Arbeit wird als Güteziffer $\nu$
bezeichnet. Aus dem ersten Hauptsatz der Thermodynamik folgt:
\begin{equation}
  Q_1 = Q_2 + A
\end{equation}
$Q_1$ ist dabei die abgegbene Wärmemenge an das wärmere Reservoir und $Q_2$ die
entzogene Wärmemenge des kälteren Reservoirs. \\
Für die Güteziffer gilt dann
\begin{equation}
  \nu = Q_1 / A .
\end{equation}
Ist die Änderung der Temperaturen der beiden Wärmereservoire nahezu Null, so folgt
aus dem zweiten Hauptsatz der Thermodynamik:
\begin{equation}
  \frac{Q_1}{T_1} - \frac{Q_2}{T_2} = 0
\end{equation}
Diese Gleichung gilt jedoch nur für reversible Prozesse, für die Wärmepumpe gilt somit:
\begin{equation}
  \frac{Q_1}{T_1} - \frac{Q_2}{T_2} > 0
\end{equation}
Mit den Gleichungen (1) und (3) folgt für die reale Wärmepumpe:
\begin{equation}
  Q_1 = A + \frac{T_1}{T_2}Q_1
\end{equation}
Zusaätzlich folgt mit Gleichung (2):
\begin{equation}
  \nu_{real} = \frac{T_1}{T_1 - T_2}
\end{equation}
Die Gleichung zeigt, dass $\nu$ möglichst groß wird, wenn die Temperaturdifferenz
der Wärmereservoire klein ist. Dies ist ein Vorteil gegenüber Methoden, welche
mechanische Arbeit direkt in Wärmenenergie umwandelt, da dort höchstens
\begin{equation}
  Q_1 \leq A
\end{equation}
gelten kann. \\
Um die reale Güteziffer zu bestimmen wird die Wärmekapazität des Wassers $c_1m_w$ in Reservoir 1 benötigt,
sowie die Wärmekapazität $c_km_k$ des Eimers und der Kupferschlange. Aus einer Messreihe mit der Temperatur
$T_1$ als Funktion der Zeit $t$ lässt sich die Güteziffer mit der Gleichung
\begin{equation}
  \frac{\Delta Q_1}{\Delta t} = (m_1 c_w+ c_k m_k) \frac{\Delta T_2}{\Delta t}
\end{equation}
\subsection{Kompressorleistung}

\label{sec:Theorie}

\cite{sample}
