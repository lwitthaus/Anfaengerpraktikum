\section{Diskussion}
\label{sec:Diskussion}

Die Abweichung der mittleren Reichweiten voneinander kann durch den idealisierten linearen Verlauf der Kurve erklärt werden. Zusätzlich ist
die Zählrate ein statistischer Prozess, was ebenfalls zu Abweichungen führt.

Bei der Bestimmung des Energieverlustes kann die Kurve der Messwerte gut durch eine lineare Regression angepasst werden. Daher fällt der Fehler
sehr gering aus. Da die Energien recht simpel über die gegebenen Informationen aus der Versuchsanleitung \cite{sample} berechnet werden, hängt
die Genauigkeit des errechneten Wertes also auch stark von der Qualität dieser Angabe ab. Aufgrund des eben bereits erwähnten nahezu linearen
Kurvenverlaufs wird jedoch vermutet, dass ein dem tatsächlichen Wert sehr naheliegendes Ergebnis errechnet wird.

Bei der Überprüfung der Statistik des radioaktiven Zerfalls treten gewisse Probleme auf, da die gemessenen Zählraten, wie im obigen Histogramm zu
erkennen, gewisse Einbrüche aufweisen. Dies erschwert die Zuordnung zu einer der beiden Verteilungen (Poisson oder Gauß). Da jedoch bei einer
Poissonverteilung für den Mittelwert und die Varianz ein nahezu gleicher Wert erwartet wird, ist die hier aufgeführte Kurve wohl eher einer
Gaußverteilung zuzuordnen.
