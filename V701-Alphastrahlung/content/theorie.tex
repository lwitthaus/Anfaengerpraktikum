\section{Theorie}
\label{sec:Theorie}

In Materie können $\alpha$-Teilchen auf unterschiedliche Weise ihre Energie abgeben. Zum einen geschieht dies durch elastische Stöße mit
der Materie in der die Teilchen sich befinden. Der Energieverlust der Teilchen durch Anregung und Dissoziation ist jedoch
größer. Für hinreichend große Energien lässt sich der Energieverlust $-\frac{\symup{d}E_{\alpha}}{\symup{d}x}$ der $\alpha$-Teilchen
mit die Bethe-Bloch-Gleichung beschreiben, bei kleinen Energien treten jedoch Ladungsaustauschprozesse auf, wodurch die Gleichung
nicht mehr gültig wird.
\begin{align}
  -\frac{\symup{d}E_{\alpha}}{\symup{d}x} = \frac{z^2 e^4}{4 \pi \epsilon_0 m_e} \frac{n Z}{v^2} ln(\frac{2 m_e v^2}{I})
\end{align}

Hierbei ist $z$ die Ladung $v$ die Geschwindigkeit der $\alpha$-Strahlung, $n$ die Teilchendichte, $Z$ die Ordnungszahl und $I$ die
Ionisierungsenergie des Targetgases.

Für die Reichweite von $\alpha$-Teilchen bis diese jegliche Energie abgegeben haben gilt:
\begin{align}
  R = \int_0^{E_0} \frac{\symup{d}E_{\alpha}}{-\symup{d}E_{\alpha} / \symup{d}x}
\end{align}

Für Energien $E_{\alpha} \leq 2,5$ MeV gilt für die Reichweite der Teilchen:
\begin{align}
  R_m = 3,1 \cdot E_{\alpha}^{3/2}
\end{align}

In Gasen ist die Reichweite von $\alpha$-Teilchen bei konstanter Temperatur und konstantem Volumen proportional zu dem Druck $p$.
Mit einem festen Abstand $x_0$ zwischen Detektor und $\alpha$-Strahler gilt:
\begin{align}
  x = x_0 \frac{p}{p_0}
\end{align}

Dabei ist $x$ die effektive Weglänge und $p_0 =1013$\,mbar der Normaldruck.
