\section{Auswertung}
\label{sec:Auswertung}
\subsection{Bestimmung der effektiven Weglänge und des Energieverlustes}
Die gesamte Anzahl an Ereignissen und der Channel des maximal gemessenen Impulses wird in Tabelle 1 und 2 in Abhängigkeit des Druckes und
des Abstandes dargestellt. Für einen Druck von $0$mbar entspricht die Postion des maximalen Pulses $4$MeV. Daraus lassen sich die Energiewerte $E$ für die
anderen Drücke berechenen, da von einer linearen Energieskala ausgegangen wird. Mit Gleichung (3) wird aus den Energien die mittlere Reichweite
der Teilchen bestimmt und ebenfalls in den Tabellen dargestellt. Die effektive Weglänge $x$ wird mit Gleichung (4) berechnet und in den Tabellen angegeben.


\begin{table}[H]
  \centering
  \caption{Zählrate, Channel, Energie, effektive Weglänge und mittlere Reichweite der $\alpha$-Teilchen für einen Abstand von $2,3$cm}
  \label{tab:Spannungsamplitude}
  \begin{tabular}{c c c c c c}
    \toprule
    $p/$mbar & $N$ & Channel & $E/$MeV & $x/$cm& $R_m/$mm \\
    \midrule
    0	  & 81389  &1046 &  4,00& 0     & 24.80         \\
    50	& 81648  & 999 &  3,82& 0.11  & 23.14        \\
    100 & 79218  & 982 &  3,76& 0.23  & 22.60        \\
    150 & 78200  & 953 &  3,64& 0.34  & 21.53        \\
    200 & 78330  & 917 &  3,51& 0.45  & 20.39        \\
    250 & 77252  & 911 &  3,48& 0.57  & 20.12        \\
    300 & 76070  & 858 &  3,28& 0.68  & 18.42        \\
    350 & 74155  & 833 &  3,19& 0.79  & 17.66        \\
    400 & 73737  & 800 &  3,06& 0.91  & 16.59        \\
    450 & 72061  & 771 &  2,95& 1.02  & 15.71        \\
    500 & 70654  & 739 &  2,83& 1.14  & 14.76        \\
    550 & 68438  & 703 &  2,69& 1.25  & 13.68        \\
    600 & 65735  & 670 &  2,56& 1.36  & 12.70        \\
    650 & 64206  & 642 &  2,46& 1.48  & 11.96        \\
    700 & 57280  & 603 &  2,31& 1.59  & 10.88        \\
    750 & 49102  & 564 &  2,16& 1.70  & 9.84        \\
    800 & 40228  & 526 &  2,01& 1.82  & 8.83        \\
    850 & 26130  & 511 &  1,95& 1.93  & 8.44        \\
    900 & 17869  &  -  & -    & 2.04  & -        \\
    950 & 9083	 &  -  & -    & 2.16  & -        \\
    1000& 3592   &  -  & -    & 2.27  & -        \\
    \bottomrule
  \end{tabular}
\end{table}





\begin{table}[H]
  \centering
  \caption{Zählrate, Channel, Energie, effektive Weglänge und mittlere Reichweite der $\alpha$-Teilchen für einen Abstand von $1,7$cm}
  \label{tab:Spannungsamplitude}
  \begin{tabular}{c c c c c c}
    \toprule
    $p/$mbar & $N$ & Channel & $E/$MeV & $x/$cm & $R_m/$mm \\
    \midrule
    0	  & 133097  & 932 &   4,00 & 0    & 24.80       \\
    50	& 132258  & 910 &   3,91 & 0.08 & 23.97      \\
    100 & 132252  & 894 &   3,84 & 0.17 & 23.33       \\
    150 & 131151  & 875 &   3,76 & 0.25 & 22.60       \\
    200 & 129489  & 860 &   3,69 & 0.34 & 21.97       \\
    250 & 127602  & 823 &   3,53 & 0.42 & 20.56       \\
    300 & 124606  & 782 &   3,37 & 0.50 & 19.18       \\
    350 & 119948  & 756 &   3,24 & 0.59 & 18.08       \\
    400 & 117771  & 729 &   3,13 & 0.67 & 17.17       \\
    450 & 115922  & 712 &   3,06 & 0.76 & 16.59       \\
    500 & 115965  & 689 &   2,96 & 0.84 & 15.79       \\
    550 & 114158  & 674 &   2,89 & 0.92 & 15.23       \\
    600 & 112766  & 652 &   2,80 & 1.01 & 14.52       \\
    650 & 110443  & 630 &   2,70 & 1.09 & 13.75       \\
    700 & 109593  & 625 &   2,68 & 1.17 & 13.60       \\
    750 & 106939  & 592 &   2,54 & 1.26 & 12.55      \\
    800 & 102803  & 569 &   2,44 & 1.34 & 11.82      \\
    850 & 98282	  & 543 &   2,33 & 1.43 & 11.03      \\
    900 & 94237	  & 525 &   2,25 & 1.51 & 10.46   \\
    950 & 89703	  & 502 &   2,15 & 1.59 & 9.77     \\
    1000& 85338   & 491 &   2,11 & 1.68 & 9.50    \\
    \bottomrule
  \end{tabular}
\end{table}





%\begin{table}
%\centering
%\caption{Frequenzen und Geschwindigkeiten in Abhängigkeit von der Dicke des Rohres, des Winkels und der Strömungsgeschwindigkeit.}
%\sisetup{table-format=2.1}
%\begin{tabular}{S [table-format=3.1] S S S | S S S S S S}
%\toprule
%& \multicolumn{4}{c}{$2,3$cm Abstand} & \multicolumn{4}{c}{$1,7$cm Abstand}  \\
%\cmidrule(lr){1-4}\cmidrule(lr){5-8}
%{$N$}& {Channel} & {$E/$ MeV} & {$R_m$/mm} & {$N$}& {Channel} & {$E/$ MeV} & {$R_m$/mm} \\
%\midrule{lr}{1-9}
%81389  &1046 &  4,00 & 24.8   & 1 & 2 & 3 & 4      \\
%81648  & 999 &  3,82 & 23.14  & 1 & 2 & 3 & 4      \\
%79218  & 982 &  3,76 & 22.60  & 1 & 2 & 3 & 4      \\
%78200  & 953 &  3,64 & 21.53  & 1 & 2 & 3 & 4      \\
%78330  & 917 &  3,51 & 20.39  & 1 & 2 & 3 & 4      \\
%77252  & 911 &  3,48 & 20.12  & 1 & 2 & 3 & 4      \\
%76070  & 858 &  3,28 & 18.42  & 1 & 2 & 3 & 4      \\
%74155  & 833 &  3,19 & 17.66  & 1 & 2 & 3 & 4      \\
%73737  & 800 &  3,06 & 16.59  & 1 & 2 & 3 & 4      \\
%72061  & 771 &  2,95 & 15.71  & 1 & 2 & 3 & 4      \\
%70654  & 739 &  2,83 & 14.76  & 1 & 2 & 3 & 4      \\
%68438  & 703 &  2,69 & 13.68  & 1 & 2 & 3 & 4      \\
%65735  & 670 &  2,56 & 12.70  & 1 & 2 & 3 & 4      \\
%64206  & 642 &  2,46 & 11.96  & 1 & 2 & 3 & 4      \\
%57280  & 603 &  2,31 & 10.88  & 1 & 2 & 3 & 4      \\
%49102  & 564 &  2,16 & 9.84   & 1 & 2 & 3 & 4     \\
%40228  & 526 &  2,01 & 8.83   & 1 & 2 & 3 & 4     \\
%26130  & 511 &  1,95 & 8.44   & 1 & 2 & 3 & 4     \\
%17869  &   0 & 1    & 2       & 1 & 2 & 3 & 4 \\
%9083	 &   0 &  1   &  2      & 1 & 2 & 3 & 4 \\
%3592   &   0 & 1    & 2       & 1 & 2 & 3 & 4 \\
%\bottomrule
%\end{tabular}
%\end{table}

Für die letzten drei Messwerte in Tabelle 1 ist die Kollimatorschwelle zu hoch, wodurch kein Maximum mehr zu messen ist.

Die Zählrate $N$ wird gegen die mittlere Reichweite für beide Abstände aufgetragen.

\begin{figure}
  \centering
  \includegraphics{plot2.pdf}
  \caption{Plot.}
  \label{fig:plot}
\end{figure}


\begin{figure}
  \centering
  \includegraphics{plot3.pdf}
  \caption{Plot.}
  \label{fig:plot}
\end{figure}





\subsection{Statistik des radioaktiven Zerfalls}
