\section{Auswertung}
\label{sec:Auswertung}
\subsection{Betimmung der Winkelrichtgröße}
Die Winkelrichtgröße $D$ wird mit Gleichung (4) ermittelt.
Die gemessenen Kräfte der Federwaage für 10 unterschiedliche Auslenkungen, sowie
die zugehörigen Winkelrichtgrößen, werden im folgenden dargestellt. Dabei beträgt
der Abstand des Ortes der gemessenen Kräfte zum Mittelpunkt der Drillachse
$(17.1 \pm 0.1)$cm.

\begin{table}[H]
  \centering
  \caption{Gemessene Kräfte und Auslenkungen}
  \label{tab:Parameter}
  \begin{tabular}{c c c}
    \toprule
    $\phi$ & $F/$N & $D/$Nm \\
    \bottomrule
     30° & 0.10  & $0.03266 \pm 0.00019$ \\
     60° & 0.22  & $0.03592 \pm 0.00021$ \\
     90° & 0.24  & $0.02613 \pm 0.00015$ \\
    120° & 0.32  & $0.02613 \pm 0.00015$ \\
    150° & 0.36  & $0.02351 \pm 0.00014$ \\
    180° & 0.49  & $0.02667 \pm 0.00016$ \\
    210° & 0.54  & $0.02519 \pm 0.00015$ \\
    240° & 0.62  & $0.02531 \pm 0.00015$ \\
    270° & 0.69  & $0.02504 \pm 0.00015$ \\
    300° & 0.75  & $0.02449 \pm 0.00014$ \\
    \bottomrule
  \end{tabular}
\end{table}

Die Unsicherheit von 1mm wird bei der Messung des Abstandes
abgeschätzt.

\subsection{Bestimmung des Trägheitsmoments der Drillachse}
Die Schwingungsdauer $T$ der als masselos angenommene Stange, sowie der Abstand $a$
der befestigten Gewichte werden im folgenden dargestellt. Der Abstand der einzelnen
Gewichte zum Mittelpunkt ist dabei identisch.
\begin{table}
  \centering
  \caption{Gemessene Schwingungsdauer und Abstände}
  \label{tab:Gemessene Schwingungsdauer und Abstände}
  \begin{tabular}{c c}
    \toprule
    $a/$cm & $T/$s \\
    \midrule
     7.0 & 2.73 \\
     8.0 & 3.06 \\
     9.0 & 3.15 \\
    10.0 & 3.52 \\
    11.0 & 3.67 \\
    12.0 & 3.87 \\
    13.0 & 4.18 \\
    14.0 & 4.29 \\
    15.0 & 4.55 \\
    16.0 & 4.69 \\
    \bottomrule
  \end{tabular}
\end{table} \\

Die Unsicherheit der gemessenen Abstände wird auf $0.1$cm geschätzt.
Das Quadrat der Schwingungsdauer wird auf das Quadrat des Abstandes
aufgetragen. Eine lineare Regression wird mit den Messpunkten durch geführt.

\begin{figure}
  \centering
  \includegraphics{plot.pdf}
  \caption{Plot.}
  \label{fig:plot}
\end{figure}

Die Gerade wird beschrieben durch die Gleichung:
\begin{equation}
  y = (703 \pm 26)x + (4.72 \pm 0.42)
\end{equation}
Es gilt nach Gleichung (3):
\begin{align}
  T^2(a^2) = \frac{4\pi^2}{D}(I_D + 2mr^2)
\end{align}
\begin{align}
  \Rightarrow T^2(a^2) = \underbrace{\frac{8\pi^2}{D}}_{Steigung \: m} \cdot a^2 + \underbrace{\frac{4\pi^2}{D}I_D}_{Achsenabschnitt \: b}
\end{align}
\begin{align}
  \Rightarrow I_D = \frac{2mb}{a}
\end{align}
Für das Trägheitsmoment der Drillachse ergibt sich:
\begin{equation}
  I_D = (0.0030 \pm 0.0003) kgm^2
\end{equation}

\subsection{Bestimmung des Trägheitsmomente zweier Körper}
Die Schwingungsdauer eines Zylinders und einer Kugel werden im folgenden tabelliert.
\begin{table}[H]
  \centering
  \caption{Schwingungsdauer von Kugel und Zylinder}
  \label{tab:Schwingungsdauer von Kugel und Zylinder}
  \begin{tabular}{c c}
    \toprule
    $T_Z/$s & $T_K/$s \\
    \midrule
    2.29 & 1.56 \\
    2.23 & 1.53 \\
    2.23 & 1.64 \\
    2.07 & 1.58 \\
    2.07 & 1.60 \\
    \bottomrule
  \end{tabular}
\end{table}

Mit diesen Werten ergeben sich nach Geleichung (3) folgende Trägheitsmomente
für Zylinder und Kugel:
\begin{table}[H]
  \centering
  \caption{Trägheitsmoment von Kugel und Zylinder}
  \label{tab:Trägheitsmoment von Kugel und Zylinder}
  \begin{tabular}{c c}
    \toprule
    $I_Z/\symup{kgm^2}$ & $I_K/\symup{kgm^2}$ \\
    \midrule
    $0.003601 \pm 0.000007$ & $0.0016712 \pm 0.0000031$ \\
    $0.003415 \pm 0.000006$ & $0.0016075 \pm 0.0000030$ \\
    $0.003415 \pm 0.000006$ & $0.0018470 \pm 0.0000034$ \\
    $0.002942 \pm 0.000005$ & $0.0017143 \pm 0.0000032$ \\
    $0.002942 \pm 0.000005$ & $0.0017580 \pm 0.0000032$ \\
    \bottomrule
  \end{tabular}
\end{table}

Als Mittelwerte ergeben sich:
\begin{equation}
  I_Z = (0.0032630 \pm 0.0000026) \symup{kgm^2}
\end{equation}
\begin{equation}
  I_K = (0.0017196 \pm 0.0000014) \symup{kgm^2}
\end{equation}

Dabei wurde das Trägheitsmoment der Drillachse nicht berücksichtigt, da sich
ansonsten negative Trägheitsmomente für Zylinder und Kugel ergeben würden.


\subsection{Bestimmung des Trägheitsmoments einer Modellpuppe}
Die Schwingundsdauer einer Modellpuppe wird gemessen. Die Puppe nimmt zwei unterschiedliche
Posen $P_1$ und $P_2$ ein. Bei der ersten Pose sind Arme und Beine am Körper angelegt. In der zweiten
streckt sie die Arme senkrecht nach außen. Die Schwingungsdauern der Puppe betragen:
\begin{table}
  \centering
  \caption{Schwingungsdauer von Modellpuppe}
  \label{tab:Schwingungsdauer von Modellpuppe}
  \begin{tabular}{c c}
    \toprule
    $T_{P_1}/$s & $T_{P_2}/$s \\
    \midrule
    0.41 & 0.60 \\
    0.41 & 0.55 \\
    0.36 & 0.50 \\
    0.50 & 0.64 \\
    0.52 & 0.64 \\
    \bottomrule
  \end{tabular}
\end{table}
