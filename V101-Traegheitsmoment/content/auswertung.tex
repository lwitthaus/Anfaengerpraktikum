\section{Auswertung}
\label{sec:Auswertung}
\subsection{Betimmung der Winkelrichtgröße}
Die Winkelrichtgröße $D$ wird mit Gleichung (4) ermittelt.
Die gemessenen Kräfte der Federwaage für 10 unterschiedliche Auslenkungen, sowie
die zugehörigen Winkelrichtgrößen, werden im folgenden dargestellt. Dabei beträgt
der Abstand des Ortes der gemessenen Kräfte zum Mittelpunkt der Drillachse
$(17.1 \pm 0.1)$cm.

\begin{table}[H]
  \centering
  \caption{Gemessene Kräfte und Auslenkungen}
  \label{tab:Parameter}
  \begin{tabular}{c c c}
    \toprule
    $\phi$ & $F/$N & $D/$Nm \\
    \bottomrule
     30° & 0.10  & $0.03266 \pm 0.00019$ \\
     60° & 0.22  & $0.03592 \pm 0.00021$ \\
     90° & 0.24  & $0.02613 \pm 0.00015$ \\
    120° & 0.32  & $0.02613 \pm 0.00015$ \\
    150° & 0.36  & $0.02351 \pm 0.00014$ \\
    180° & 0.49  & $0.02667 \pm 0.00016$ \\
    210° & 0.54  & $0.02519 \pm 0.00015$ \\
    240° & 0.62  & $0.02531 \pm 0.00015$ \\
    270° & 0.69  & $0.02504 \pm 0.00015$ \\
    300° & 0.75  & $0.02449 \pm 0.00014$ \\
    \bottomrule
  \end{tabular}
\end{table}

Die Unsicherheit von 1mm wird bei der Messung des Abstandes
abgeschätzt.

Es ergibt sich der folgende Mittelwert:
\begin{equation}
  D = (0.02711 \pm 0.00005) \symup{Nm}
\end{equation}

\subsection{Bestimmung des Trägheitsmoments der Drillachse}
Die Schwingungsdauer $T$ der als masselos angenommenen Stange, sowie der Abstand $a$
der befestigten Gewichte werden im folgenden dargestellt. Der Abstand der einzelnen
Gewichte zum Mittelpunkt ist dabei identisch.
\begin{table}
  \centering
  \caption{Gemessene Schwingungsdauern und Abstände}
  \label{tab:Gemessene Schwingungsdauern und Abstände}
  \begin{tabular}{c c}
    \toprule
    $a/$cm & $T/$s \\
    \midrule
     7.0 & 2.73 \\
     8.0 & 3.06 \\
     9.0 & 3.15 \\
    10.0 & 3.52 \\
    11.0 & 3.67 \\
    12.0 & 3.87 \\
    13.0 & 4.18 \\
    14.0 & 4.29 \\
    15.0 & 4.55 \\
    16.0 & 4.69 \\
    \bottomrule
  \end{tabular}
\end{table} \\

Die Unsicherheit der gemessenen Abstände wird auf $0.1$cm geschätzt.
Das Quadrat der Schwingungsdauer wird auf das Quadrat des Abstandes
aufgetragen. Eine lineare Regression wird mit den Messpunkten durchgeführt.

\begin{figure}[H]
  \centering
  \includegraphics{plot.pdf}
  \caption{Plot.}
  \label{fig:plot}
\end{figure}

Die Gerade wird beschrieben durch die Gleichung:
\begin{equation}
  y = (703 \pm 26)x + (4.72 \pm 0.42)
\end{equation}
Es gilt nach Gleichung (3):
\begin{align}
  T^2(a^2) = \frac{4\pi^2}{D}(I_D + 2ma^2)
\end{align}
\begin{align}
  \Rightarrow T^2(a^2) = \underbrace{\frac{8\pi^2}{D}}_{Steigung \: m} \cdot a^2 + \underbrace{\frac{4\pi^2}{D}I_D}_{Achsenabschnitt \: b}
\end{align}
\begin{align}
  \Rightarrow I_D = \frac{2mb}{a}
\end{align}
Für das Trägheitsmoment der Drillachse ergibt sich:
\begin{equation}
  I_D = (0.0030 \pm 0.0003) \symup{kgm^2}
\end{equation}

\subsection{Bestimmung des Trägheitsmomente zweier Körper}
Die Schwingungsdauer eines Zylinders und einer Kugel werden im folgenden tabelliert.
\begin{table}[H]
  \centering
  \caption{Schwingungsdauer von Zylinder und Kugel}
  \label{tab:Schwingungsdauer von Zylinder und Kugel}
  \begin{tabular}{c c}
    \toprule
    $T_Z/$s & $T_K/$s \\
    \midrule
    2.29 & 1.56 \\
    2.23 & 1.53 \\
    2.23 & 1.64 \\
    2.07 & 1.58 \\
    2.07 & 1.60 \\
    \bottomrule
  \end{tabular}
\end{table}

Mit diesen Werten ergeben sich nach Gleichung (3) folgende Trägheitsmomente
für Zylinder und Kugel:
\begin{table}[H]
  \centering
  \caption{Trägheitsmoment von Kugel und Zylinder}
  \label{tab:Trägheitsmoment von Kugel und Zylinder}
  \begin{tabular}{c c}
    \toprule
    $I_Z/\symup{kgm^2}$ & $I_K/\symup{kgm^2}$ \\
    \midrule
    $0.003601 \pm 0.000007$ & $0.0016712 \pm 0.0000031$ \\
    $0.003415 \pm 0.000006$ & $0.0016075 \pm 0.0000030$ \\
    $0.003415 \pm 0.000006$ & $0.0018470 \pm 0.0000034$ \\
    $0.002942 \pm 0.000005$ & $0.0017143 \pm 0.0000032$ \\
    $0.002942 \pm 0.000005$ & $0.0017580 \pm 0.0000032$ \\
    \bottomrule
  \end{tabular}
\end{table}

Als Mittelwerte ergeben sich:
\begin{equation}
  I_Z = (0.0032630 \pm 0.0000026) \symup{kgm^2}
\end{equation}
\begin{equation}
  I_K = (0.0017196 \pm 0.0000014) \symup{kgm^2}
\end{equation}

Dabei wurde das Trägheitsmoment der Drillachse nicht berücksichtigt, da sich
ansonsten negative Trägheitsmomente für Zylinder und Kugel ergeben würden.
Der Theoriewert des Trägheitsmoments des Zylinders mit einer Masse von $m=0.2369$ kg und einem Radius von
$r=0.051$ m beträgt $I_Z = 0.0031160 \symup{kgm^2}$. Die Abweichung beträgt 4.5 \%.
Der Theoriewert der Kugel bei einer Masse $m = 0.8125$ kg und einem Radius von $r = 0.066$ m
beträgt $I_K = 0.0014157 \symup{kgm^2}$. Die Abweichung beträgt 21.5 \%.


\subsection{Bestimmung des Trägheitsmoments einer Modellpuppe}
Die Schwingundsdauer einer Modellpuppe wird gemessen. Die Puppe nimmt zwei unterschiedliche
Posen $P_1$ und $P_2$ ein. Bei der ersten Pose sind Arme und Beine am Körper angelegt. In der zweiten
streckt sie die Arme senkrecht nach außen. Die Schwingungsdauern der Puppe betragen:
\begin{table}
  \centering
  \caption{Schwingungsdauer der Modellpuppe}
  \label{tab:Schwingungsdauer der Modellpuppe}
  \begin{tabular}{c c}
    \toprule
    $T_{P_1}/$s & $T_{P_2}/$s \\
    \midrule
    0.41 & 0.60 \\
    0.41 & 0.55 \\
    0.36 & 0.50 \\
    0.50 & 0.64 \\
    0.52 & 0.64 \\
    \bottomrule
  \end{tabular}
\end{table}

Mit diesen Werten ergeben sich nach Gleichung (3) folgende Trägheitsmomente
für die Puppe:
\begin{table}[H]
  \centering
  \caption{Trägheitsmomente der Puppe}
  \label{tab:Trägheitsmomente der Puppe}
  \begin{tabular}{c c}
    \toprule
    $I_{P_1}/\symup{kgm^2}$ & $I_{P_2}/\symup{kgm^2}$ \\
    \midrule
    $(1.1543 \pm 0.0021)\cdot 10^{-4}$ & $(2.472 \pm 0.005)\cdot 10^{-4}$ \\
    $(1.1543 \pm 0.0021)\cdot 10^{-4}$ & $(2.077 \pm 0.004)\cdot 10^{-4}$ \\
    $(0.8900 \pm 0.0016)\cdot 10^{-4}$ & $(1.717 \pm 0.003)\cdot 10^{-4}$ \\
    $(1.7168 \pm 0.0032)\cdot 10^{-4}$ & $(2.813 \pm 0.005)\cdot 10^{-4}$ \\
    $(1.8568 \pm 0.0034)\cdot 10^{-4}$ & $(2.813 \pm 0.005)\cdot 10^{-4}$ \\
    \bottomrule
  \end{tabular}
\end{table}

Als Mittelwerte ergeben sich:
\begin{equation}
  I_{P_1} = (1.3544 \pm 0.0012)\cdot 10^{-4}\: \symup{kgm^2}
\end{equation}
\begin{equation}
  I_{P_2} = (2.3784 \pm 0.0020)\cdot 10^{-4}\: \symup{kgm^2}
\end{equation}
Das Trägheitsmoment der Puppe in Pose 1 lässt sich mit Gleichung (9)
berechnen. Die Arme und Beine, sowie der Rumpf der Puppe werden als Zylinder angenähert.
Der Kopf wird als Kugel angenähert.
Die dafür benötigten Größen sind in der folgenden Tabelle angegeben.
\begin{table}[H]
  \centering
  \caption{Eigenschaften der Puppe}
  \label{tab:Eigenschaften der Puppe}
  \begin{tabular}{c c c c c c c c}
    \toprule
    $I_K/ \symup{kgm^2}$ & $m_A /$kg & $I_A/\symup{kgm^2}$ & $R_A/$m & $R_R/$m & $I_R/\symup{kgm^2}$ & $m_B/$m & $I_B/\symup{kgm^2}$  \\
    \midrule
    $4.3\cdot 10^{-6}$ & 0.02 & $6.89\cdot 10^{-7}$ & 0.0083 & 0.0188 & $1.27\cdot 10^{-5}$ & 0.024 & $8.67\cdot 10^{-7}$ \\
    \bottomrule
  \end{tabular}
\end{table}
Mit diesen Werten folgt für das Trägheitsmoment der Puppe in Pose 1:
\begin{equation}
I_{P_1} = 1.894 \cdot 10^{-4} \symup{kgm^2}
\end{equation}
Die Abweichung des berechneten Wertes von dem Theoriewert beträgt $28.5\%$

Für das Trägheitsmoment der Puppe in Pose 2 gilt:
\begin{equation}
  I_{P_2} = I_K + I_R + 2(I_A + m_A(R_R + \frac{h_A}{2})^2) + 2(I_B + m_B R_B^2)
\end{equation}
$I_A$ ist nun das Trägheitsmoment eines Zylinder dessen Drehachse durch den Zylindermantel
verläuft und $h_A= 0.0138$m die Höhe des Zylinders.
Das Trägheitsmoment $I_{P_2}$ beträgt:
\begin{equation}
  I_{P_2} = 1.2787 \cdot 10^{-4} \symup{kgm^2}
\end{equation}
Die Abweichung des berechneten Wertes von dem Theoriewert beträgt $46.2 \%$.
