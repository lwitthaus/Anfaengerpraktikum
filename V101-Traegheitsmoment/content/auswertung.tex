\section{Auswertung}
\label{sec:Auswertung}
\subsection{Betimmung der Winkelrichtgröße}
Die Winkelrichtgröße $D$ wird mit Gleichung (4) ermittelt.
Die gemessenen Kräfte der Federwaage für 10 unterschiedliche Auslenkungen, sowie
die zugehörigen Winkelrichtgrößen, werden im folgenden dargestellt. Dabei beträgt
der Abstand des Ortes der gemessenen Kräfte zum Mittelpunkt der Drillachse
$(17.1 \pm 0.1)$cm.

\begin{table}[H]
  \centering
  \caption{Gemessene Kräfte und Auslenkungen}
  \label{tab:Parameter}
  \begin{tabular}{c c c}
    \toprule
    $\phi$ & $F/$N & $D/$Nm \\
    \bottomrule
     30° & 0.10  & $0.03266 \pm 0.00019$ \\
     60° & 0.22  & $0.03592 \pm 0.00021$ \\
     90° & 0.24  & $0.02613 \pm 0.00015$ \\
    120° & 0.32  & $0.02613 \pm 0.00015$ \\
    150° & 0.36  & $0.02351 \pm 0.00014$ \\
    180° & 0.49  & $0.02667 \pm 0.00016$ \\
    210° & 0.54  & $0.02519 \pm 0.00015$ \\
    240° & 0.62  & $0.02531 \pm 0.00015$ \\
    270° & 0.69  & $0.02504 \pm 0.00015$ \\
    300° & 0.75  & $0.02449 \pm 0.00014$ \\
    \bottomrule
  \end{tabular}
\end{table} 

Die Unsicherheit von 1mm wird bei der Messung des Abstandes
abgeschätzt.

\subsection{Bestimmung des Trägheitsmoments der Drillachse}
Die Schwingungsdauer $T$ der als masselos angenommene Stange, sowie der Abstand $a$
der befestigten Gewichte werden im folgenden dargestellt. Der Abstand der einzelnen
Gewichte zum Mittelpunkt ist dabei identisch.
\begin{table}
  \centering
  \caption{Gemessene Schwingungsdauer und Abstände}
  \label{tab:Gemessene Schwingungsdauer und Abstände}
  \begin{tabular}{c c}
    \toprule
    $a/$cm & $T/$s \\
    \midrule
     7.0 & 2.73 \\
     8.0 & 3.06 \\
     9.0 & 3.15 \\
    10.0 & 3.52 \\
    11.0 & 3.67 \\
    12.0 & 3.87 \\
    13.0 & 4.18 \\
    14.0 & 4.29 \\
    15.0 & 4.55 \\
    16.0 & 4.69 \\
    \bottomrule
  \end{tabular}
\end{table} \\

Die Unsicherheit der gemessenen Abstände wird auf $0.1$cm geschätzt.
Das Quadrat der Schwingungsdauer wird auf das Quadrat des Abstandes
aufgetragen. Eine lineare Regression wird mit den Messpunkten durch geführt.

\begin{figure}
  \centering
  \includegraphics{plot.pdf}
  \caption{Plot.}
  \label{fig:plot}
\end{figure}


\subsection{Bestimmung des Trägheitsmomente zweier Körper}
Die Schwingungsdauer eines Zylinders und einer Kugel werden im folgenden tabelliert.
\begin{table}
  \centering
  \caption{Schwingungsdauer von Kugel und Zylinder}
  \label{tab:Schwingungsdauer von Kugel und Zylinder}
  \begin{tabular}{c c}
    \toprule
    $T_Z/$s & $T_K/$s \\
    \midrule
    2.29 & 1.56 \\
    2.23 & 1.53 \\
    2.23 & 1.64 \\
    2.07 & 1.58 \\
    2.07 & 1.60 \\
    \bottomrule
  \end{tabular}
\end{table} \\
