\section{Diskussion}
\label{sec:Diskussion}
Die Abmessungen von Abständen konnte nur mit ungenauen Methoden durchgeführt werden,
da es aus dem Versuchsaufbau nicht möglich war die Schieblehre effektiv einzusetzen.
Das Ablesen der Auslenkungen aus der Ruhelage wird mit dem Auge vorgenommen, was
den Winkel $\phi$ ungenau werden lässt. Das Messen der Periodendauer der
Schwingungen mit einer Stoppuhr führt ebenfalls zu systematischen Fehler.
Außerdem wird angenommen, dass die Stange masselos ist, was in der Realität
nicht der Fall ist. Für die Berechnung der Drillachse werden die Gewichte idealerweise
als Punktmassen angenommen, was für eine starke Abweichung sorgt, welche später für
negative Trägheitsmomente verantwortlich ist.
Gleichung (4) gilt nur für kleine Auslenkungen $\phi$ , wobei
in diesem Versuch bis zu 300° ausgelenkt wird. Das führt zu größeren Abweichungen
der Trägheitsmomente von den Theoriewerten. Es ist zudem Fragwürdig inwiefern die
berechneten Trägheitsmomente mit den realen übereinstimmen, da das Trägheitsmoment
der Drillachse vernachlässigt wurde.
Der Theoriewert des Trägheitsmoments
der Puppe weicht stark von dem Trägheitsmoment der tatsächlichen Puppe ab, da für die
Berechnungen starke Vereinfachungen vorgenommen werden. Die Messung der Periodendauer
der Schwingung der Puppe ist wegen einer sehr kleinen Periodendauer schwierig und
ungenau. Zudem ändert die Puppe während der Schwingung ein wenig ihre Haltung was zu
weiteren Abweichungen von den tatsächlichen Trägheitsmomenten führt. Somit ist der
Vergleich der berechneten Trägheitsmomente zu den Theoriewerten nicht wirklich
möglich, da es zu viele systematische Fehler gibt.
