\section{Diskussion}
\label{sec:Diskussion}
Die Abmessungen von Abständen konnte nur mit ungenauen Methoden durchgeführt werden,
da es aus dem Versuchsaufbau nicht möglich war die Schieblehre effektiv einzusetzen.
Das Ablesen der Auslenkungen aus der Ruhelage wird mit dem Auge vorgenommen, was
den Winkel $\phi$ ungenau werden lässt. Das Messen der Periodendauer der
Schwingungen mit einer Stoppuhr führ ebenfalls zu systematischen Fehler.
Außerdem wird angenommen, dass die Stange masselos ist, was in der Realität
nicht der Fall ist. Gleichung (4) gilt nur für kleine Auslenkungen $\phi$ , wobei
in diesem Versuch bis zu 300° ausgelenkt wird. Das führt zu größeren Abweichungen
der Trägheitsmomente von den Theorie werten. Es ist zudem Fragwürdig inwiefern die
berechneten Trägheitsmomente mit den realen über einstimmen, da das Trägheitsmoment
der Drillachse vernachlässigt wurde.
Der Theoriewert des Trägheitsmoments
der Puppe weicht stark von dem Trägheitsmoment der tatsächlichen Puppe ab, da für die
Berechnungen starke vereinfachungen getroffen werden.
