\section{Theorie}
\label{sec:Theorie}

\cite{sample}
Dreht sich eine punktförmige Masse $m$ mit dem Abstand $r$ zu seiner Drehachse, so
ist das Trägheitsmoment $I$ dieser Masse definiert als:
\begin{equation}
  I = m r^2
\end{equation}
Bei mehreren Massepunkten in einem Körper addieren sich die einzelnen
Trägheitsmomente.
Wenn die Drehachse nicht durch den Schwerpunkt des Körpers verläuft, so kann
die Achse parallel in den Schwerpunkt verschoben werden. Dies besagt der Steiner'sche
Satz. Es gilt:
\begin{equation}
  I = I_s + m a^2
\end{equation}
$I_s$ ist die neue Drehachse durch den Schwerpunkt, $m$ die Gesmatmasse
des Körpers und $a$ der Abstand zwischen der alten und der neuen Drehachse.
In diesem Versuch wird ein schwingungsfähiges System betrachte. Wird also ein Körper
um den Winkel $\phi$ ausgelenkt, so wirkt der Auslenkung ein Drehmoment $\symbf{M}$ entgegen.
Der Körper wird eine harmonische Schwingung ausführen, wenn er ausgelenkt und
losgelassen wird. Die Periodendauer $T$ der Schwingung ist definiert durch:
\begin{equation}
  T = 2 \pi \sqrt{\frac{I}{D}}
\end{equation}
D ist dabei die Winkelrichtgröße. Diese Gleichung gilt nur für
kleine Auslenkungen von $\phi$ und für $\symup{D}$ gilt außerdem:
\begin{equation}
  \symbf{M} = \symbf{D} \phi
\end{equation}
Das Drehmoment $\symbf{M}$ ist definiert als:
\begin{equation}
  \symbf{M} = \symbf{F}{\symbf{r}} = F r sin(\phi)
\end{equation}
Die Körper werden in dem Versuch an eine Drillachse befestigt. Um das Trägheitsmoment
der Körper zu berechnen wird das Trägheitsmoment der Drillachse benötigt, da sich
das Trägheitsmoment $I$ aus Gleichung (3) wie folgt zusammensetzt:
\begin{equation}
  I = I_K + I_D
\end{equation}
$I_K$ ist dabei das Trägheitsmoment des Körpers und $I_D$ das
Trägheitsmoment der Drill\-achse.

Die Trägheitsmomente von einer Kugel und einem Zylinder werden für den Versuch benötigt
und betragen:
\begin{align}
  I_K &= \frac{2}{5}mr^2 \\
  I_Z &= \frac{1}{2}mr^2
\end{align}

\subsection{Statische Methode}
Hierbei wird die winkelrichtgröße aus Gleichung (4) ermittelt. Dafür muss die
Kraft $\symbf{F}$ bekannt sein, welche in einem Abstand $\symbf{r}$ auf den Körper wirkt,
sowie der Auslenkungswinkel $\phi$.
\subsection{Dynamische Methode}
Aus Gleichung (3) kann bei bekanntem $I$ die Winkelrichtgröße berechnet werden. Dafür
misst man die Periodendauer der Schwingung des Systems.
