\section{Diskussion}
\label{sec:Diskussion}


Die einzelnen Anpassungsfunktionen beschreiben den Verlauf der Messwerte grundsätzlich ganz gut. Nur bei dem
roten Licht gibt es größere Ungenauigkeiten, welche auf den letzten Messwert zurückzuführen sind. Für eine
bessere Anpassungsfunktion wären mehr Messwerte im relevanten Bereich von Vorteil gewesen. Während des Versuches ist
die rote Spektrallinie nur sehr schwach bis gar nicht erkennbar, wodurch weitere Fehler entstehen können.
Auch die gelbe Spektrallinie konnte nicht gemessen werden, sondern lediglich eine orangene.

Das berechnete Verhältnis von $\frac{h}{e_0}$ liegt näherungsweise an dem Literaturwert von $\frac{h}{e_0} = 2.58 *10^{-4}$eV, jedoch
imernoch außerhalb der Standardabweichung. Die Anpassungsfunktion des zugehörigen graphen weist eine deutliche Abweichung auf,
welche auf den Messwert des roten Lichtes zurückgeführt werden kann. Die Abweichung des Messwertes kann durch die oben
genannten gründe erklärt werden. 
