\section{Auswertung}
\label{sec:Auswertung}



Die gemessene Stromstärke in Abhängigkeit von der Bremsspannung wird für das grüne
Licht in Tabelle 1 dargestellt. Negative Spannungen bezeichenen dabei das Gegenfeld, also
die Spannung mit der Elektronen gebremst werden. Positive Spannungnen beschleunigen die Elektronen.


\begin{table}[H]
  \centering
  \caption{Gemessene Stromstärke bei verschiedenen Spannungen für grünes Licht}
  \label{tab:Spannungsamplitude}
  \begin{tabular}{c c}
    \toprule
    $U/$V & $I/$nA \\
    \midrule
    -0,5 & 0 \\
    -0,3 & 0,02 \\
    -0,1 & 0,08 \\
     0	 & 0,13 \\
     0,1 & 0,21 \\
     0,3 & 0,30 \\
     0,5 & 0,49 \\
     0,7 & 0,66 \\
     1,0 & 0,80 \\
     1,5 & 1,49 \\
     2,0 & 2,50 \\
     3,0 & 5,10 \\
     5,0 & 9,60 \\
     8,0 & 11,0 \\
    12,0 & 11,0 \\
    \bottomrule
  \end{tabular}
\end{table}

Für die lineare Regression werden die letzten drei Messwerte weggelassen, da bei hinreichend
großen Spannungen kein lineares Verhältnis mehr zu sehen ist und somit die lineare Regression
ungenauer macht. Die Anpassungsfunktion wird mit Python erstellt.

\begin{figure}[H]
  \centering
  \includegraphics{plot.pdf}
  \caption{Lineare Regression der Messwerte für das grüne Licht}
  \label{fig:plot}
\end{figure}

Die Paramater betragen:
\begin{align*}
  a_1 &= \SI{6,2(1)e-5}{\sqrt\ampere\per\volt} \\
  b_1 &= \SI{3.4(2)e-5}{\sqrt\ampere}
\end{align*}

Die Fehler werden mit Python berechnet.

Für $U_B$ muss der Schnittpunkt der Geraden mit der U-Achse ermittelt werden, die x-Koordinate ist dann $U_B$.
Es gilt $U_B = -\frac{b}{a}$. Für $U_B$ ergibt sich somit:
\begin{equation*}
  U_B = ?
\end{equation*}


In Tabelle 2 wird die gemessene Stromstärke und die zugehörige Spannung für das rote Licht dargestellt.

\begin{table}[H]
  \centering
  \caption{Gemessene Stromstärke bei verschiedenen Spannungen für rotes Licht}
  \label{tab:Spannungsamplitude}
  \begin{tabular}{c c}
    \toprule
    $U/$V & $I/$nA \\
    \midrule
     0 & 0 \\
     1,0 & 0,014 \\
     2,0 & 0,040 \\
     3,0 & 0,080 \\
     4,0 & 0,125 \\
     6,0 & 0,16 \\
     8,0 & 0,18 \\
    10,0 & 0,20 \\
    15,0 & 0,20 \\
    \bottomrule
  \end{tabular}
\end{table}

Mit diesen Messwerten wird erneut eine lineare Regression durchgeführt, wobei wieder die letzten drei Messwerte
aud den oben genannten Gründen vernachlässigt werden.

\begin{figure}[H]
  \centering
  \includegraphics{plotrot.pdf}
  \caption{Lineare Regression der Messwerte für das rote Licht}
  \label{fig:plot}
\end{figure}

Die Paramater betragen:
\begin{align*}
  a_2 &= \SI{0.67(9)e-5}{\sqrt\ampere\per\volt} \\
  b_2 &= \SI{0.5(3)e-5}{\sqrt\ampere}
\end{align*}

Daraus folgt für $U_B$:
\begin{align*}
  U_B = ?
\end{align*}


In Tabelle 3 wird die gemessene Stromstärke und die zugehörige Spannung für das violette Licht dargestellt.

\begin{table}[H]
  \centering
  \caption{Gemessene Stromstärke bei verschiedenen Spannungen für violettes Licht}
  \label{tab:Spannungsamplitude}
  \begin{tabular}{c c}
    \toprule
    $U/$V & $I/$nA \\
    \midrule
    -0,9 &	0,016 \\
    -0,7 &	0,072 \\
    -0,5 &	0,141 \\
    -0,3 &	0,31 \\
    -0,1 &	0,50 \\
    0	   &  0,60 \\
    0,2	 &  0,89 \\
    0,4	 &  1,00 \\
    0,6	 &  1,30 \\
    0,8	 &  1,60 \\
    1,0  &  2,20 \\
    1,5	 &  3,60 \\
    2,0  &  4,80 \\
    3,0  &  7,40 \\
    4,0	 &  8,50 \\
    6,0	 &  12,0 \\
    8,0	 &  14,0 \\
    10,0 &  15,0 \\
    15,0 &  16,0 \\
    \bottomrule
  \end{tabular}
\end{table}

Mit diesen Messwerten wird erneut eine lineare Regression durchgeführt, wobei die letzten fünf Messwerte
aus den oben genannten Gründen vernachlässigt werden.


\begin{figure}[H]
  \centering
  \includegraphics{plotviolett.pdf}
  \caption{Lineare Regression der Messwerte für das violette Licht}
  \label{fig:plot}
\end{figure}


Die Parameter betragen:
\begin{align*}
  a_3 &= \SI{6.9(1)e-5}{\sqrt\ampere\per\volt} \\
  b_3 &= \SI{7.6(2)e-5}{\sqrt\ampere}
\end{align*}

Daraus folgt für $U_B$:
\begin{align*}
  U_B = ?
\end{align*}

In Tabelle 4 wird die gemessene Stromstärke und die zugehörige Spannung für das violette Licht dargestellt.

\begin{table}[H]
  \centering
  \caption{Gemessene Stromstärke bei verschiedenen Spannungen für ultraviolettes Licht}
  \label{tab:Spannungsamplitude}
  \begin{tabular}{c c}
    \toprule
    $U/$V & $I/$nA \\
    \midrule
    -1,0 &	0,006 \\
    -0,8 &	0,02 \\
    -0,6 &	0,05 \\
    -0,3 &	0,10 \\
    0,0 &	0,17 \\
    0,5   &  0,34 \\
    1,0	 &  0,53 \\
    2,0	 &  1,00 \\
    4,0	 &  2,20 \\
    5,0	 &  2,90 \\
    \bottomrule
  \end{tabular}
\end{table}

Mit diesen Messwerten wird erneut eine lineare Regression durchgeführt, wobei die letzten fünf Messwerte
aus den oben genannten Gründen vernachlässigt werden.

\begin{figure}[H]
  \centering
  \includegraphics{plotuv.pdf}
  \caption{Lineare Regression der Messwerte für das ultraviolette Licht.}
  \label{fig:plotuv}
\end{figure}

Die Parameter betragen:
\begin{align*}
  a_3 &= \SI{2.7(1)e-5}{\sqrt\ampere\per\volt} \\
  b_3 &= \SI{4.0(2)e-5}{\sqrt\ampere}
\end{align*}

Daraus folgt für $U_B$:
\begin{align*}
  U_B = ?
\end{align*}
