\section{Diskussion}
\label{sec:Diskussion}

Die Messwerte der Spannungen weichen in dem Bereich um das Maximum herum sichtbar ab, wodurch sich größere Abweichungen
in den experimentell ermittelten Breiten der Resonanzkurven ergeben. ein ähnlicher Verlauf der Spannung anhand
der Theoriekurve ist dennoch ersichtlich.
Ein systematischer Fehler liegt bei dem Tastkopf vor,
welcher bei nur kleinen Störungen die Kurve deutlich verändert und somit die Messwerte verfälscht. Außerdem
werden die Widerstände von Kondensator und Spule vernachlässigt.


Die experimentell ermittelte Resonanzüberhöhung weicht nur gering von der theoretischen ab. Daraus wird
geschlossen, dass die sich aus dem Graphen ergebende Resonanzfrequenz präzise dargestellt wird.

Mithilfe der Phasenverschiebung gelang eine sehr genaue Bestimmung der Resonanzfrequenz. Die Werte
für $f_1$ und $f_2$ weichen hingegen verhältnismäßig stark von den Theoriewerten ab. Systematische
Fehler können dabei schon das einfache ablesen der Werte aus den erstellten Diagrammen sein.
Jedoch rechtfertigt dies noch nicht eine solch starke Abweichung. Ein weiterer Grund
kann widerum der oben breits erwähnte Tastkopf sein, mit dem zudem z.B. die eigentlich
erwartete Frequenzabhängigkeit bezüglich der Messung der Eingangsspannung erst gar nicht
sichtlich festgestellt werden konnte.
