\section{Diskussion}
\label{sec:Diskussion}

Die Messwerte der Spannungen weichen in dem Bereich um das Maximum herum sichtbar ab, wodurch sich größere Abweichungen
in den experimentell ermittelten Breiten der Resonanzkurven ergeben. ein ähnlicher Verlauf der Spannung anhand
der Theoriekurve ist dennoch ersichtlich.
Ein systematischer Fehler liegt bei dem Tastkopf vor,
welcher bei nur kleinen Störungen die Kurve deutlich verändert und somit die Messwerte verfälscht. Außerdem
werden die Widerstände von Kondensator und Spule vernachlässigt.


Die experimentell ermittelte Resonanzüberhöhung weicht nur gering von der theoretischen ab. Daraus wird
geschlossen, dass die sich aus dem Graphen ergebende Resonanzfrequenz präzise dargestellt wird.
