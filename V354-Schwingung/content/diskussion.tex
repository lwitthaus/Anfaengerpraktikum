\section{Diskussion}
\label{sec:Diskussion}

Die Messwerte der Spannungen weichen in dem Bereich um das Maximum herum sichtbar ab, wodurch größere Abweichungen
in den experimentell ermittelten Breiten der Resonanzkurven ergeben. Ein systematischer Fehler ist der Tastkopf,
welcher bei nur kleinen Störungen die Kurve deutlicverändert und somit die Messwerte verfälscht. Außerdem
werden die Widerstände der von Kondensator und Spule vernachlässigt.  


Die experimentell ermittlete Resonanzüberhöhung weicht nur gering von der theoretischen ab. Daraus wird
geschlossen, dass die aus dem Graphen abgelsene Frequenz des Maximums präzise bestimmt wird.
