\section{Auswertung}
\label{sec:Auswertung}

Die Bauteile des RLC-Kreises haben folgende Werte:
\begin{align*}
  L &= \SI{10.11(3)}{\milli\henry} \\
  C &= \SI{2.098(6)}{\nano\farad} \\
  R_1 &= \SI{48.1(1)}{\ohm} \\
  R_2 &= \SI{509.5(5)}{\ohm}
\end{align*}

\subsection{Berechnung des Dämpfungswiderstandes}

Die gemessenen Spannungsamplituden $U_C$ in Abhängigkeit von der Zeit $t$ wird in Tabelle 1 dargestellt.

\begin{table}[H]
  \centering
  \caption{Gemessene Spannungen in Abhängigkeit von der Zeit}
  \label{tab:Spannungsamplitude}
  \begin{tabular}{c c}
    \toprule
    $t/$s & $U_C/$V \\
    \midrule
    29.00 & 0.69 \\
    58.00 & 0.58 \\
    88.00 & 0.50 \\
    117.0 & 0.42 \\
    147.0 & 0.34 \\
    176.0 & 0.28 \\
    205.0 & 0.24 \\
    235.0 & 0.20 \\
    264.0 & 0.17 \\
    294.0 & 0.14 \\
    323.0 & 0.12 \\
    353.0 & 0.10 \\
    \bottomrule
  \end{tabular}
\end{table}

Die Messwerte und die Regression werden in Abbildung 5 dargestellt. Die Ausgleichsfunktion ist $Ae^{-2\pi \mu t}$.

\begin{figure}[H]
  \centering
  \includegraphics{plot.pdf}
  \caption{Spannungsamplituden und Einhüllende}
  \label{fig:plot}
\end{figure}

Die Parameter betragen:
\begin{align*}
  A &= (0.827 \pm 0.007)V \\
  \mu &= (954 \pm 10)\frac{1}{s}
\end{align*}

Die Parameter und deren Fehler werden mit Python berechnet.
Die Abklingzeit $T_{ex}$ wird mit Gleichung (6) berechnet und der effektive Dämpfungswiderstand $R_{eff}$ mit
der Gleichung (4).
\begin{align*}
T_{ex} &= \SI{1.67(2)e4}{\second} \\
R_{eff} &= \SI{121(1)}{\ohm}
\end{align*}

Der in der Schaltung eingebaute Widerstand beträgt $R = \SI{48.1(1)}{\ohm}$. Zusätzlich wird der
Generatorinnenwiderstand $R_G = 50 \Omega$ dazu gerechnet. Der theoretische Widerstand beträgt $R_t = \SI{98.1(1)}{\ohm}$
Die relative Abweichung des berechneten Wertes zu dem Theoriewert beträgt $18.9\%$.


Der theoretische Wert des Widerstandes für den aperiodischen Grenzfall wird mit der Bedingung
$R_{ap} =\sqrt{\frac{4L}{C}}$ bestimmt. Der Theoriewert beträgt:
\begin{equation*}
  R_{ap,t} = \SI{4390(9)}{\ohm}
\end{equation*}

Der aus dem Versuch berechnete Wert beträgt:
\begin{equation*}
  R_{ap,e} = 3100 \symup{\Omega}
\end{equation*}

Die absolute Abweichung der beiden Werte beträgt $(1290 \pm 9) \symup{\Omega}$.
Diese Diskrepanz ist durch die nicht beachteten Widerstände von Spule, Kondensator und Kabeln zu erklären. Zusätzlich
kann bei der Messung nicht genau festgestellt werden, wann genau der Fall des aperiodischen Grenzfalles eintritt.


\subsection{Bestimmung der Güte und der Breite der Schwingung}

In Tabelle 2 werden die gemessenen Spannungen sowie der Abstand der Nulldurchgänge $a$ zu den jeweiligen
Frequenzen dargestellt. Zudem wird die Phasenverschiebung $\varphi$ angegebn, welche in späteren Rechnungen
benötigt wird.

\begin{table}[H]
  \centering
  \caption{Gemessene Spannungen, Frequenzen, Abstände und Phasenwinkel}
  \label{tab:Spannungsamplitude}
  \begin{tabular}{c c c c c c}
    \toprule
    $f/$Hz & $U_0/V$ & $U_C/V$ & $a/\symup{\mu s}$ & $\frac{U_C}{U_0}$ & $\varphi/$rad   \\
    \midrule
    5000 & 3.28  &  3.36  &  2.00  &  1.02  &  0.06   \\
   10000 & 3.28  &  3.60  &  2.00  &  1.10  &  0.13   \\
   15000 & 3.28  &  4.00  &  2.00  &  1.22  &  0.19   \\
   20000 & 3.28  &  4.80  &  2.00  &  1.46  &  0.25   \\
   25000 & 3.28  &  6.48  &  2.40  &  1.98  &  0.38   \\
   27000 & 3.20  &  7.80  &  2.60  &  2.44  &  0.44   \\
   28000 & 3.28  &  8.40  &  3.20  &  2.56  &  0.56   \\
   29000 & 3.28  &  9.20  &  3.40  &  2.80  &  0.62   \\
   30000 & 3.28  &  10.2  &  3.60  &  3.11  &  0.68   \\
   31000 & 3.28  &  11.0  &  4.60  &  3.35  &  0.90   \\
   32000 & 3.28  &  11.8  &  6.20  &  3.60  &  1.25   \\
   33000 & 3.28  &  12.2  &  6.60  &  3.72  &  1.37   \\
   34000 & 3.28  &  12.2  &  7.60  &  3.72  &  1.62   \\
   35000 & 3.28  &  11.4  &  8.40  &  3.48  &  1.85   \\
   36000 & 3.28  &  10.4  &  9.40  &  3.17  &  2.13   \\
   37000 & 3.28  &  9.40  &  9.80  &  2.87  &  2.28   \\
   38000 & 3.28  &  8.20  &  9.80  &  2.50  &  2.34   \\
   39000 & 3.28  &  7.20  &  9.80  &  2.20  &  2.40   \\
   40000 & 3.28  &  6.40  &  10.20 &  1.95  &  2.56   \\
   45000 & 3.28  &  4.00  &  10.40 &  1.22  &  2.94   \\
   50000 & 3.28  &  2.56  &  9.80  &  0.78  &  3.08   \\
   55000 & 3.28  &  1.84  &  9.60  &  0.56  &  3.32   \\
   60000 & 3.24  &  1.44  &  8.00  &  0.44  &  3.02   \\
    \bottomrule
  \end{tabular}
\end{table}


Der Quotient $\frac{U_C}{U_0}$ wird gegen die Frequenz in einem halblogarithmisch skaliertem Diagramm
dargestellt.


\begin{figure}[H]
  \centering
  \includegraphics{plot2.pdf}
  \caption{Spannungsamplituden in Abhängigkeit von der Frequenz}
  \label{fig:plot2}
\end{figure}
