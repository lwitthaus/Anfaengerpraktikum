\section{Auswertung}
\label{sec:Auswertung}

\subsection{Berechnung des Dämpfungswiderstandes}

Die Bauteile des RLC-Kreises haben folgende Werte:
\begin{align*}
  L &= \SI{10.11(3)}{\milli\henry} \\
  C &= \SI{2.098(6)}{\nano\farad} \\
  R_1 &= \SI{48.1(1)}{\ohm} \\
  R_2 &= \SI{509.5(5)}{\ohm}
\end{align*}

Die gemessenen Spannungsamplituden $U_C$ in Abhängigkeit von der Zeit $t$ wird in Tabelle 1 dargestellt.

\begin{table}[H]
  \centering
  \caption{Gemessene Spannungen in Abhängigkeit von der Zeit}
  \label{tab:Spannungsamplitude}
  \begin{tabular}{c c}
    \toprule
    $t/$s & $U_C/$V \\
    \midrule
    29,00 & 0,69 \\
    58,00 & 0,58 \\
    88,00 & 0,50 \\
    117,0 & 0,42 \\
    147,0 & 0,34 \\
    176,0 & 0,28 \\
    205,0 & 0,24 \\
    235,0 & 0,20 \\
    264,0 & 0,17 \\
    294,0 & 0,14 \\
    323,0 & 0,12 \\
    353,0 & 0,10 \\
    \bottomrule
  \end{tabular}
\end{table}

Die Messwerte und die Regression werden in Abbildung 5 dargestellt. Die Ausgleichsfunktion ist $Ae^{-2\pi \mu t}$.

\begin{figure}[H]
  \centering
  \includegraphics{plot.pdf}
  \caption{Spannungsamplituden und Einhüllende}
  \label{fig:plot}
\end{figure}

Die Parameter betragen:
\begin{align*}
  A &= (0,827 \pm 0,007) \symup{V} \\
  \mu &= (954 \pm 10)\frac{1}{\symup{s}}
\end{align*}

Die Parameter und deren Fehler werden mit Python berechnet.
Die Abklingzeit $T_{ex}$ wird mit Gleichung (6) berechnet und der effektive Dämpfungswiderstand $R_{eff}$ mit
der Gleichung (4).
\begin{align*}
T_{ex} &= \SI{1.67(2)e4}{\second} \\
R_{eff} &= \SI{121(1)}{\ohm}
\end{align*}

Der in der Schaltung eingebaute Widerstand beträgt $R = \SI{48.1(1)}{\ohm}$. Zusätzlich wird der
Generatorinnenwiderstand $R_G = 50 \Omega$ dazu gerechnet. Der theoretische Widerstand beträgt $R_t = \SI{98.1(1)}{\ohm}$
Die relative Abweichung des berechneten Wertes zu dem Theoriewert beträgt $18.9\%$.


Der theoretische Wert des Widerstandes für den aperiodischen Grenzfall wird mit der Bedingung
$R_{ap} =\sqrt{\frac{4L}{C}}$ bestimmt. Der Theoriewert beträgt:
\begin{equation*}
  R_{ap,t} = \SI{4390(9)}{\ohm}
\end{equation*}

Der aus dem Versuch berechnete Wert beträgt:
\begin{equation*}
  R_{ap,e} = 3100 \symup{\Omega}
\end{equation*}

Die absolute Abweichung der beiden Werte beträgt $(1290 \pm 9) \symup{\Omega}$.



\subsection{Bestimmung der Güte und der Breite der Schwingung}

In Tabelle 2 werden die gemessenen Spannungen sowie der Abstand der Nulldurchgänge $a$ zu den jeweiligen
Frequenzen dargestellt. Zudem wird die Phasenverschiebung $\varphi$ angegeben, welche in späteren Rechnungen
benötigt wird.

\begin{table}[H]
  \centering
  \caption{Gemessene Spannungen, Frequenzen, Abstände und Phasenwinkel}
  \label{tab:Spannungsamplitude}
  \begin{tabular}{c c c c c c}
    \toprule
    $f/$Hz & $U_0/V$ & $U_C/V$ & $a/\symup{\mu s}$ & $\frac{U_C}{U_0}$ & $\varphi/$rad   \\
    \midrule
    5000 & 3,28  &  3,36  &  2,00  &  1,02  &  0,06   \\
   10000 & 3,28  &  3,60  &  2,00  &  1,10  &  0,13   \\
   15000 & 3,28  &  4,00  &  2,00  &  1,22  &  0,19   \\
   20000 & 3,28  &  4,80  &  2,00  &  1,46  &  0,25   \\
   25000 & 3,28  &  6,48  &  2,40  &  1,98  &  0,38   \\
   27000 & 3,20  &  7,80  &  2,60  &  2,44  &  0,44   \\
   28000 & 3,28  &  8,40  &  3,20  &  2,56  &  0,56   \\
   29000 & 3,28  &  9,20  &  3,40  &  2,80  &  0,62   \\
   30000 & 3,28  &  10,2  &  3,60  &  3,11  &  0,68   \\
   31000 & 3,28  &  11,0  &  4,60  &  3,35  &  0,90   \\
   32000 & 3,28  &  11,8  &  6,20  &  3,60  &  1,25   \\
   33000 & 3,28  &  12,2  &  6,60  &  3,72  &  1,37   \\
   34000 & 3,28  &  12,2  &  7,60  &  3,72  &  1,62   \\
   35000 & 3,28  &  11,4  &  8,40  &  3,48  &  1,85   \\
   36000 & 3,28  &  10,4  &  9,40  &  3,17  &  2,13   \\
   37000 & 3,28  &  9.40  &  9,80  &  2,87  &  2,28   \\
   38000 & 3,28  &  8,20  &  9,80  &  2,50  &  2,34   \\
   39000 & 3,28  &  7,20  &  9,80  &  2,20  &  2,40   \\
   40000 & 3,28  &  6,40  &  10,20 &  1,95  &  2,56   \\
   45000 & 3,28  &  4,00  &  10,40 &  1,22  &  2,94   \\
   50000 & 3,28  &  2,56  &  9,80  &  0,78  &  3,08   \\
   55000 & 3,28  &  1,84  &  9,60  &  0,56  &  3,32   \\
   60000 & 3,24  &  1,44  &  8,00  &  0,44  &  3,02   \\
    \bottomrule
  \end{tabular}
\end{table}

Die Phasenverschiebung $\varphi$ wird dabei mit $\varphi = 2 \pi a \cdot f$ berechnet.


Der Quotient $\frac{U_C}{U_0}$ wird gegen die Frequenz in einem halblogarithmisch skaliertem Diagramm
dargestellt.


\begin{figure}[H]
  \centering
  \includegraphics{plot2.pdf}
  \caption{Spannungsamplituden in Abhängigkeit von der Frequenz}
  \label{fig:plot2}
\end{figure}

Aus den Messwerten wird die Resonanzüberhöhung $q_{exp} = 3,72$ bei $\omega_0 = 213 \cdot 10^{3} \symup{Hz}$ abgelesen . Der theoretische Wert der
Güte $q_{theo}$ wird mit Gleichung (11) berechnet.
\begin{equation*}
  q_{theo} = 3,99 \pm 0,01
\end{equation*}

Der Fehler von $q_{theo}$ wird der Gaußschen Fehlerfortpflanzung berechnet.:

\begin{align*}
  \sigma_q =  \sqrt{
      \left( \frac{\partial \left(\frac{1}{\omega_0 R_2C}\right)}{\partial R_2} \sigma_{R_2} \right)^{\!\! 2} +
      \left( \frac{\partial \left(\frac{1}{\omega_0 R_2C} \right)}{\partial C} \sigma_{C} \right)^{\!\! 2}
    } \\
    \Rightarrow \sigma_q = \sqrt{\left(\frac{1}{\omega_0 R_2^2 C}\sigma_{R_2} \right)^2 +
    \left(\frac{1}{\omega_0 R_2 C^2} \sigma_C \right)^2
    }
\end{align*}


Die relative Abweichung des experimentell ermittelten Wertes zu dem Theoriewert beträgt $6,8\%$.

Um die Breite der Resonanzkurve zu ermitteln, wird der Frequenzbereich um das Maximum herum linear dargestellt.

\begin{figure}[H]
  \centering
  \includegraphics{plot5.pdf}
  \caption{Lineare Darstellung des Frequenzbereiches in der Umgebung des Maximums}
  \label{fig:plot5}
\end{figure}

Das Maximum hat den Wert $\frac{U_{C,max}}{U_0} = 3,72$. Fällt das Maximum auf den Bruchteil $\frac{1}{\sqrt{2}}$ seines Maximalwertes:
$\frac{1}{\sqrt{2}} \cdot \frac{U_{C,max}}{U_0} = 2,63$, betragen
die zugehörigen Frequenzen $f_+$ und $f_-$:
\begin{align*}
  f_+ &= 37500 \symup{Hz} \\
  f_- &= 28500 \symup{Hz} \\
  f_+ - f_- &= 9000 \symup{Hz}
\end{align*}

Der Theoriewert von $f_+ - f_-$ wird mit Gleichung (11) berechnet:
\begin{equation*}
  f_+ - f_- = (7811 \pm 18) \:\symup{kHz}
\end{equation*}

Der Fehler wird mit der Gaußschen Fehlerfortpflanzung bestimmt:
\begin{align*}
  \sigma_f = f_0 \sqrt{
      \left(\frac{\partial \left(\frac{1}{q}\right)}{\partial q} \sigma_{q} \right)^{\!\! 2}
      } \\
      \Rightarrow \sigma_f = f_0 \sqrt{\left(\frac{1}{q^2} \sigma_q \right)^2}
\end{align*}
Die relative Abweichung des experimentell bestimmten Wertes zu dem Theoriewert beträgt $13,2 \%$.


\subsection{Bestimmung der Resonanzfrequenz aus der Phasenverschiebung}

Die Phasenverschiebung $\varphi$ wird gegen die Frequenz $f$ in einem halblogarithmischen
Diagramm aufgetragen.

\begin{figure}[H]
  \centering
  \includegraphics{plot3.pdf}
  \caption{Phasenverschiebung von Eingangs- und Kondensatorspannung (halblogarithmisch).}
  \label{fig:plot3}
\end{figure}



Bei der Berechnung der Theoriekurve ist dabei zu beachten, dass die Kondensatorspannung
bereits phasenverschoben ist.
Der Bereich um die Resonanzfrequenz wird noch einmal in einem linearen Diagramm aufgetragen.

\begin{figure}[H]
  \centering
  \includegraphics{plot4.pdf}
  \caption{Phasenverschiebung von Eingangs- und Kondensatorspannung (linear).}
  \label{fig:plot4}
\end{figure}

Aus diesen Graphen folgt für die Resonanzfrequenz:
\begin{equation*}
  f_{res} \approx 34 \:\symup{kHz}
\end{equation*}

Für die Frequenzen $f_1$ und $f_2$, bei denen die Phasenverschiebung gerade
$\frac{\pi}{4}$ bzw. $\frac{3\pi}{4}$ beträgt, folgt:
\begin{equation*}
  f_1 \approx 30,5 \:\symup{kHz}
\end{equation*}
\begin{equation*}
  f_2 \approx 38 \:\symup{kHz}
\end{equation*}

Der theoretische Wert für die Resonanzfrequenz ergibt sich aus Gleichung (10).
Dabei wird auch der Generatorinnenwiderstand berücksichtigt.
\begin{equation*}
  f_{res} = (33,99 \pm 0,07) \:\symup{kHz}
\end{equation*}

Der Fehler von $f_{res}$ wird mit der Gaußschen Fehlerfortpflanzung berechnet:

\begin{align*}
  \sigma_f = 2\pi \sqrt{
  \left( \frac{\partial \left(\sqrt{\frac{1}{LC} - \frac{R_2^2}{2L^2}}\right)}{\partial R_2} \sigma_{R_2} \right)^{\!\! 2} +
  \left( \frac{\partial \left(\sqrt{\frac{1}{LC} - \frac{R_2^2}{2L^2}}\right)}{\partial L} \sigma_{L} \right)^{\!\! 2} +
  \left( \frac{\partial \left(\sqrt{\frac{1}{LC} - \frac{R_2^2}{2L^2}}\right)}{\partial C} \sigma_{C} \right)^{\!\! 2}
      } \\
    \Rightarrow  \sigma_f = 2 \pi \sqrt{\left(\frac{R_2}{2L^2 \sqrt{\frac{1}{LC} - \frac{R_2^2}{2L^2}}} \sigma_{R_2} \right)^2 +
  \left(\frac{\frac{R_2^2}{L^3}-\frac{1}{CL^2}}{2\sqrt{\frac{1}{LC} - \frac{R_2^2}{2L^2}}} \sigma_L \right)^2 +
  \left(\frac{1}{2LC^2  \sqrt{\frac{1}{LC} - \frac{R_2^2}{2L^2}}} \sigma_C \right)^2}
\end{align*}

Die relative Abweichung entspricht also circa 0,02\%.

Aus dem zuvor bereits berechneten Wert der Breite ergibt sich zu dem experimentell
ermittelten Wert von $f_1 - f_2 \approx 7500$ Hz eine relative Abweichung von 4,1\%.



%\begin{equation}
%  \sigma_f = 2\pi \sqrt{
%      \left( \frac{\partial \left(\sqrt{\frac{1}{LC} - \frac{R^2}{2L^2}}\right)}{\partial R} \sigma_{R} \right)^{\!\! 2} +
%      \left( \frac{\partial \left(\sqrt{\frac{1}{LC} - \frac{R^2}{2L^2}}\right)}{\partial L} \sigma_{L} \right)^{\!\! 2} +
%      \left( \frac{\partial \left(\sqrt{\frac{1}{LC} - \frac{R^2}{2L^2}}\right)}{\partial C} \sigma_{C} \right)^{\!\! 2}
%    }
%\end{equation}

%\begin{equation}
%  \sigma_f = 2 \pi \sqrt{\left(\frac{R}{2L^2 \sqrt{\frac{1}{LC} - \frac{R^2}{2L^2}}} \sigma_R \right)^2 +
%  \left(\frac{\frac{R^2}{L^3}-\frac{1}{CL^2}}{2\sqrt{\frac{1}{LC} - \frac{R^2}{2L^2}}} \sigma_L \right)^2 +
%  \left(\frac{1}{2LC^2  \sqrt{\frac{1}{LC} - \frac{R^2}{2L^2}}} \sigma_C \right)^2}
%\end{equation}
