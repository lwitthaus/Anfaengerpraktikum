\section{Theorie}
\label{sec:Theorie}

Brückenschaltungen können jede physikalische Größe messen, welche sich eindeutig durch den
elektrischen Widerstand darstellbar ist. In einer Brückenschaltung untersucht man die Potentialdifferenz zweier elektrischer Leiter
in Abhängigkeit von ihrem Widerstandsverhältnis.


\begin{figure}[H]
  \centering
  \includegraphics[height=5cm]{bruecke.PNG}
  \caption{Darstellung einer allgemeinen Brückenschaltung}
  \label{fig:Brückenschaltung}
\end{figure}

Mithilfe der beiden Kirchhoffschen Gesetze lässt sich die Spannungsz $U$ der Brückenschaltung wie folgt darstellen.
\begin{equation}
  U = \frac{R_2 R_3 - R_1 R_4}{(R_3+R_4)(R_1+R_2)}U_S
\end{equation}

Dabei ist $U_S = I_1(R_1+R_2)$ die Speisespannung. Für den Fall $R_2R_3=R_1R_4$ verschwindet die gemessene Spannung, wodurch der Fall
der abgeglichene Brücke vorliegt. Ist $R_1$ ein unbekannter Widerstand, kann dieser durch das variieren eines der anderen Widerstände bestimmt werden,
indem untersucht wird, wann die Spannung verschwindet.

Sind die Widerstände komplex, also $R= X + jY$, so müssen zwei Bedingungen gleichzeitig erfüllt sein:
\begin{align}
  X_1X_4 -Y_1Y_4 &= X_2X_3 - Y_2Y_3 \\
  X_1Y_4 +X_4Y_1 &= X_2Y_3 + X_3Y_2
\end{align}


Im folgenden werden verschiedene
Brückenschaltungen beschrieben, welche die für diesen Versuch gesuchten Größen ermitteln.

\subsection{Wheatstonesche Brücke}
Die Wheatstonesche Brückenschaltung besteht aus vier Widerständen.

\begin{figure}[H]
  \centering
  \includegraphics[height=5cm]{wheat.PNG}
  \caption{Darstellung einer Wheatstoneschen Brückenschaltung}
  \label{fig:Wheat}
\end{figure}

Für den unbekannten Widerstand $R_x$ gilt die Abgleichbedingung:
\begin{equation}
  R_x = R_2 \frac{R_3}{R_4}
\end{equation}

\subsection{Kapazitätsmessbrücke}
Mit dieser Schaltung werden Kapazitäten bestimmt, weshalb mit komplexen Widerständen gerechnet werden muss.

\begin{figure}[H]
  \centering
  \includegraphics[height=5cm]{kapazitaet.PNG}
  \caption{Darstellung einer Kapazitätsmessbrücke}
  \label{fig:kapazität}
\end{figure}

Jeder Kondensator besitzt einen realen Widerstand, welcher auch in Abbildung 3 dargestellt wird.
Der Widerstand eines Kondensators ist:
\begin{equation}
  R_{C_{real}} = R - \frac{j}{\omega C}
\end{equation}

Die unbekannten Größen $C_x$ und $R_x$ betragen:
\begin{align}
  C_x &= C_2 \frac{R_4}{R_3} \\
  R_x &= R_2 \frac{R_3}{R_4}
\end{align}

Für Frequenzen in dem Bereich von $10^4$Hz und kleiner gilt $R_2 \approx 0$.


\subsection{Induktivitätsmessbrücke}
Der komplexe Widerstand einer realen Spule ist definiert durch:
\begin{equation}
  R_{L_{real}} = R - j \omega L
\end{equation}

Die Induktivitätsmessbrücke ist analog zur Kapazitätsmessbrücke aufgebaut.

\begin{figure}[H]
  \centering
  \includegraphics[height=5cm]{induktivitaet.PNG}
  \caption{Darstellung einer Induktivitätsmessbrücke}
  \label{fig:induktivitaet}
\end{figure}

Die Abgleichbedingungen sind:

\begin{align}
  L_x &= L_2 \frac{R_3}{R_4} \\
  R_x &= R_2 \frac{R_3}{R_4}
\end{align}

Damit diese Schaltung präzise Werte misst, muss die Spule einen möglichst kleinen Widerstand haben, für kleine Frequenzen
ist dieser dennoch zu groß, weshalb in diesem Fall die Maxwell-Brücke verwemdmet wird.

\subsection{Maxwell-Brücke}

\begin{figure}[H]
  \centering
  \includegraphics[height=5cm]{maxwell.PNG}
  \caption{Darstellung einer Maxwell-Brücke}
  \label{fig:maxwell}
\end{figure}

Der Widerstand $R_2$ soll dabei bekannt sein und als Abgleichelement dienen $R_3$ und $R_4$.
Für $L_x$ und $R_x$ ergibt sich dann:
\begin{align}
  L_x &= R_2 R_3 C_4 \\
  R_x &= \frac{R_2 R_3}{R_4}
\end{align}


\subsection{Wien-Robinson-Brücke}
Prinzipiell lassen sich Brückenschaltungen bei allen Frequenzen abgleichen. Es gibt jedoch einen
Frequenzereich in dem ein Abgleich unter optimalen Bedingungen durchgeführt werden kann. Mit der
Wien-Robinson-Brücke soll dieser Bereich untersucht werden.

\begin{figure}[H]
  \centering
  \includegraphics[height=5cm]{wien.PNG}
  \caption{Darstellung einer Wien-Robinson-Brücke}
  \label{fig:wien}
\end{figure}
Der Kondensator sollte in dieser schaltung möglichst geringe Verluste besitzen.

Für das Betragsquadrat des Verhältnisses von der Brückenspannung $U_{Br}$ und der Speisespannung $U_S$ folgt:
\begin{equation}
  \left|\frac{U_{Br}}{U_S} \right|^2 = \frac{(\omega^2 R^2 C^2 -1)^2}{9((1- \omega^2 R^2 C^2) + 9 \omega^2 R^2 C^2)}
\end{equation}

Wird das Frequenzverhältnis $\Omega = \frac{\omega}{\omega_0}$ eingeführt, folgt aus Gleichung (13):
\begin{equation}
  \left|\frac{U_{Br}}{U_S} \right|^2 = \frac{1}{9} \frac{(\Omega^2 -1)^2}{(1- \Omega^2)^2 + 9 \Omega^2}
\end{equation}

Gleichung (14) hat die Form eines Filters. Die Wien-Robinson-Brücke schwächt Schwingungen in einem
Bereich um $\omega_0$ ab. Mit dieser Schaltung wird der Klirr-Faktor gemessen. Ein Sinusgenerator kann keine
Sinuschwingungen ohne Oberwellen erzeugen. Eine ideale Sinusschwingung besteht nur aus einer Grundschwingung.
Der Klirr-Faktor ist ein Maß für die Qualität des Sinusgenerators. Hat der Sinusgenerator die Sperrfrequenz $\omega_0$
der Wien-Robinson-Brücke, bleiben an dem Ausgang nur noch die Oberwellen. Die Summe deren Amplituden wird mit einem
Breitband-Millivoltmeter gemessen.
