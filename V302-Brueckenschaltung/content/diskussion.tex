\section{Diskussion}
\label{sec:Diskussion}
Die berechneten Fehler zu den ermittelten Messgrößen befinden sich bei den meisten
Brückenschaltungen in dem Toleranzbereich. Die Fehler sind durch systematische Fehler zu erklären.
Das Minimieren der Spannung ist nur endlich präzise durchführbar. Der Widerstand der Kabel und des Oszilloskop wird
idealisiert als Null angenommen, was in der Realität nicht der Fall ist.
Bei der Induktivitätsmessbrücke ist dies
für $L_x$ nicht der Fall. Da die Spannung für zwei Spulen nicht auf Null minimiert werden kann, lässt sich dies
als primärer Faktor für die Abweichung erklären. Die Induktivität, welche mit der Maxwell-Brücke berechnet wird, weist
nur kleine Abweichungen im Mittelwert auf und kann somit als präziser angesehen werden. Die Kondensatoren
zu den beiden Kapazitäten $C_{x_1}$ und $C_{x_2}$ besitzen einen sehr kleinen Widerstand, sodass mit dem
Messverfahren der Widerstand nicht berechnet werden kann.
Die gemessene Frequenzabhängigkeit der Wien-Robinson-Brücke entspricht sehr genau der theoretisch erwarteten Abhängigkeit. Vor allem
im Bereich um die errechnete Frequenz $f_0$ liegen die gemessenen Werte unmittelbar auf der Theoriekurve. Ihr tatsächlicher Wert sollte
also nur gering von diesem abweichen.
Auch der Klirrfaktor fällt wie erwartet klein aus, was bedeutet, dass es sich bei der Eingangsspannung um ein relativ reines
Sinussignal handelt. Es ist dabei auch anzumerken, dass das Einstellen der entsprechenden Frequenz, bei welcher das Signal
verschwinden sollte aufgrund der oben bereits genannten idealisierten Annahmen nur mit endlicher Genauigkeit durchgeführt werden kann.
Jedoch erweist sich die tatsächliche Bewertung dieses Werts als schwierig, da kein entsprechender Vergleichswert existiert.
