\section{Diskussion}
\label{sec:Diskussion}
Die berechneten Fehler zu den ermittelten Messgrößen befinden sich bei den meisten
Brückenschaltungen in dem Toleranzbereich. Die Fehler sind durch systematische Fehler zu erklären.
Das Minimieren der Spannung ist nur endlich präzise durchführbar. Der Widerstand der Käbel und des Oszilloskop wird
idealisiert als Null angenommen, was in der Realität nicht der Fall ist.
Bei der Induktivitätsmessbrücke ist dies
für $L_x$ nicht der Fall. Da die Spannung für zwei Spulen nicht auf Null minimiert werden kann, lässt sich dies
als primärer Faktor für die Abweichung erklären. Die Induktivität, welche mit der Mawell-Brücke berechnet wird, weist
nur kleine Abweichungen im Mittelwert auf und kann somit als präziser angesehen werden. Die Kondensatoren
zu den beiden Kapazitäten $C_{x_1}$ und $C_{x_2}$ besitzen einen sehr kleinen Widerstand, sodass mit dem
Messverfahren der Widerstand nicht berechnet werden kann.
