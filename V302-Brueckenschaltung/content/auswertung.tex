\section{Auswertung}
\label{sec:Auswertung}

\subsection{Wheatsonesche Brückenschaltung}
Der unbekannte Widerstand $R_{x_1}$ (Wert 10) in der Brückenschaltung wird mit Gleichung (4) bestimmt. Die gemessenen Widerstände
werden in Tabelle 1 dargestellt.

\begin{table}[H]
  \centering
  \caption{Gemessene Widerstände für $R_{x_1}$}
  \label{tab:Widerstand}
  \begin{tabular}{c c c}
    \toprule
    $R_2/ \symup{\Omega}$ & $R_3/ \symup{\Omega}$ & $R_4 / \symup{\Omega}$ \\
    \midrule
     500  &  321 & 679 \\
    1000  &  190 & 820  \\
     332  &  417 & 583  \\
    \bottomrule
  \end{tabular}
\end{table}

Daraus folgt für den Mittelwert  des unbekannten Widerstands:
\begin{align*}
  R_{x_1} = (235 \pm 1\text{(Bauteile)} \pm 2\text{(Mittelwert)}) \, \symup{\Omega}
\end{align*}

Das Bauteil $R_2$ hat eine Toleranz von $0.2\%$ und das Verhältnis $\frac{R_3}{R_4}$ hat eine Abweichung von
$0.5\%$.

Für den zweiten unbekannten Widerstand $R_{x_2}$ (Wert 12) werden die gemessenen Widerstände in Tabelle 2 dargestellt.

\begin{table}[H]
  \centering
  \caption{Gemessene Widerstände für $R_{x_2}$}
  \label{tab:Widerstand}
  \begin{tabular}{c c c}
    \toprule
    $R_2/ \symup{\Omega}$ & $R_3/ \symup{\Omega}$ & $R_4 / \symup{\Omega}$ \\
    \midrule
      312  &  530 & 470 \\
      500  &  428 & 572  \\
     1000  &  263 & 737  \\
    \bottomrule
  \end{tabular}
\end{table}

Daraus folgt für den Mittelwert des unbekannten Widerstandes:
\begin{align*}
  R_{x_2} = (368 \pm 2\text{(Bauteile)} \pm 8\text{(Mittelwert)}) \, \symup{\Omega}
\end{align*}


\subsection{Berechnung der Kapazität}
Zur Bestimmung der Messgrößen $C_x$ und $R_x$ werden die gemessenen Wiederstände und Kapazitäten in
Tabelle 3 dargestellt.

\begin{table}[H]
  \centering
  \caption{Gemessene Widerstände und Kapazitäten für $C_{x_1} und R_{x_1}$}
  \label{tab:Widerstand}
  \begin{tabular}{c c c c}
    \toprule
    $C_2/$nF & $R_2/ \symup{\Omega}$ & $R_3/ \symup{\Omega}$ & $R_4 / \symup{\Omega}$ \\
    \midrule
    992 & 0 & 600 & 400 \\
    597 & 0 & 474 & 526 \\
    399 & 0 & 376 & 624 \\
    \bottomrule
  \end{tabular}
\end{table}

Die unbekannten Größen $C_{x_1}$ (Wert 1) und $R_{x_1}$ werden mit Gleichung (6) und (7) berechnet.
\begin{align*}
  C_{x_1} = (6.62 \pm 0.04\text{(Bauteile)} \pm 0.01\text{(Mittelwert)}) \cdot 10^{2} \symup{nF} \\
  R_{x_1} = (0 \pm 0\text{(Bauteile)} \pm 0\text{(Bauteile)}) \, \symup{\Omega}
\end{align*}

Die gemessene Werte der zweiten Messreihe werden in Tabelle 4 dargestellt.

\begin{table}[H]
  \centering
  \caption{Gemessene Widerstände und Kapazitäten für $C_{x_2} und R_{x_2}$}
  \label{tab:Widerstand}
  \begin{tabular}{c c c c}
    \toprule
    $C_2/$nF & $R_2/ \symup{\Omega}$ & $R_3/ \symup{\Omega}$ & $R_4 / \symup{\Omega}$ \\
    \midrule
    992 & 0 & 703 & 297 \\
    597 & 0 & 587 & 413 \\
    399 & 0 & 487 & 513 \\
    \bottomrule
  \end{tabular}
\end{table}

Die Kapazität $C_{x_2}$ (Wert 3) wird analog berechnet.

\begin{align*}
  C_{x_2} = (4.20 \pm 0.02 \text{(Bauteile)} \pm 0.01 \text{(Mittelwert)}) \cdot 10^{2} \symup{nF} \\
  R_{x_2} = (0 \pm 0 \text{(Bauteile)} \pm 0 \text{(Mittelwert)}) \, \symup{\Omega}
\end{align*}


Die gemessene Werte der dritten Messreihe werden in Tabelle 5 dargestellt.

\begin{table}[H]
  \centering
  \caption{Gemessene Widerstände und Kapazitäten für $C_{x_3} und R_{x_3}$}
  \label{tab:Widerstand}
  \begin{tabular}{c c c c}
    \toprule
    $C_2/$nF & $R_2/ \symup{\Omega}$ & $R_3/ \symup{\Omega}$ & $R_4 / \symup{\Omega}$ \\
    \midrule
    992 & 170 & 771 & 229 \\
    597 & 281 & 671 & 329 \\
    399 & 422 & 576 & 424 \\
    \bottomrule
  \end{tabular}
\end{table}

Die Kapazität $C_{x_3}$ (Wert 8) eines verlustbehafteten Kondensator und $R_{x_3}$ werden analog berechnet.
Der Fehler von $R_2$ beträgt $3\%$

\begin{align*}
  C_{x_3} = (2.94 \pm 0.02 \text{(Bauteile)} \pm 0.01 \text{(Mittelwert)}) \cdot 10^{2} \symup{nF} \\
  R_{x_3} = (571 \pm 17 \text{(Bauteile)} \pm 3\text{(Mittelwert)}) \cdot 10^{2} \symup{nF}
\end{align*}



\subsection{Berechnung der Induktivität mit der Induktivitätsbrücke}

Die gemessenen Induktivitäten und Widerstände werden in Tabelle 6 dargestellt.

\begin{table}[H]
  \centering
  \caption{Gemessene Widerstände und Induktivitäten für $L_{x_1} und R_{x_1}$}
  \label{tab:Widerstand}
  \begin{tabular}{c c c c}
    \toprule
    $L_2/$mH & $R_2/ \symup{\Omega}$ & $R_3/ \symup{\Omega}$ & $R_4 / \symup{\Omega}$ \\
    \midrule
    27.5 &   61 & 607 & 393 \\
    20.1 & 1000 &  91 & 909 \\
    14.6 & 1000 &  99 & 901 \\
    \bottomrule
  \end{tabular}
\end{table}


Die Induktivität $L_{x_1}$ und der Widerstand $R_{x_1}$ werden mit den Gleichungen (9) und (10) berechnet. Dabei hat
das Bauteil $L_2$ eine Toleranz von $0.2\%$.
