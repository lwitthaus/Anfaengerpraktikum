\section{Auswertung}
\label{sec:Auswertung}

\subsection{Wheatsonesche Brückenschaltung}
Der unbekannte Widerstand $R_{x_1}$ (Wert 10) in der Brückenschaltung wird mit Gleichung (4) bestimmt. Die gemessenen Widerstände
werden in Tabelle 1 dargestellt.

\begin{table}[H]
  \centering
  \caption{Gemessene Widerstände für $R_{x_1}$}
  \label{tab:Widerstand}
  \begin{tabular}{c c c}
    \toprule
    $R_2/ \symup{\Omega}$ & $R_3/ \symup{\Omega}$ & $R_4 / \symup{\Omega}$ \\
    \midrule
     500  &  321 & 679 \\
    1000  &  190 & 820  \\
     332  &  417 & 583  \\
    \bottomrule
  \end{tabular}
\end{table}

Daraus folgt für den Mittelwert  des unbekannten Widerstands:
\begin{align*}
  R_{x_1} = (235 \pm 1\text{(Bauteile)} \pm 2\text{(Mittelwert)}) \, \symup{\Omega}
\end{align*}

Für den zweiten unbekannten Widerstand $R_{x_2}$ (Wert 12) werden die gemesenen Widerstände in Tabelle 2 dargestellt.

\begin{table}[H]
  \centering
  \caption{Gemessene Widerstände für $R_{x_2}$}
  \label{tab:Widerstand}
  \begin{tabular}{c c c}
    \toprule
    $R_2/ \symup{\Omega}$ & $R_3/ \symup{\Omega}$ & $R_4 / \symup{\Omega}$ \\
    \midrule
      312  &  530 & 470 \\
      500  &  428 & 572  \\
     1000  &  263 & 737  \\
    \bottomrule
  \end{tabular}
\end{table}

Daraus folgt für den Mittelwert des unbekannten Widerstandes:
\begin{align*}
  R_{x_2} = (368 \pm 2\text{(Bauteile)} \pm 8\text{(Mittelwert)}) \, \symup{\Omega}
\end{align*}
