\section{Auswertung}
\label{sec:Auswertung}

\subsection{Wheatsonesche Brückenschaltung}
Der unbekannte Widerstand $R_{x_1}$ (Wert 10) in der Brückenschaltung wird mit Gleichung (4) bestimmt. Die gemessenen Widerstände
werden in Tabelle 1 dargestellt.

\begin{table}[H]
  \centering
  \caption{Gemessene Widerstände für $R_{x_1}$}
  \label{tab:Widerstand}
  \begin{tabular}{c c c c}
    \toprule
    $R_2/ \symup{\Omega}$ & $R_3/ \symup{\Omega}$ & $R_4 / \symup{\Omega}$ & $R_{x_1}/ \symup{\Omega}$\\
    \midrule
     500  &  321 & 679  &  236.5 \\
    1000  &  190 & 820  &  232.0  \\
     332  &  417 & 583  &  237.4  \\
    \bottomrule
  \end{tabular}
\end{table}



Das Bauteil $R_2$ hat eine Toleranz von $0.2\%$ und das Verhältnis $\frac{R_3}{R_4}$ hat eine Abweichung von
$0.5\%$.

Daraus folgt für den Mittelwert  des unbekannten Widerstands:
\begin{align*}
  R_{x_1} = (235 \pm 1\text{(Bauteile)} \pm 2\text{(Mittelwert)}) \, \symup{\Omega}
\end{align*}

Die Fehler der unbekannnten Größen werden allesamt mit Python berechnet.

Für den zweiten unbekannten Widerstand $R_{x_2}$ (Wert 12) werden die gemessenen Widerstände in Tabelle 2 dargestellt.

\begin{table}[H]
  \centering
  \caption{Gemessene Widerstände für $R_{x_2}$}
  \label{tab:Widerstand}
  \begin{tabular}{c c c c}
    \toprule
    $R_2/ \symup{\Omega}$ & $R_3/ \symup{\Omega}$ & $R_4 / \symup{\Omega}$ & $R_{x_2}/ \symup{\Omega}$ \\
    \midrule
      312  &  530 & 470 & 374.0 \\
      500  &  428 & 572 & 357.0  \\
     1000  &  263 & 737 & 374.5  \\
    \bottomrule
  \end{tabular}
\end{table}

Daraus folgt für den Mittelwert des unbekannten Widerstandes:
\begin{align*}
  R_{x_2} = (368 \pm 2\text{(Bauteile)} \pm 8\text{(Mittelwert)}) \, \symup{\Omega}
\end{align*}


\subsection{Berechnung der Kapazität}
Zur Bestimmung der Messgrößen $C_x$ und $R_x$ werden die gemessenen Wiederstände und Kapazitäten in
Tabelle 3 dargestellt.

\begin{table}[H]
  \centering
  \caption{Gemessene Widerstände und Kapazitäten für $C_{x_1} und R_{x_1}$}
  \label{tab:Widerstand}
  \begin{tabular}{c c c c c c}
    \toprule
    $C_2/$nF & $R_2/ \symup{\Omega}$ & $R_3/ \symup{\Omega}$ & $R_4 / \symup{\Omega}$ & $C_{x_1}/$nF & $R_{x_1} / \symup{\Omega}$\\
    \midrule
    992 & 0 & 600 & 400 & 6.62  & 0\\
    597 & 0 & 474 & 526 & 6.63  & 0\\
    399 & 0 & 376 & 624 & 6.62  & 0\\
    \bottomrule
  \end{tabular}
\end{table}

Die unbekannten Größen $C_{x_1}$ (Wert 1) und $R_{x_1}$ werden mit Gleichung (6) und (7) berechnet.
\begin{align*}
  C_{x_1} = (6.62 \pm 0.04\text{(Bauteile)} \pm 0.01\text{(Mittelwert)}) \cdot 10^{2} \symup{nF} \\
  R_{x_1} = (0 \pm 0\text{(Bauteile)} \pm 0\text{(Bauteile)}) \, \symup{\Omega}
\end{align*}

Die gemessene Werte der zweiten Messreihe werden in Tabelle 4 dargestellt.

\begin{table}[H]
  \centering
  \caption{Gemessene Widerstände und Kapazitäten für $C_{x_2} und R_{x_2}$}
  \label{tab:Widerstand}
  \begin{tabular}{c c c c c c}
    \toprule
    $C_2/$nF & $R_2/ \symup{\Omega}$ & $R_3/ \symup{\Omega}$ & $R_4 / \symup{\Omega}$ & $C_{x_2}/$nF &  $R_{x_2} / \symup{\Omega}$\\
    \midrule
    992 & 0 & 703 & 297 &  4.20 & 0\\
    597 & 0 & 587 & 413 &  4.20 & 0\\
    399 & 0 & 487 & 513 &  4.12 & 0\\
    \bottomrule
  \end{tabular}
\end{table}

Die Kapazität $C_{x_2}$ (Wert 3) wird analog berechnet.

\begin{align*}
  C_{x_2} = (4.20 \pm 0.02 \text{(Bauteile)} \pm 0.01 \text{(Mittelwert)}) \cdot 10^{2} \symup{nF} \\
  R_{x_2} = (0 \pm 0 \text{(Bauteile)} \pm 0 \text{(Mittelwert)}) \, \symup{\Omega}
\end{align*}


Die gemessene Werte der dritten Messreihe werden in Tabelle 5 dargestellt.

\begin{table}[H]
  \centering
  \caption{Gemessene Widerstände und Kapazitäten für $C_{x_3} und R_{x_3}$}
  \label{tab:Widerstand}
  \begin{tabular}{c c c c c c}
    \toprule
    $C_2/$nF & $R_2/ \symup{\Omega}$ & $R_3/ \symup{\Omega}$ & $R_4 / \symup{\Omega}$ & $C_{x_x}/$nF & $R_{x_3}/ \symup{\Omega}$  \\
    \midrule
    992 & 170 & 771 & 229 &  2.95 & 667  \\
    597 & 281 & 671 & 329 &  2.93 & 673  \\
    399 & 422 & 576 & 424 &  2.94 & 673  \\
    \bottomrule
  \end{tabular}
\end{table}

Die Kapazität $C_{x_3}$ (Wert 8) eines verlustbehafteten Kondensator und $R_{x_3}$ werden analog berechnet.
Der Fehler von $R_2$ beträgt $3\%$

\begin{align*}
  C_{x_3} = (2.94 \pm 0.02 \text{(Bauteile)} \pm 0.01 \text{(Mittelwert)}) \cdot 10^{2} \symup{nF} \\
  R_{x_3} = (571 \pm 17 \text{(Bauteile)} \pm 3\text{(Mittelwert)}) \cdot 10^{2} \symup{nF}
\end{align*}



\subsection{Berechnung der Induktivität mit der Induktivitätsmessbrücke}

Die gemessenen Induktivitäten und Widerstände werden in Tabelle 6 dargestellt.

\begin{table}[H]
  \centering
  \caption{Gemessene Widerstände und Induktivitäten für $L_{x} und R_{x}$ mit der Induktivitätsmessbrücke}
  \label{tab:Widerstand}
  \begin{tabular}{c c c c c c}
    \toprule
    $L_2/$mH & $R_2/ \symup{\Omega}$ & $R_3/ \symup{\Omega}$ & $R_4 / \symup{\Omega}$ & $L_x/$mH & $R_x / \symup{\Omega}$  \\
    \midrule
    27.5 &   61 & 607 & 393 & 42.5   & 94.2 \\
    20.1 & 1000 &  91 & 909 &  2.0   & 100.0\\
    14.6 & 1000 &  99 & 901 &  1.6   & 110.0\\
    \bottomrule
  \end{tabular}
\end{table}


Die Induktivität $L_{x}$ und der Widerstand $R_{x}$ werden mit den Gleichungen (9) und (10) berechnet. Dabei hat
das Bauteil $L_2$ eine Toleranz von $0.2\%$.

\begin{align*}
  L_{x} = (15.37 \pm 0.09\text{(Bauteile)} \pm 19.18\text{(Mittelwert)}) \, \symup{mH} \\
  R_x = (101 \pm 3\text{(Bauteile)} \pm 6\text{(Mittelwert)}) \, \symup{\Omega}
\end{align*}


\subsection{Berechnung der Induktivität mit der Maxwell-Brücke}
Die Induktivität derselben Spule wird nun mit der Maxwell-Brücke gemessen. Die gemessenen Werte, werden in Tabelle 7
dargestellt.

\begin{table}[H]
  \centering
  \caption{Gemessene Widerstände und Induktivitäten für $L_{x} und R_{x}$ mit der Maxwell-Brücke}
  \label{tab:Widerstand}
  \begin{tabular}{c c c c c}
    \toprule
     $R_2/ \symup{\Omega}$ & $R_3/ \symup{\Omega}$ & $R_4 / \symup{\Omega}$ & $L_x/$mH & $R_x/\symup{\Omega}$\\
    \midrule
    332 & 203 & 715 & 40.2 & 94 \\
    500 & 139 & 715 & 41.5 & 97 \\
   1000 &  68 & 715 & 40.6 & 95 \\
    \bottomrule
  \end{tabular}
\end{table}

Die Induktivität $L_{x}$ und der Widerstand $R_{x}$ werden mit den Gleichungen (11) und (12) berechnet. Die
Toleranz von $R_3$ und $R_4$ beträgt $3\%$.

\begin{align*}
  L_x = (40 \pm 1\text{(Bauteile)} \pm 1 \text{(Mittelwert)})  \, \symup{mH} \\
  R_x = (95 \pm 4\text{(Bauteile)} \pm 1\text{(Mittelwert)})   \, \symup{\Omega}
\end{align*}




\subsection{Berechnung der Frequenzabhängigkeit der Wien-Robinson-Brücke}
Die gemessenen Werte von Speise- und Brückenspannung der Wien-Robinson-Brücke bei
den entsprechenden Frequenzen werden in Tabelle \ref{tab:wien} dargestellt.

\begin{table}[H]
  \centering
  \caption{Gemessene Spannungen $U_{Br}$ und $U_s$ für Frequenzen $f$.}
  \label{tab:wien}
  \begin{tabular}{c c c}
    \toprule
     $f / \symup{Hz}$ & $U_{Br} / \symup{V}$ & $U_s / \symup{V}$ \\
    \midrule
    20	   &  1.51	   &  4.38   \\
    50	   &  1.47	   &  4.56\\
    100	   &  1.33     & 	4.64\\
    150	   &  1.09     & 	4.56\\
    200	   &  0.86     & 	4.48\\
    250	   &  0.63     & 	4.40\\
    300	   &  0.46     & 	4.44\\
    350	   &  0.30     & 	4.40\\
    400	   &  0.18     & 	4.40\\
    420	   &  0.13     & 	4.27\\
    440	   &  0.08     & 	4.27\\
    460	   &  0.04     & 	4.27\\
    470	   &  0.03     & 	4.27\\
    480	   &  0.02     & 	4.27\\
    482.7  & 	0.02     & 	4.27\\
    490	   &  0.02     & 	4.27\\
    500	   &  0.03     & 	4.27\\
    550	   &  0.12     & 	4.27\\
    600	   &  0.20     & 	4.32\\
    700	   &  0.35     & 	4.32\\
    800	   &  0.46     & 	4.32\\
    900	   &  0.57     & 	4.24\\
    1000	 &  0.66	   &  4.24\\
    2000	 &  1.08	   &  4.12\\
    3000	 &  1.24	   &  4.08\\
    4000	 &  1.29	   &  4.04\\
    5000	 &  1.31	   &  4.00\\
    10000  & 	1.35     &  3.96\\
    \bottomrule
  \end{tabular}
\end{table}

Aus diesen Werten lässt sich das Verhältnis von Brücken- und Speisespannung gegen
das Frequenzverhältnis $\frac{f}{f_0}$ auftragen.
Für $f_0$ ergibt sich gemäß Gleichung (14):
\begin{equation*}
  f_0 = 482,7 \: \symup{Hz}
\end{equation*}

\begin{figure}[H]
  \centering
  \includegraphics{plot.pdf}
  \caption{Spannungsverhältnis von Brücken- und Speisespannung in Abhängigkeit von der Frequenz.}
  \label{fig:plot}
\end{figure}

Die Theoriekurve ergibt sich dabei aus Gleichung (15).

\subsection{Berechnung des Klirrfaktors}
Zur Berechnung des Klirrfaktors wird zur Vereinfachung angenommen, dass die Summe
der Oberwellen nur aus der zweiten Oberwelle besteht.

Der Klirrfaktor ergibt sich dann aus Gleichung (16). Wobei $U_2$ durch Gleichung
(17) ausgedrückt wird. f(2) ist dabei gegeben durch:
\begin{equation*}
  f(2) = \sqrt{\frac{(2^2-1)^2}{9\cdot((1-2^2)^2+9\cdot2^2)}} = \frac{\sqrt{5}}{15}
\end{equation*}

Nun ist $U_1$ die Speisespannung und $U_{Br}$ die Brückenspannungbei der entsprechenden Freqeuenz $f_0$
(kann Tabelle \ref{tab:wien} entnommen werden). Für den Klirrfaktor ergibt sich damit:
\begin{equation*}
  k = \frac{0.02\symup{V}\cdot15}{4.27\symup{V}\cdot\sqrt{5}} = 0.03
\end{equation*}
