\section{Durchführung}
\label{sec:Durchführung}
Bei diesem Versuch besteht der Aufbau im wesentlichen aus einer optischen Bank, auf
der verschieden optische Elemente wie Linsen und Lichtquellen mit hilfe von Reitern
verschoben werden können.

Zu Beginn sollen die Brennweiten zweier Linsen überprüft werden. Dazu ist eine Halogenlampe als Lichtquelle auf
der Bank platziert. Als Gegenstand wird ein sogenanntes "Perl L" (mehrere Glaskugeln, angeordnet zu einem "L")
verwendet. Dieses wird direkt vor der Lampe platziert. Dahinter wird dann eine Linse (Brennweite $f = 100$mm) und ein Schirm aufgebaut.
Bei fester Gegenstandsweite wird der Schirm solange verschoben, bis auf ihm eine möglichst scharfe Abbildung des
Gegenstands zu erkennen ist. Dann werden jeweils die Gegenstandsweite und die Bildweite in dieser Konfiguration
gemessen. Dies geschieht über an der optischen Banke befindliche Meterangaben. Diese Messung wird für insgesamt 10 unterschiedliche
Gegenstandsweiten und zusätzlich für eine weitere Linse durchgeführt. Als zweite Linse wird dabei eine mit Wasser befüllbare Linse
gewählt.

Anschließend wird dann eine Messung nach der Methode von Bessel durchgeführt. Dafür bleibt die Versuchsanordung im
Prinzip gleich. Es werden nun lediglich die Gegenstandsweite und die Bildweite für die beiden möglichen Konfigurationen
scharfer Abbildungen bei einem konstanten Abstand zwichen Gegenstand und Schirm gemessen. Dies wird für insgesamt 10 unterschiedliche
Abstände gemacht und auch noch einmal mit einem roten bzw. blauen Filter vor dem Gegenstand wiederholt (dabei jedoch nur für 5 unterschiedliche
Abstände).

Die Messung nach der Methode von Abbe erfolgt dann analog zur ersten Messung. Die einzelne Linse wird jedoch durch ein Linsensystem aus einer
Sammel- und einer Zerstreungslinse ersetzt und die gemessenen Gegenstands- und Bildweiten zu einem vorher festgelegten Referenzpunkt A gemessen.
Dieser Punkt ist der Mittelpunkt der Sammellinse und bewegt sich folglich bei Änderung der Linsenposition mit.
