\section{Diskussion}
\label{sec:Diskussion}
Die berechneten Abbildungsmaßstäbe mit dem Abbildungsgesetz liegen innerhalb ihrer Standardabweichung, welche
jedoch bei beiden Werten relativ groß ist. Die Abweichungen lassen sich durch systematische Fehler erklären.
Das scharf Stellen des Bildes kann für viele Positionen nur ungenau vorgenommen werden, da ein
scharfes Bild sich bei Verschiebung oft nur minimal ändert. Dies führt zur ungenauen Werten für die Bildweite.
Zusätzlich führt das Abmessen der Bildgröße mit einem Geodreieck zu ungenauen Werten für die Bildgröße.
Statistische Fehler sind als primäre Fehlerquelle auszuschließen.

Die Abweichung der berechneten Brennweiten ist ebenfalls relativ gering. Dennoch liegt der
Theoriewert der Brennweite nur von 5 Standardabweichungen des berechneten Mittelwertes der Brennweite. Auch hier lässt
sich die Abweichung durch die obigen genannten Gründe erklären.

Die ermittelten Brennweiten der mit Wasser gefüllten Linse liegen innerhalb von 8 Standardabweichungen voneinander.
Trotz der systematischen Fehler ist die prozentuale Abweichung mit $\SI{5.2}{} \, \%$ relativ gering. %Jedoch ist
%zu erkennen, dass die y-Koordinate nicht in dem Schnittpunkt, wenn fü
Das Verfahren ist also im Grunde geeignet um Brennweiten von Linsen bestimmen zu können.


Bei der Methode nach Bessel ergeben sich Werte, die wieder nur sehr gering von der tatsächlichen Brennweite der
Linse abweichen. Trotzdem liegen sie nicht innerhalb der Standardabweichung.
Bei der Messung mit Rot- und Blaufilter ist zu erwarten, dass die aus der Messung mit dem Rotfilter errechneten Werte
im Verhältnis zu den Werten ohne Filter vergrößern. Andersrum sollten sie sich bei der Messung mit dem Blaufilter verkleinern.
Der Grund dafür ist die unterschiedliche Wellenlänge. Blaues Licht daher stärker als rotes Licht. Bei den errechneten Werten
ist es schwierig über diesen Sachverhalt zu urteilen. Da die Fehler der Werte im Verhältnis zu dem Intervall, auf der sich
die Verschiebung offenbar ereignen muss groß sind, kann nicht genau festgestellt werden, ob es tatsächlich zu der erwarteten
Verschiebung kommt. Beide Werte liegen nämlich innerhalb der jeweils anderen Standardabweichung. Der tatsächliche Einfluss der unterschiedlichen Wellenlängen
kann sich also gegen die oben bereits genannten systematischen Fehler nicht durchsetzen.

Bei der Methode nach Abbe ergeben sich nahezu lineare Kurvenverläufe bei den beiden Diagrammen. Daher ist eine
lineare Regression problemlos möglich und auch die entstehenden Unsicherheiten fallen nicht sonderlich groß aus.
Trotzdem decken die beiden errechneten Werte sich nicht, was jedoch der Fall sein sollte, da die gleiche Brennweite
bestimmt wird. Da auch hier die Bildgrößen nur mit einem Geodreieck bestimmt werden und auch das Scharfstellen des Bildes
durch Augenmaß geschieht, ist dieser Sachverhalt durch die obigen Überlegungen bezüglich der systematischen Fehler
bereits erklärt.
