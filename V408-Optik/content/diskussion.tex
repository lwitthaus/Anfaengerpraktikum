\section{Diskussion}
\label{sec:Diskussion}
Die berechneten Abbildungsmaßstäbe mit dem Abbildungsgesetz liegen innerhalb ihrer Standardabweichung, welche
jedoch bei beiden Werten relativ groß ist. Die Abweichungen lassen sich durch systematische Fehler erklären.
Das scharf stellen des Bildes kann für viele Positionen nur ungenau vorgenommen werden, da ein
scharfes Bild sich bei Verschiebung oft nur minimal ändert. Dies führt zur ungenauen Werten für die Bildweite.
Zusätzlich führt das Abmessen der Bildgröße mit einem Geodreieck zu ungenauen Werten für die Bildgröße.
Statistische Fehler sind als primäre Fehlerquelle auszuschließen.

Die Abweichung der berechneten Brennweiten ist ebenfalls relativ gering. Dennoch liegt der
Theoriewert der Brennweite nur von 5 Standarabweichungen des berechneten Mittelwertes der Brennweite. Auch hier lässt
sich die Abweichung durch die obigen genannten Gründe erklären.

Die ermittelten Brennweiten der mit Wasser gefüllten Linse liegen innerhalb von 8 Standardabweichungen voneinander.
Trotz der systematischen Fehler ist die prozentuale Abweichung mit $4,6$\, \% relativ gering.
Das Verfahren ist also im Grunde geeignet um Brennweiten von Linsen bestimmen zu können. 
