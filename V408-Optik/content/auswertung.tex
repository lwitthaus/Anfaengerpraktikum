\section{Auswertung}
\label{sec:Auswertung}

\subsection{Bestimmung der Brennweite mit der Bildweite und der Gegenstandsweite}

Die gemessenen Gegenstandsweiten $g$, Bildweiten $b$ und Bildgrößen $B$ einer Linse, mit einer
Brennweite von 100, werden in Tabelle 1 dargestellt. Zusätzlich werden die daraus ermittelten
Brennweite angeben die mit der Linsengleichung berechnet werden, sowie die Abbildungsmaßstäbe $V_1 = B/G$ und $V_2=b/g$
um das Abbildungsgesetz zu überprüfen.
\begin{table}[H]
  \centering
  \caption{Gemessene und berechnete Daten einer Linse}
  \label{tab:Widerstand}
  \begin{tabular}{c c c c c c}
    \toprule
    $g/$cm  & $b/$cm & $B / $cm & $V_1$ & $V_2$ & $f/$cm \\
    \midrule
    12,5 &    56,8  & 13,2 & 4,71 &   4,54   &  10,25   \\
    12,0 &    72,0  & 17,5 & 6,25 &   6,00   &  10,29   \\
    11,5 &    97,1  & 24,7 & 8,82 &   8,44   &  10,28   \\
    13,5 &    40,8  & 8,8  & 3,14 &   3,02   &  10,14  \\
    14,0 &    36,6  & 7,6  & 2,71 &   2,61   &  10,13  \\
    14,5 &    32,9  & 6,5  & 2,32 &   2,27   &  10,06  \\
    15,0 &    30,8  & 5,9  & 2,11 &   2,05   &  10,09  \\
    15,5 &    29,0  & 5,3  & 1,89 &   1,87   &  10,10  \\
    16,0 &    27,9  & 4,9  & 1,75 &   1,74   &  10,17  \\
    16,5 &    25,5  & 3,5  & 1,25 &   1,55   &  10,02  \\
    \bottomrule
  \end{tabular}
\end{table}



Für den Mittelwert von der Brennweite ergibt sich $f = \SI{10.15(3)}{\centi\meter}$. Der
Mittelwert von $V_1 - V_2$ beträgt $\SI{0.09(6)}{}$.
%Die Abweichung der Abbildungsmaßstäbe voneinander beträgt $2,9$\%.

Aus den Messwerten lässt sich nun grafisch einen Wert für die Brennweite ermitteln. Die Gegenstandsweiten
werden auf die x-Achse und die Bildweiten auf die y-Achsen aufgetragen. Der erste Messwert von $g$ wird dann mit
dem letzten über eine Gerade verbunden, für die anderen Messwerte wird dies auch gemacht. Dies ist in
Abbildung 5 dargestellt.




\begin{figure}[H]
  \centering
  \includegraphics{plot.pdf}
  \caption{Aufgetragene Bildweiten und Gegenstandsweiten zur Bestimmung der Brennweite.}
  \label{fig:plot}
\end{figure}


Die x- und y-Koordinate des Schnittpunktes der einzelnen Geraden stellt nun die Brennweite dar.
Der Schnittpunkt der schwarzen Linien beschreibt den Literaturwert der Brennweite. Die blauen Linien den Schnittpunkt der
Geraden in dem Diagramm. Für die Brennweite kann
ein ungefährer von $\SI{10.3}{\centi\meter}$ ermittelt werden. Nach Herstellerangaben beträgt die Brennweite der Linse $\SI{10.3}{\centi\meter}$.



Dieses Verfahren wird mit einer Linse unbekannter Brennweite wiederholt. In Tabelle 2 werden
die gemessenen Werte zu dieser Linse dargestellt.

\begin{table}[H]
  \centering
  \caption{Gemessene und berechnete Daten einer Linse mit unbekannter Brennweite}
  \label{tab:Widerstand}
  \begin{tabular}{c c c}
    \toprule
    $g/$cm  & $b/$cm & $f/$cm \\
    \midrule
    9,5     &     42,0 & 7,75 \\
    9,0     &     62,6 & 7,87 \\
    10,0    &     35,8 & 7,82 \\
    10,5    &     29,4 & 7,74 \\
    11,0    &     24,9 & 7,63 \\
    11,5    &     22,2 & 7,58 \\
    12,0    &     20,1 & 7,51 \\
    12,5    &     18,1 & 7,39 \\
    13,0    &     17,8 & 7,51 \\
    13,5    &     16,8 & 7,49 \\
    \bottomrule
  \end{tabular}
\end{table}

Der Mittelwert der Brennweiten beträgt $\SI{7.63(5)}{\centi\meter}$.

Erneut werden $g$ und $b$ in einem Diagramm aufgetragen und Geraden durch alle Messwertepaare gelegt. Dies ist in Abbildung 6
dargestellt.

\begin{figure}[H]
  \centering
  \includegraphics{plot2.pdf}
  \caption{Aufgetragene Bildweiten und Gegenstandsweiten zur Bestimmung der Brennweite einer mit Wasser gefüllten Linse.}
  \label{fig:plot}
\end{figure}


Aus dem Diagramm lässt sich für die Brennweite ein ungefährer Wert von  $\SI{8.05}{\centi\meter}$. Dieser Wert wird aus
dem Schnittpunkt der vertikalen Linie mit der x_Achse entnommen. Die zugehörige y-Koordinate weicht für $y=8.05$ von dem
Schnittpunkt der Geraden ab.


\subsection{Bestimmung der Brennweite nach der Methode von Bessel}
Die Brennweite der verwendeten Linse beträgt $\SI{10}{\centi\meter}$.

In Tabelle \ref{tab:bessel} werden die gemessenen Abstände $e, g_1, b_1, g_2, b_2$ sowie die daraus nach Gleichung (3)
errechneten Brennweiten $f_1$ und $f_2$.
\begin{table}[H]
  \centering
  \caption{Messwerte und errechnete Brennweiten bei der Methode von Bessel}
  \label{tab:bessel}
  \begin{tabular}{c c c c c c c}
    \toprule
    $e$/cm  & $g_1/$cm & $b_1$/cm & $g_2$/cm & $b_2$/cm & $f_1$/cm & $f_2$/cm \\
    \midrule
    40  &   17,7  & 22,3  &  21,8 &   18,2 & 9,87  & 9,92 \\
    50   &  14,2  & 35,8   & 36,4  &  13,6 & 10,17 & 9,90 \\
    55   &  13,6  & 41,4   & 42,2  &  12,8 & 10,24 & 9,82 \\
    60  &   13,0  & 47,0   & 47,3  &  12,7 & 10,18 & 10,01 \\
    70  &   12,5 &  57,5  &  58,0  &  12,0 & 10,27 & 9,94 \\
    75  &   12,8 &  62,8  &  63,9  &  11,1 & 10,42 & 9,46 \\
    80  &   12,1 &  67,1  &  68,2  &  11,8 & 10,55 & 10,06 \\
    90  &   11,9 &  78,1  &  78,6  &  11,4 & 10,33 & 9,96 \\
    100  &  11,7 &  88,3  &  88,8  &  11,2 & 10,33 & 9,95 \\
    110  &  11,5 &  98,5  &  98,8  &  11,2 & 10,30 & 10,06  \\
    \bottomrule
  \end{tabular}
\end{table}

Daraus ergeben sich die folgenden Mittelwerte
\begin{align*}
  &f_1 = (10.26 \pm 0.06) \: \symup{cm} \\
  &f_2 = (9.91 \pm 0.06) \: \symup{cm}
\end{align*}

Aus $\sigma_1 = 1 - \frac{f_{\symup{s}}}{f_1}$ bzw. $\sigma_2 = 1 - \frac{f_{\symup{s}}}{f_2}$ ergeben sich relative Abweichungen von
\begin{align*}
  &\sigma_1 = 2,5 \% \\
  &\sigma_2 = -0,9 \%
\end{align*}

Auf dieselbe Weise werden die Brennweiten für die Messung mit einem Rotfilter vor dem Gegenstand errechnet. Die
entsprechenden Werte sind in Tabelle \ref{tab:besselrot} zu finden.
\begin{table}[H]
  \centering
  \caption{Messwerte und errechnete Brennweiten bei der Methode von Bessel mit Rotfilter}
  \label{tab:besselrot}
  \begin{tabular}{c c c c c c c}
    \toprule
    $e_{\symup{r}}$/cm  & $g_{\symup{1r}}/$cm & $b_{\symup{1r}}$/cm & $g_{\symup{2r}}$/cm & $b_{\symup{2r}}$/cm & $f_{\symup{1r}}$/cm & $f_{\symup{2r}}$/cm \\
    \midrule
    60  &   13,0 &  47,0  &  47,4  &  12,6 & 10,18 & 9,95  \\
    70   &  12,5 &  57,5  &  58,0  &  12,0 & 10,27 & 9,94 \\
    80   &  12,2 &  67,8  &  68,7  &  11,3 & 10,34 & 9,70 \\
    90   &  11,8 &  78,2  &  78,5  &  11,5 & 10,25 & 10,03 \\
    100  &  11,7 &  88,3  &  88,9  &  11,1 & 10,33 & 9,87 \\
    \bottomrule
  \end{tabular}
\end{table}

Daraus ergeben sich diesmal die folgenden Mittelwerte
\begin{align*}
  &f_{\symup{1r}} = (10.27 \pm 0.03) \: \symup{cm} \\
  &f_{\symup{2r}} = (9.90 \pm 0.06) \: \symup{cm}
\end{align*}

Es ergeben sich hierbei relative Abweichungen von
\begin{align*}
  &\sigma_{\symup{1r}} = 2,6 \% \\
  &\sigma_{\symup{2r}} = -1,0 \%
\end{align*}

Und ein weiteres Mal für die Messung mit einem Blaufilter vor dem Gegenstand. Die Werte sind Tabelle
\ref{tab:besselblau} zu entnehmen.
\begin{table}[H]
  \centering
  \caption{Messwerte und errechnete Brennweiten bei der Methode von Bessel mit Blaufilter}
  \label{tab:besselblau}
  \begin{tabular}{c c c c c c c}
    \toprule
    $e_{\symup{b}}$/cm  & $g_{\symup{1b}}/$cm & $b_{\symup{1b}}$/cm & $g_{\symup{2b}}$/cm & $b_{\symup{2b}}$/cm & $f_{\symup{1b}}$/cm & $f_{\symup{2b}}$/cm \\
    \midrule
    60  &   13,0 &  47,0  &  47,7  &  12,3 & 10,18 & 9,78 \\
    70   &  12,5 &  57,5  &  57,9  &  12,1 & 10,27 & 10,01 \\
    80   &  12,1 &  67,9  &  68,6  &  11,4 & 20,27 & 9,78 \\
    90   &  12,0 &  78,0  &  78,8  &  11,2 & 10,40 & 9,81 \\
    100  &  11,8 &  88,2  &  88,7  &  11,3 & 10,41 & 10,02 \\
    \bottomrule
  \end{tabular}
\end{table}

Daraus ergeben sich die Mittelwerte
\begin{align*}
  &f_{\symup{1b}} = (10.31 \pm 0.05) \: \symup{cm} \\
  &f_{\symup{2b}} = (9.88 \pm 0.06) \: \symup{cm}
\end{align*}

Daraus ergeben sich relative Abweichungen von
\begin{align*}
  &\sigma_{\symup{1b}} = 3,0 \% \\
  &\sigma_{\symup{2b}} = -1,2 \%
\end{align*}

\subsection{Bestimmung der Brennweite nach der Methode von Abbe}
Es wird ein Linsensystem aus einer Zerstreuungslinse mit $f_{\symup{z}} = -10\,\symup{cm}$ und einer Sammellinse
mit einer Brennweite von $f_{\symup{s}} = 10\,\symup{cm}$ verwendet.

Die bei der Methode von Abbe gemessenen Abstände $g'$ und $b'$ sowie die gemessene Bildgröße $B$
werden in Tabelle \ref{tab:abbe} aufgeführt.

\begin{table}[H]
  \centering
  \caption{Messwerte bei der Messung nach der Methode von Abbe}
  \label{tab:abbe}
  \begin{tabular}{c c c}
    \toprule
    $g'$/cm  & $b'$/cm & $B$/cm \\
    \midrule
    30,0  &  62,5  &  9,5 \\
    31,0  &  53,3  &  7,5 \\
    32,0  &  46,4  &  6,5 \\
    33,0  &  42,9  &  5,6 \\
    34,0  &  37,7  &  5,0 \\
    35,0  &  35,7  &  4,5 \\
    36,0  &  32,4  &  4,0 \\
    37,0  &  30,0  &  3,5 \\
    38,0  &  28,0  &  3,1 \\
    39,0  &  26,9  &  2,9 \\
    \bottomrule
  \end{tabular}
\end{table}

Wird $g'$ gegen $1+1/V$ aus Gleichung (4) aufgetragen, so ergibt sich das folgende Diagramm.

\begin{figure}[H]
  \centering
  \includegraphics{plot1.pdf}
  \caption{Diagramm zur Bestimmung der Brennweite aus $g'$.}
  \label{fig:plot1}
\end{figure}

Aus der linearen Regression ergibt sich eine Brennweite von
\begin{equation*}
  f_{\symup{g}} = (13,36 \pm 0,42)\: \symup{cm}
\end{equation*}

Wird $b'$ gegen $1+V$ aus Gleichung (5) aufgetragen, so ergibt sich das folgende Diagramm.

\begin{figure}[H]
  \centering
  \includegraphics{plot3.pdf}
  \caption{Diagramm zur Bestimmung der Brennweite aus $b'$.}
  \label{fig:plot3}
\end{figure}

Aus der linearen Regression ergibt sich eine Brennweite von
\begin{equation*}
  f_b = (15,45 \pm 0,31)\: \symup{cm}
\end{equation*}
