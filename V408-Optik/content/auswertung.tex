\section{Auswertung}
\label{sec:Auswertung}

\subsection{Bestimmung der Brennweite mit der Bildweite und der Gegenstandsweite}

Die gemessenen Gegenstandsweiten $g$, Bildweiten $b$ und Bildgrößen $B$ einer Linse, mit einer
Brennweite von 100, werden in Tablle 1 dargestellt. Zusätzlich werden die daraus ermittelten
Brennweite angeben die mit der Linsengleichung berechnet werden, sowie die Abbildungsmaßstäbe $V_1 = B/G$ und $V_2=b/g$
um das Abbildungsgesetz zu überprüfen.
\begin{table}[H]
  \centering
  \caption{Gemessene und berechnete Daten einer Linse}
  \label{tab:Widerstand}
  \begin{tabular}{c c c c c c}
    \toprule
    $g/$cm  & $b/$cm & $B / $cm & $V_1$ & $V_2$ & $f/$cm \\
    \midrule
    12,5 &    56,8  & 13,2 & 4.71 &   4.54   &  10,25   \\
    12,0 &    72,0  & 17,5 & 6.25 &   6.00   &  10,29   \\
    11,5 &    97,1  & 24,7 & 8.82 &   8.44   &  10,28   \\
    13,5 &    40,8  & 8,8  & 3.14 &   3.02   &  10,14  \\
    14,0 &    36,6  & 7,6  & 2.71 &   2.61   &  10,13  \\
    14,5 &    32,9  & 6,5  & 2.32 &   2.27   &  10,06  \\
    15,0 &    30,8  & 5,9  & 2.11 &   2.05   &  10,09  \\
    15,5 &    29,0  & 5,3  & 1.89 &   1.87   &  10,10  \\
    16,0 &    27,9  & 4,9  & 1.75 &   1.74   &  10,17  \\
    16,5 &    25,5  & 3,5  & 1.25 &   1.55   &  10,02  \\
    \bottomrule
  \end{tabular}
\end{table}



Für den Mittelwert von der Brennweite ergibt sich $f = \SI{10.15(3)}{\centi\meter}$. Der
Mitelwert von $V_1$ beträgt $V_1 = \SI{3.5(8)}{}$ und für den Mittelwert von $V_2$ ergibt sich
$\SI{3.4(8)}{}$.
Die Abweichung der Abbildungsmaßstäbe von einander beträgt $2,9$\%.

Aus den Messwerten lässt sich nun graphisch einen Wert für die Brennweite ermitteln. Die Gegenstandsweiten
werden auf die x-Achse und die Bildweiten auf die y-Achsen aufgetragen. Der erste Messwert von $g$ wird dann mit
dem letzten über eine Gerade verbunden, für die anderen Messwerte wird dies auch gemacht. Dies ist in
Abbildung 5 dargestellt.




\begin{figure}
  \centering
  \includegraphics{plot.pdf}
  \caption{Aufgetragene Bildweiten und Gegenstandsweiten zur Bestimmung der Brennweite.}
  \label{fig:plot}
\end{figure}


Die x-und y-Koordinate des Schnittpunktes der einzelnen Geraden stellt nun die Brennweite dar. Für die Brennweite kann
ein ungefährer von $10,3$cm abgelesen werden. Nach Herstellerangaben beträgt die Brennweite der Linse $10$cm.



Dieses Verfahren wird mit einer Linse unbekannter Brennweite wiederholt. In Tabelle 2 werden
die gemessenen Werte zu dieser Linse dargestellt.

\begin{table}[H]
  \centering
  \caption{Gemessene und berechnete Daten einer Linse mit unbekannter Brennweite}
  \label{tab:Widerstand}
  \begin{tabular}{c c c}
    \toprule
    $g/$cm  & $b/$cm & $f/$cm \\
    \midrule
    9.5     &     42.0 & 7.75 \\
    9.0     &     62.6 & 7.87 \\
    10.0    &     35.8 & 7.82 \\
    10.5    &     29.4 & 7.74 \\
    11.0    &     24.9 & 7.63 \\
    11.5    &     22.2 & 7.58 \\
    12.0    &     20.1 & 7.51 \\
    12.5    &     18.1 & 7.39 \\
    13.0    &     17.8 & 7.51 \\
    13.5    &     16.8 & 7.49 \\
    \bottomrule
  \end{tabular}
\end{table}

Der Mittelwert der Brennweiten beträgt $\SI{7.63(5)}{\centi\meter}$.

Erneut werden $g$ und $b$ in einem Diagramm aufgetragen und Geraden durch die Messwertepaare durchgelegt. Dies ist in Abbilung 6
dargestellt.

\begin{figure}
  \centering
  \includegraphics{plot2.pdf}
  \caption{Aufgetragene Bildweiten und Gegenstandsweiten zur Bestimmung der Brennweite einer mit Wasser gefüllten Linse.}
  \label{fig:plot}
\end{figure}


Aus dem Diagramm lässt sich für die Brennweite ein ungefährer Wert von  $f = 8,0$cm.
