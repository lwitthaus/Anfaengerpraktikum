\section{Auswertung}
\label{sec:Auswertung}

\subsection{Bestimmung der Empfindlichkeit}

Die Tabellen 1 bis 5 stellen die Daten für die gemessenen Verschiebungen in Abhängigkeit von
der Ablenkspannung dar.

\begin{table}[H]
  \centering
  \caption{Gemessene Verschiebung in Abhängigkeit von der Ablenkspannung mit $U_B=200$V}
  \label{tab:Spannungsamplitude}
  \begin{tabular}{c c}
    \toprule
    $U_d/$V & $D/$mm \\
    \midrule
    -19,72 & 0 \\
    -16,00 & 6 \\
    -12,50 & 12 \\
     -9,14 & 18 \\
     -5,72 & 24 \\
     -2,02 & 30 \\
      1,69 & 36 \\
      5,41 & 42 \\
      9,12 & 48 \\
    \bottomrule
  \end{tabular}
\end{table}

\begin{table}[H]
  \centering
  \caption{Gemessene Verschiebung in Abhängigkeit von der Ablenkspannung mit $U_B=250$V}
  \label{tab:Spannungsamplitude}
  \begin{tabular}{c c}
    \toprule
    $U_d/$V & $D/$mm \\
    \midrule
     -24,6 & 0 \\
     -20,3 & 6 \\
     -16,3 & 12 \\
     -11,8 & 18 \\
      -7,5 & 24 \\
      -2,9 & 30 \\
       1,8 & 36 \\
       6,8 & 42 \\
      11,8 & 48 \\
    \bottomrule
  \end{tabular}
\end{table}



\begin{table}[H]
  \centering
  \caption{Gemessene Verschiebung in Abhängigkeit von der Ablenkspannung mit $U_B=300$V}
  \label{tab:Spannungsamplitude}
  \begin{tabular}{c c}
    \toprule
    $U_d/$V & $D/$mm \\
    \midrule
     -29,5 & 0 \\
     -24,6 & 6 \\
     -19,1 & 12 \\
     -14,2 & 18 \\
      -9,2 & 24 \\
      -3,9 & 30 \\
       1,6 & 36 \\
       7,4 & 42 \\
      13,3 & 48 \\
    \bottomrule
  \end{tabular}
\end{table}


\begin{table}[H]
  \centering
  \caption{Gemessene Verschiebung in Abhängigkeit von der Ablenkspannung mit $U_B=350$V}
  \label{tab:Spannungsamplitude}
  \begin{tabular}{c c}
    \toprule
    $U_d/$V & $D/$mm \\
    \midrule
     -34,1 & 0 \\
     -28,2 & 6 \\
     -22,5 & 12 \\
     -16,2 & 18 \\
     -10,1 & 24 \\
      -4,1 & 30 \\
       2,0 & 36 \\
       8,7 & 42 \\
      15,2 & 48 \\
    \bottomrule
  \end{tabular}
\end{table}


\begin{table}[H]
  \centering
  \caption{Gemessene Verschiebung in Abhängigkeit von der Ablenkspannung mit $U_B=400$V}
  \label{tab:Spannungsamplitude}
  \begin{tabular}{c c}
    \toprule
    $U_d/$V & $D/$mm \\
    \midrule
     -32,2 & 6 \\
     -25,2 & 12 \\
     -18,2 & 18 \\
     -11,4 & 24 \\
      -4,2 & 30 \\
       2,9 & 36 \\
      10,5 & 42 \\
      17,8 & 48 \\
    \bottomrule
  \end{tabular}
\end{table}


Mit den Messwerten wird eine lineare Regression durchgefürht, welche in Abbildung 6 zusehen ist.

\begin{figure}
  \centering
  \includegraphics{plot11.pdf}
  \caption{Empfindlichkeiten bei verschiedenen Beschleunigungsspannungen}
  \label{fig:plot}
\end{figure}

Mit Python wird die Steigung der Geraden berechnet. \\

\begin{table}[H]
  \centering
  \caption{Empfindlichkeiten bei unterschiedlichen Beschleunigungspannungen}
  \label{tab:Spannungsamplitude}
  \begin{tabular}{c c}
    \toprule
    $U_B/$V & $\frac{D}{U_d} / 10^{-3}\symup{\frac{m}{V}}$ \\
    \midrule
    200 & $\SI{1.67(1)}{}$ \\
    250 & $\SI{1.32(2)}{}$ \\
    300 & $\SI{1.13(1)}{}$ \\
    350 & $\SI{0.98(1)}{}$ \\
    400 & $\SI{0.84(1)}{}$ \\
    \bottomrule
  \end{tabular}
\end{table}


Nun wird werden die ermittelten Empfindlichkeiten gegen $\frac{1}{U_B}$ aufgetragen. Eine weitere
lineare Regression wird durchgeführt.

\begin{figure}
  \centering
  \includegraphics{plot12.pdf}
  \caption{Gegen $1/U_B$ aufegtragene Empfindlichkeiten}
  \label{fig:plot}
\end{figure}

Die Steigung $a$ beträgt:
\begin{align*}
  a = \SI{0.327(8)}{\meter}
\end{align*}

Der Plattenabstand der Ablenkplatten ist $d = 0,38$cm. Die Länge der Platten beträgt $p= 1,9$cm und der
Abstand dieser zum Leuchtschirm beträgt $L = 14,3$cm.

Die Größe $\frac{pL}{2d}$ beträgt dann:
\begin{equation*}
  \frac{pL}{2d} = 0,358
\end{equation*}

Die Abweichung dieses Wertes zu $a$ beträgt 8,7 \%.

\subsection{Bestimmung der Frequenz der Sinusspannung}

Die Sägezahnfrequenzen $\nu_{Sä}$ und die Strahlauslenkung s wurden für die 4 verschiedenen n gemessen.


\begin{table}[H]
  \centering
  \caption{Empfindlichkeiten bei unterschiedlichen Beschleunigungspannungen}
  \label{tab:Spannungsamplitude}
  \begin{tabular}{c c c}
    \toprule
    $n$ & $\nu_{Sä} / \symup{\frac{1}{s}}$ &  $s/$mm\\
    \midrule
    1/2 &  40,00 & 9 \\
      1 &  79,98 & 9 \\
      2 & 159,93 & 9 \\
      3 & 239,87 & 9 \\
    \bottomrule
  \end{tabular}
\end{table}

Mit Gleichung (9) folgt daraus, dass die Frequenz der Sinusspannung näherungsweise 80Hz.
Aus Gleichung (8) folgt: $U_d = \frac{D U_B}{a}$.

\subsection{Berechnung der spezifischen Ladung aus der Ablenkung im Magnetfeld}
Tabelle \ref{tab:Magnetfeld} stellt die gemessenen Werte für die Verschiebung des Leuchtpunkts in Abhängigkeit
des Induzierten Magnetfelds bei einer Beschleunigungsspannung von $U_B = 250$ dar.

\begin{table}[H]
  \centering
  \caption{Stromstärke der Helmholtz-Spulen bei unterschiedliechen Ablenkungen.}
  \label{tab:Magnetfeld}
  \begin{tabular}{c c}
    \toprule
    $D/$ mm & $I/$A \\
    \midrule
    0 & 0 \\
    6 & 0,36 \\
    12 & 0,66 \\
    18 & 1,00 \\
    24 & 1,30 \\
    30 & 1,60 \\
    36 & 1,94 \\
    42 & 2,26 \\
    48 & 2,58 \\
    \bottomrule
  \end{tabular}
\end{table}

Die Anzahl der Windungen der Helmholtz-Spulen beträgt $N=20$ und sie haben einen Radius von $R = 28,2$ cm.
Der Abstand von Beginn der Ablenkung durch das Magnetfeld bis zum Detektorschirm beträgt $L = 17,5$ cm
Daraus ergibt sich der folgende Graph.

\begin{figure}[H]
  \centering
  \includegraphics{plot.pdf}
  \caption{Graph zur Ausgleichsrechnung für den Proportionalitätsfaktor bei 250V Beschleunigungsspannung.}
  \label{fig:plot}
\end{figure}

Die lineare Regression wird dabei mit Python erstellt.

Es ergibt sich ein Proportionalitätsfaktor von
\begin{equation*}
  a = (8,97 \pm 0,12)\cdot 10^3\, \symup{\frac{1}{mT}}.
\end{equation*}

Aus Gleichung 15 ergibt sich dann der Wert von $\frac{e_0}{m_0}$ zu
\begin{equation*}
  \frac{e_0}{m_0} = (1,61 \pm 0,04)\cdot 10^{11} \, \symup{\frac{C}{kg}}.
\end{equation*}



Tabelle \ref{tab:Magnetfeld1} stellt die gemessenen Werte für die Verschiebung des Leuchtpunkts in Abhängigkeit
des Induzierten Magnetfelds bei einer Beschleunigungsspannung von $U_B = 400$ dar.

\begin{table}[H]
  \centering
  \caption{Stromstärke der Helmholtz-Spulen bei unterschiedliechen Ablenkungen.}
  \label{tab:Magnetfeld1}
  \begin{tabular}{c c}
    \toprule
    $D/$ mm & $I/$A \\
    \midrule
    0 & 0 \\
    6 & 0,4 \\
    12 & 0,82 \\
    18 & 1,22 \\
    24 & 1,61 \\
    30 & 2,02 \\
    36 & 2,41 \\
    42 & 2,85 \\
    48 & 3,26 \\
    \bottomrule
  \end{tabular}
\end{table}

Daraus ergibt sich der folgende Graph.

\begin{figure}[H]
  \centering
  \includegraphics{plot2.pdf}
  \caption{Graph zur Ausgleichsrechnung für den Proportionalitätsfaktor bei 400V Beschleunigungsspannung.}
  \label{fig:plot222}
\end{figure}

Die lineare Regression wird wieder mit Python erstellt.

Es ergibt sich ein Proportionalitätsfaktor von
\begin{equation*}
  a = (7.06 \pm 0.10)\cdot 10^3\, \symup{\frac{1}{mT}}.
\end{equation*}

Aus Gleichung 15 ergibt sich dann der Wert von $\frac{e_0}{m_0}$ zu
\begin{equation*}
  \frac{e_0}{m_0} = (1,59 \pm 0,05)\cdot 10^{11} \, \symup{\frac{C}{kg}}.
\end{equation*}

Als Mittelwert der beiden Ergebnisse ergibt sich dann
\begin{equation*}
  \frac{e_0}{m_0} = (1,60 \pm 0,03)\cdot 10^{11} \, \symup{\frac{C}{kg}}.
\end{equation*}
Dieser wurde widerum mit Python brechnet.

Der Literaturwert lautet $\frac{e_0}{m_0} = 1,76 \cdot 10^{11} \, \symup{\frac{C}{kg}}$.
Die Relative Abweichung beträgt also etwa 10\%.

\subsection{Berechnung des Erdmagnetfelds aus der Elektronenablenkung}

Die gemessene Stromstärke zum Ausgleichen des Erdmagnetfeldes beträgt $I = 0,08$ A.
Daraus ergibt sich nach Gleichung 17 eine Magnetfeldstäre von

\begin{equation*}
  B_{hor} = 5,10 \cdot 10^{-6} \, \symup{T}
\end{equation*}

Mit dem gemessenen Inklinationswinkel $\phi = 72°$ ergibt sich das totale
Erdmagnetfeld zu
\begin{equation*}
  B_{tot} = \frac{B_{hor}}{cos(\phi)} = 1,65 \cdot 10^{-5} \, \symup{T}
\end{equation*}
