\section{Auswertung}
\label{sec:Auswertung}

\subsection{Bestimmung der Empfindlichkeit}

Die Tabellen 1 bis 5 stellen die Daten für die gemessenen Verschiebungen in Abhängigkeit von
der Ablenkspannung dar.

\begin{table}[H]
  \centering
  \caption{Gemessene Verschiebung in Abhängigkeit von der Ablenkspannung mit $U_B=200$V}
  \label{tab:Spannungsamplitude}
  \begin{tabular}{c c}
    \toprule
    $U_d/$V & $D/$mm \\
    \midrule
    -19,72 & 0 \\
    -16,00 & 6 \\
    -12,50 & 12 \\
     -9,14 & 18 \\
     -5,72 & 24 \\
     -2,02 & 30 \\
      1,69 & 36 \\
      5,41 & 42 \\
      9,12 & 48 \\
    \bottomrule
  \end{tabular}
\end{table}

\begin{table}[H]
  \centering
  \caption{Gemessene Verschiebung in Abhängigkeit von der Ablenkspannung mit $U_B=250$V}
  \label{tab:Spannungsamplitude}
  \begin{tabular}{c c}
    \toprule
    $U_d/$V & $D/$mm \\
    \midrule
     -24,6 & 0 \\
     -20,3 & 6 \\
     -16,3 & 12 \\
     -11,8 & 18 \\
      -7,5 & 24 \\
      -2,9 & 30 \\
       1,8 & 36 \\
       6,8 & 42 \\
      11,8 & 48 \\
    \bottomrule
  \end{tabular}
\end{table}



\begin{table}[H]
  \centering
  \caption{Gemessene Verschiebung in Abhängigkeit von der Ablenkspannung mit $U_B=300$V}
  \label{tab:Spannungsamplitude}
  \begin{tabular}{c c}
    \toprule
    $U_d/$V & $D/$mm \\
    \midrule
     -29,5 & 0 \\
     -24,6 & 6 \\
     -19,1 & 12 \\
     -14,2 & 18 \\
      -9,2 & 24 \\
      -3,9 & 30 \\
       1,6 & 36 \\
       7,4 & 42 \\
      13,3 & 48 \\
    \bottomrule
  \end{tabular}
\end{table}


\begin{table}[H]
  \centering
  \caption{Gemessene Verschiebung in Abhängigkeit von der Ablenkspannung mit $U_B=350$V}
  \label{tab:Spannungsamplitude}
  \begin{tabular}{c c}
    \toprule
    $U_d/$V & $D/$mm \\
    \midrule
     -34,1 & 0 \\
     -28,2 & 6 \\
     -22,5 & 12 \\
     -16,2 & 18 \\
     -10,1 & 24 \\
      -4,1 & 30 \\
       2,0 & 36 \\
       8,7 & 42 \\
      15,2 & 48 \\
    \bottomrule
  \end{tabular}
\end{table}


\begin{table}[H]
  \centering
  \caption{Gemessene Verschiebung in Abhängigkeit von der Ablenkspannung mit $U_B=400$V}
  \label{tab:Spannungsamplitude}
  \begin{tabular}{c c}
    \toprule
    $U_d/$V & $D/$mm \\
    \midrule
     -32,2 & 6 \\
     -25,2 & 12 \\
     -18,2 & 18 \\
     -11,4 & 24 \\
      -4,2 & 30 \\
       2,9 & 36 \\
      10,5 & 42 \\
      17,8 & 48 \\
    \bottomrule
  \end{tabular}
\end{table}


Mit den Messwerten wird eine lineare Regression durchgefürht, welche in Abbildung 6 zusehen ist.

\begin{figure}
  \centering
  \includegraphics{plot11.pdf}
  \caption{Empfindlichkeiten bei verschiedenen Beschleunigungsspannungen}
  \label{fig:plot}
\end{figure}

Mit Python wird die Steigung der Geraden berechnet. \\

\begin{table}[H]
  \centering
  \caption{Empfindlichkeiten bei unterschiedlichen Beschleunigungspannungen}
  \label{tab:Spannungsamplitude}
  \begin{tabular}{c c}
    \toprule
    $U_B/$V & $\frac{D}{U_d} / 10^{-3}\symup{\frac{m}{V}}$ \\
    \midrule
    200 & $\SI{1.67(1)}{}$ \\
    250 & $\SI{1.32(2)}{}$ \\
    300 & $\SI{1.13(1)}{}$ \\
    350 & $\SI{0.98(1)}{}$ \\
    400 & $\SI{0.84(1)}{}$ \\
    \bottomrule
  \end{tabular}
\end{table}



\subsection{Magnetfeld}

\begin{figure}
  \centering
  \includegraphics{plot.pdf}
  \caption{Plot.}
  \label{fig:plot}
\end{figure}
