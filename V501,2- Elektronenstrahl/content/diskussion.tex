\section{Diskussion}
\label{sec:Diskussion}

Die Messwert der Verschiebungen in Abhängigkeit von der Ablenkspannung liegen näherungsweise auf den Anpssungsfunktion.
Der Wert der Größe $\frac{pL}{2d}$ liegt außerhalb der Standardabweichung von $a$. Statistische Fehler
werden primäre Fehlerquelle augeschlossen, da es keine merkbaren Ausreißer gibt. Die genaue Abmessung
des Koordinatennetzes ist nicht möglich, weshalb Ungenauigkeiten bei den Werten der Verschiebung auftreten können.
Zusätzlich kann die Verschiebung des Leuchtpunktes wegen seiner Ausdehnung nicht optimal bestimmt werden.

Die Frequenz der Sinusspannung kann näherungsweise bestimmt werden. Da jedoch nicht eindeutig
festgelegt werden kann, wann ein stehendes Bild zu sehen ist, ist eine genaue Bestimmung nicht möglich.
Außerdem schwankt die Frequenz die der Frequenzzähler anzeigt stark, was ebenfalls zu Ungenauigkeiten führt.

Bei der Messung der Elektronenablenkung im Magnetfeld liegen die Messwerte ebenfalls näherungsweise auf der Anpassungsfunktion.
Der erwartete lineare Zusammenhang ist also gut zu erkennen. Dennoch ergibt sich letztendlich eine Abweichung vom etwa 10\% zum
Literaturwert. Dieser liegt zudem nicht innerhalb der Standardabweichung. Auch hier werden statistische als Fehlerquelle ausgeschlossen.
Der Grund wird viel mehr in systematischen Fehlern angenommen. Mit dem Deklinatorium-Inklinatorium konnte die Richtung des Erdmagentfelds
nicht einwandfrei bestimmt werden, da sich dieses als nicht sehr genau erwies. Kurze Zeit nach der optimalen Ausrichtung der Apparatur
zeigte die Nadel wieder in eine leicht andere Richtung als sie es zuvor tat. Der zusätliche Einfluss des Erdmagnetfelds würde nicht den
ermittelten linearen Zusammenhang stören, die Messwerte ingesamt aber um einen gewissen Betrag verschieben. Diese Annahme trifft also
tatsächlich auf das bestehende Fehlerbild zu. Dennoch kann auch die idealisert angenommene Flugbahn der Elektronen und die damit
verbundene Rechnung eine Fehlerquelles sein. Vor allem durch die vorherige Ablenkung der Elektronen durch das elektrische Feld fliegen
diese nicht zwingend senkrecht auf den Detektorschirm zu.
