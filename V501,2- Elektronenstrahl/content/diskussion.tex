\section{Diskussion}
\label{sec:Diskussion}

Die Messwert der Verschiebungen in Abhängigkeit von der Ablenkspannung liegen näherungsweise auf den Anpssungsfunktion.
Der Wert der Größe $\frac{pL}{2d}$ liegt außerhalb der Standardabweichung von $a$. Statistische Fehler
werden primäre Fehlerquelle augeschlossen, da es keine merkbaren Ausreißer gibt. Die genaue Abmessung
des Koordinatennetzes ist nicht möglich, weshalb Ungenauigkeiten bei den Werten der Verschiebung auftreten können.
Zusätzlich kann die Verschiebung des Leuchtpunktes wegen seiner Ausdehnung nicht optimal bestimmt werden.

Die Frequenz der Sinusspannung kann näherungsweise bestimmt werden. Da jedoch nicht eindeutig
festgelegt werden kann, wann ein stehendes Bild zu sehen ist, ist eine genaue Bestimmung nicht möglich.
Außerdem schwankt die Frequenz die der Frequenzzähler anzeigt stark, was ebenfalls zu Ungenauigkeiten führt.
