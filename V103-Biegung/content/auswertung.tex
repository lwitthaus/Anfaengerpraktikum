\section{Auswertung}
\label{sec:Auswertung}
\subsection{Fehlerrechnung}
Der Mittelwert eines Datensatzes mit $N$ Werten ist definiert durch:
\begin{equation}
  \bar{x} = \frac{1}{N} \sum_{i=1}^N x_i
\end{equation}
Die Standardabweichung eines Datensatzes von seinem Mittelwert durch:
\begin{equation}
  \sigma = \sqrt{\frac{1}{N(N-1)} \sum_{i=1}^N (x_i - \bar{x})}
\end{equation}
Pflanzen sich Unsicherheiten fort, wird der Fehler mit der gaußschen
Fehlerfortpflanzung berechnet:
\begin{equation}
  \sigma_f = \sqrt{
      \sum\limits_{i = 1}^N
       \left( \frac{\partial f}{\partial x_i} \sigma_i \right)^{\!\! 2}
     }
\end{equation}
Der Fehler der Steigung $m$ und des Achsenabschnittes $b$ einer linearen Regression
wird wie folgt berechnet:
\begin{align}
  \sigma_m &= \frac{\overline{xy}-\bar{x}\cdot\bar{y}}{\bar{x^2}-\bar{x}^2} \\
  \sigma_b &= \frac{\bar{x^2}\bar{y}-\overline{xy}\bar{x}}{\bar{x^2}-\bar{x}^2}
\end{align}

\subsection{Einseitig eingespannter, runder Stab}
Die gemessenen Durchbiegungen $D_R$ und deren Orte werden in Tabelle 1 dargestellt. Zudem wird
das Polynom angegeben, welches für die lineare Regression benötigt wird.

\begin{table}[H]
  \centering
  \caption{Gemessene Kräfte und Auslenkungen}
  \label{tab:Parameter}
  \begin{tabular}{c c c}
    \toprule
    $x/\left(10^{-3}\symup{m}\right)$ & $D_R(x)/\left(10^{-3}\symup{m}\right)$ & $\left(Lx^2 -\frac{x^3}{3}\right)/\left(10^{-3}\symup{m^3}\right)$\\
    \midrule
     40 & 0.1 &  0.8 \\
     60 & 0.2 &  1.7 \\
     80 & 0.4 &  3.0 \\
    100 & 0.7 &  4.7 \\
    120 & 0.8 &  6.6 \\
    140 & 0.9 &  8.9 \\
    160 & 1.0 & 11.4 \\
    180 & 1.0 & 14.3 \\
    200 & 1.4 & 17.3 \\
    220 & 1.5 & 20.7 \\
    240 & 1.6 & 24.2 \\
    260 & 1.7 & 27.9 \\
    280 & 2.0 & 31.9 \\
    300 & 2.1 & 36.0 \\
    320 & 2.4 & 40.3 \\
    340 & 2.9 & 44.7 \\
    360 & 3.0 & 49.2 \\
    380 & 3.2 & 53.9 \\
    400 & 3.8 & 58.7 \\
    420 & 3.9 & 63.5 \\
    440 & 4.3 & 68.4 \\
    460 & 4.8 & 73.4 \\
    480 & 4.8 & 78.3 \\
    \bottomrule
  \end{tabular}
\end{table}

Die Länge des eingespannten Stabes beträgt $L = 0.5 \symup{m}$. Die Masse des
Gewichtes beträgt $m = 533.8$g.
Die Durchbiegung wird gegen das Polynom $\left(Lx^2 -\frac{x^3}{3}\right)$ aufgetragen. Eine
lineare Regression wird durchgeführt.

\begin{figure}[H]
  \centering
  \includegraphics{build/plot.pdf}
  \caption{Ausgleichsgerade zur Bestimmung des Elastizitätsmodul}
  \label{fig:Elastizitätsmodul}
\end{figure}

Die Gerade wird durch die Gleichung $y = ax$ beschrieben. Der Parameter $a$ beträgt:
\begin{align*}
  a &= (0.059 \pm 0.001) \symup{\frac{1}{m^2}}\\
\end{align*}

Mit der Steigung $a$ und Gleichung (10) folgt für den Elastizitätsmodul:
\begin{equation}
  E = \frac{mg}{2I \cdot a}
\end{equation}

Das Flächenträgheitsmoment eines zylinderförmigen Stabes mit dem Radius
$r = 0.5$cm beträgt:
\begin{equation}
  I_{Kreis} = \frac{\pi r^4}{4} = 4.909 \cdot 10^{-10} \symup{m^4} .
\end{equation}
Für den Elastizitätsmodul folgt:
\begin{align*}
  E = \SI{9.0(2)e10}{\kilo\gram\per\second\squared\per\meter}
\end{align*}
Der Fehler von $E$ wird mit Gleichung (20) berechnet.

Die Länge des Stabes beträgt $L = 0.55$m und die Masse des Stabes beträgt $m = 0.3604$kg.
Daraus folgt für die Dichte:
\begin{equation}
  \rho = \frac{m}{V} = 8343 \symup{\frac{kg}{m^3}}
\end{equation}

Dieser Wert stimmt mit dem Literaturwert von Messing, $\rho =8100\symup{\frac{kg}{m^3}}\: \text{bis}\:\: 8700\symup{\frac{kg}{m^3}}$ \cite{sample2},
über ein.
Der Literaturwert des Elastizitätsmodul von Messing beträgt \cite{sample2}:
\begin{align*}
  E = \SI{7.8e10}{\kilo\gram\per\second\squared\square\meter} \:\: \text{bis} \:\:
  \SI{12.3e10}{\kilo\gram\per\second\squared\square\meter}.
\end{align*}

Der aus dem Versuch berechnete Elastizitätsmodul liegt in dem erwarteten Bereich des Theoriewertes.

\subsection{Einseitig eingespannter, quadratischer Stab}
Die gemessenen Durchbiegungen $D_Q$ und deren Orte werden in Tabelle 2 dargestellt. Zudem wird
das Polynom angegeben, welches für die lineare Regression benötigt wird.

\begin{table}[H]
  \centering
  \caption{Gemessene Kräfte und Auslenkungen des quadratischen Stabes}
  \label{tab:quadratischer Stab}
  \begin{tabular}{c c c}
    \toprule
    $x/\left(10^{-3}\symup{m}\right)$ & $D_Q(x)/\left(10^{-3}\symup{m}\right)$ & $\left(Lx^2 -\frac{x^3}{3}\right)/\left(10^{-3}\symup{m^3}\right)$\\
    \midrule
     40 & 0.1 &  0.8 \\
     60 & 0.3 &  1.7 \\
     80 & 0.4 &  3.0 \\
    100 & 0.4 &  4.7 \\
    120 & 0.5 &  6.6 \\
    140 & 0.7 &  8.9 \\
    160 & 0.8 & 11.4 \\
    180 & 0.8 & 14.3 \\
    200 & 0.9 & 17.3 \\
    220 & 1.0 & 20.7 \\
    240 & 1.2 & 24.2 \\
    260 & 1.4 & 27.9 \\
    280 & 1.7 & 31.9 \\
    300 & 1.9 & 36.0 \\
    320 & 2.0 & 40.3 \\
    340 & 2.1 & 44.7 \\
    360 & 2.5 & 49.2 \\
    380 & 2.8 & 53.9 \\
    400 & 2.9 & 58.7 \\
    420 & 3.0 & 63.5 \\
    440 & 3.2 & 68.4 \\
    460 & 3.6 & 73.4 \\
    480 & 3.8 & 78.3 \\
    \bottomrule
  \end{tabular}
\end{table}

Die Länge des eingespannten Stabes beträgt $L = 0.5 \symup{m}$. Die Masse des
Gewichtes beträgt $m = 533.8$g.
Die Durchbiegung wird gegen das Polynom $\left(Lx^2 -\frac{x^3}{3}\right)$ aufgetragen. Eine
lineare Regression wird durchgeführt.

\begin{figure}[H]
  \centering
  \includegraphics{build/plot2.pdf}
  \caption{Ausgleichsgerade zur Bestimmung des Elastizitätsmodul des quadratischen Stabes}
  \label{fig:Elastizitätsmodul des quadratischen Stabes}
\end{figure}

Die Gerade wird durch die Gleichung $y = cx$ beschrieben. Der Parameter $c$ beträgt:
\begin{align*}
  c &= (0.0458 \pm 0.0007) \symup{\frac{1}{m}} \\
\end{align*}


Mit der Steigung $c$ und Gleichung (10) folgt für den Elastizitätsmodul:
\begin{equation}
  E = \SI{6.9(1)e10}{\kilo\gram\per\second\squared\per\meter}
\end{equation}

Der Fehler von $E$ wird mit Gleichung (20) berechnet.

Das Flächenträgheitsmoment $I_Q$ eines quaderförmigen Stabes mit der Kantenlänge
$k = 1$cm beträgt:
\begin{equation}
  I_Q = \frac{k^4}{12} = 8.333 \cdot 10^{-10} \symup{m^4}
\end{equation}


Die Länge des Stabes beträgt $L = 0.59$m und die Masse des Stabes beträgt $m = 0.1641$kg.
Daraus folgt für die Dichte:
\begin{equation}
  \rho = \frac{m}{V} = 2781 \symup{\frac{kg}{m^3}}
\end{equation}
Der Literaturwert der Dichte von Aluminium mit $\rho = 2700 \symup{\frac{kg}{m^3}}$ \cite{sample4},
stimmt mit dem berechneten Wert am besten überein. Der Literaturwert des Elastizitätsmodul
von Aluminium ist $E = 7.0 \cdot 10^{9} \symup{\frac{kg}{s^2m}}$ \cite{sample3} und stimmt mit dem berechnetem Wert
überein. Die relative Abweichung beträgt $1.4 \%$.

\subsection{Beidseitig gelagerter Stab}
Die gemessenen Durchbiegungen $D_{B}$ und deren Orte werden in in Tabelle 3 dargestellt. Zudem wird
das Polynom angegeben, welches für die lineare Regression benötigt wird.

\begin{table}[H]
  \centering
  \caption{Gemessene Durchbiegungen und Auslenkungen des beidseitig eingespannten Stabes}
  \label{tab:beidseitig}
  \begin{tabular}{c c c}
    \toprule
    $x/\left(10^{-3}\symup{m}\right)$ & $D_{B}(x)/\left(10^{-3}\symup{m}\right)$ & $\left(3L^2x-4x^3\right)/\symup{\left(10^{-3}m^3\right)}$ \\
    \midrule
     40 & 0.4 &  36.6 \\
     60 & 0.6 &  54.4 \\
     80 & 0.9 &  71.6 \\
    100 & 1.1 &  88.1 \\
    120 & 1.4 & 103.6 \\
    140 & 1.8 & 117.9 \\
    160 & 2.1 & 130.9 \\
    180 & 2.2 & 142.4 \\
    200 & 2.3 & 152.1 \\
    220 & 2.4 & 160.0 \\
    240 & 2.4 & 165.8 \\
    260 & 2.8 & 169.1 \\
    280 & 2.8 & 170.0 \\
    \bottomrule
  \end{tabular}
\end{table}

Die Länge des eingespannten Stabes beträgt $L = 0.554 \symup{m}$. Die Masse des
Gewichtes beträgt $m = 4.6995$kg.
Die Durchbiegung wird gegen das Polynom $\left(3L^2x-4x^3\right)/\symup{\left(10^{-3}m^3\right)}$ aufgetragen. Eine
lineare Regression wird durchgeführt.

\begin{figure}[H]
  \centering
  \includegraphics{build/Beidseitig1.pdf}
  \caption{Ausgleichsgerade zur Bestimmung des Elastizitätsmodul des runden Stabes (x < L/2)}
  \label{fig:Elastizitätsmodul des runden Stabes}
\end{figure}

Die Gerade wird durch die Gleichung $y = ex + f$ beschrieben. Die Parameter betragen:
\begin{align*}
  e &= (0.0178 \pm 0.0007) \: \frac{1}{m^2} \\
  f &= (0.18  \pm 0.0001) \: m
\end{align*}

Mit der Steigung $e$ und Gleichung (16) folgt für den Elastizitätsmodul:
\begin{equation}
  E = \frac{mg}{48I \cdot e}
\end{equation}

Dabei entspricht das Flächenträgheitsmoment I aufgrund des gleichen Radius der Stäbe
dem aus Gleichung (24).

Mit Gleichung (28) folgt für den Elastizitätsmodul:
\begin{equation}
  E = \SI{1.10(4)e11}{\kilo\gram\per\second\squared\per\meter}
\end{equation}

Der Fehler von $E$ wird mit Gleichung (20) berechnet.

Die Länge des Stabes beträgt $L = 0.60$m und die Masse des Stabes beträgt $m = 0.3943$kg.
Daraus folgt für die Dichte:
\begin{equation}
  \rho = \frac{m}{V} = 8367 \symup{\frac{kg}{m^3}} .
\end{equation}

Dies entspricht der Dichte von Messing. Das errechnete Elastizitätsmodul liegt in dem
erwarteten Bereich des Theoriewertes für Messing.

Die Messwerte für die andere Seite des Stabes werden in Tabelle 4 angegeben.

\begin{table}[H]
  \centering
  \caption{Gemessene Durchbiegung und Auslenkungen des beidseitig eingespannten Stabes}
  \label{tab:beidseitig}
  \begin{tabular}{c c c}
    \toprule
    $x/\left(10^{-3}\symup{m}\right)$ & $D_{B2}(x)/\left(10^{-3}\symup{m}\right)$ & $\left(4x^3-12Lx^2 + 9L^2x-L^3\right)/\symup{(10^{-3}m^3)}$ \\
    \midrule
    320 & 3.0 & 164.2 \\
    340 & 2.9 & 157.8 \\
    360 & 2.6 & 149.4 \\
    380 & 2.3 & 139.1 \\
    400 & 2.5 & 127.2 \\
    420 & 2.2 & 113.8 \\
    440 & 1.9 & 99.0 \\
    460 & 1.6 & 83.2 \\
    480 & 1.0 & 66.5 \\
    500 & 0.9 & 49.1 \\
    520 & 0.6 & 31.1 \\
    540 & 0.4 & 12.9 \\
    \bottomrule
  \end{tabular}
\end{table}

Eine lineare Regression wird für $x>L/2$ durchgeführt.

\begin{figure}[H]
  \centering
  \includegraphics{build/Beidseitig2.pdf}
  \caption{Ausgleichsgerade zur Bestimmung des Elastizitätsmodul des runden Stabes (x > L/2)}
  \label{fig:Elastizitätsmodul des runden Stabes1}
\end{figure}

Die Gerade wird durch die Gleichung $y = hx + i$ beschrieben. Die Parameter betragen:
\begin{align*}
  h &= (0.0176 \pm 0.0008) \: \frac{1}{m^2} \\
  i &= (0.0001  \pm 0.0001) \: m
\end{align*}

Mit der Steigung $h$ und Gleichung (17) folgt für den Elastizitätsmodul:
\begin{equation}
  E = \frac{mg}{48I \cdot h}
\end{equation}


Mit Gleichung (31) folgt für den Elastizitätsmodul:
\begin{equation}
  E = \SI{1.11(5)e11}{\kilo\gram\per\second\squared\per\meter}
\end{equation}

Der Fehler von $E$ wird mit Gleichung (20) berechnet.
Der berechnete Wert liegt in dem erwarteten Bereich des Theoriewertes von Messing.
