\section{Auswertung}
\label{sec:Auswertung}
Die gemessenen Durchbiegungen und deren Orte werden in in Tabelle 1 dargestellt. Zudem wird
das Polynom angegeben, welches für die lineare Regression benötigt wird.

\begin{table}[H]
  \centering
  \caption{Gemessene Kräfte und Auslenkungen}
  \label{tab:Parameter}
  \begin{tabular}{c c c}
    \toprule
    $x/\left(10^{-3}\symup{m}\right)$ & $D(x)/\left(10^{-3}\symup{m}\right)$ & $\left(Lx^2 -\frac{x^3}{3}\right)/\left(10^{-3}\symup{m}\right)$\\
    \midrule
     40 & 0.1 &  0.8 \\
     60 & 0.2 &  1.7 \\
     80 & 0.4 &  3.0 \\
    100 & 0.7 &  4.7 \\
    120 & 0.8 &  6.6 \\
    140 & 0.9 &  8.9 \\
    160 & 1.0 & 11.4 \\
    180 & 1.0 & 14.3 \\
    200 & 1.4 & 17.3 \\
    220 & 1.5 & 20.7 \\
    240 & 1.6 & 24.2 \\
    260 & 1.7 & 27.9 \\
    280 & 2.0 & 31.9 \\
    300 & 2.1 & 36.0 \\
    320 & 2.4 & 40.3 \\
    340 & 2.9 & 44.7 \\
    360 & 3.0 & 49.2 \\
    380 & 3.2 & 53.9 \\
    400 & 3.8 & 58.7 \\
    420 & 3.9 & 63.5 \\
    440 & 4.3 & 68.4 \\
    460 & 4.8 & 73.4 \\
    480 & 4.8 & 78.3 \\
    \bottomrule
  \end{tabular}
\end{table}

Die Länge des eingespannten Stabes beträgt $L = 0.5 \symup{m}$. Die Masse des
Gewichtes beträgt $m = 533.8$g.
Die Durchbiegung wird gegen das Polynom $\left(Lx^2 -\frac{x^3}{3}\right)$ aufgetragen. Eine
lineare Regression wird durchgeführt.

\begin{figure}[H]
  \centering
  \includegraphics{build/plot.pdf}
  \caption{Ausgleichsgerade zur Bestimmung des Elastizitätsmodul}
  \label{fig:Elastizitätsmodul}
\end{figure}

Die Gerade wird durch die Gleichung $y = ax + b$ beschrieben. Die Parameter betragen:
\begin{align*}
  a &= 0.059 \pm 0.001 \\
  b &= 0.22  \pm 0.05
\end{align*}

Mit der Steigung $a$ und Gleichung (10) folgt für das Elastizitätsmodultizität:
\begin{equation}
  E = \frac{mg}{2I \cdot a}
\end{equation}

Das Flächenträgheitsmoment eines zylinderförmigen Stabes mit dem Radius
$r = $ beträgt:
\begin{equation}
  I_{Kreis} = \frac{\pi r^4}{4} =
\end{equation}
