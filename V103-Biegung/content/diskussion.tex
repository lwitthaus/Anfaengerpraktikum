\section{Diskussion}
\label{sec:Diskussion}
Die berechneten Elastitzitätsmodule stimmen mit den Theoriewerten überein. Aus der Dichte der Stäbe und deren
Elastizitätsmodule ist eine Bestimmung des Materials möglich. Das Elastizitätsmodul von Messing ist kein
genauer Wert, sondern ein Intervall, in dem die berechneten Werte des einseitig und beidseitig eingespannten Messingstabes
mittig liegen. Aluminium hat einen genauen Wert für den Elastizitätsmodul. Die Abweichung von $1.4\%$ ist innerhalb der Standardabweichung.
Die Messuhren zum Messen der Durchbiegung
zeigen ungenaue Werte an, wodurch es zu systematischen Fehlern kommt. Die Längen der Stäbe werden mit
einem Maßband gemessen, wodurch ebenfalls ungenaue Längen für das Berechnen der Dichte folgen. Auch
ohne Einwirkung einer Gewichtskraft der Massenstücke weisen die Stäbe eine Krümmung auf. Dies
verfälscht den Wert der Durchbiegung in der Ruhelage der Stäbe. Die Formel für das Berechnen der
Durchbiegung wird für größere Krümmungen ungenauer, da zur Herleitung dieser Formel der
Krümmungsradius als klein genähert wird. Es werden über 20 Messwerte pro Stab gemessen, wodurch
statistische Fehler als primäre Fehlerquelle ausgeschlossen werden können.
Trotz mehrerer Fehlerquellen ist der Versuch geeignet Elastizitätsmodule der Stäbe
bestimmen und so auf das Material schließen zu können.
