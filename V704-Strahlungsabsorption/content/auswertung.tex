\section{Auswertung}
\label{sec:Auswertung}

\subsection{Gamma-Strahlung}
\subsubsection{Bestimmung des Absorptionskoeffizienten eines Zinkabsorbers}
Ohne einen Absorber zu verwenden hat das Zählrohr in $100$ Sekunden $994$ Ereignisse gemessen.

In Tabelle 1 werden die Anzahl an Ereignissen $N$ und die Aktivität $A = N/t$ in Abhängigkeit von der Dicke $D$ des Absorbers und der
Dauer der Messung $t$ angeben. Der Fehler $\Delta N$ von $N$ ist dabei $\sqrt{N}$, da es sich bei dem vorliegendem Problem um eine
Poissonverteilung handelt. Für die Aktivität wird ebenfalls der Fehler angegeben.

\begin{table}[H]
  \centering
  \caption{Zählrate und Aktivität in Abhängigkeit von der Zeit und der Dicke des Zinkabsorbers}
  \label{tab:Rechteckspannung}
  \begin{tabular}{c c c c}
    \toprule
    Zeit$/$s & $D/$mm & $N \pm \Delta N$ & $(A \pm \Delta A)/$s\\
    \midrule
    80 & 2 & $\num{15280(130)}$  &  $\num{149.4(17)}$ \\
    80 & 4 & $\num{11950(110)}$ &   $\num{132.0(14)}$ \\
    90 & 6 & $\num{10236(110)}$ &   $\num{113.7(13)}$ \\
    100 & 8 & $\num{10944(110)}$ &  $\num{109.4(11)}$ \\
    120 & 10 & $\num{11544(110)}$ & $\num{96.2(10)}$ \\
    140 & 12 & $\num{12317(120)}$ & $\num{88.0(9)}$ \\
    150 & 14 & $\num{11972(110)}$ & $\num{79.8(8)}$ \\
    160 & 16 & $\num{11734(110)}$ & $\num{73.3(7)}$ \\
    170 & 18 & $\num{11780(110)}$ & $\num{69.3(7)}$ \\
    180 & 20 & $\num{11035(110)}$ & $\num{61.3(7)}$ \\
    \bottomrule
  \end{tabular}
\end{table}

Es wird eine lineare Regression durchgeführt. Die Anpassungsfunktion wird durch die Gerade $y = ax + b$  dargestellt und die Parameter $a$ und $b$
mit Python berechnet.

\begin{figure}[H]
  \centering
  \includegraphics{plot.pdf}
  \caption{Lineare Regression der Aktivität aufgetragen gegen die Dicke des Zinkabsorbers}
  \label{fig:plot}
\end{figure}

Dabei gilt $A_0 =$ 1/s.

Der Absorptionskoeffizient $\mu$ entspricht dann der negativen Steigung der Geraden.
\begin{align*}
  -a = \mu_{Zn} = \SI{0.0478(15)}{\per\milli\meter} = \SI{48(2)}{\per\meter}
\end{align*}

Die Aktivität bei einer Schichtdicke von Null beträgt:
\begin{align*}
  A_{Zn} = \exp(b) \cdot \frac{1}{\symup{s}} = \SI{158(3)}{\per\second} \\
  \text{mit} \: b = \SI{5.06(2)}{}
\end{align*}

\subsubsection{Bestimmung des Absorptionskoeffizienten eines Bleiabsorber}

In Tabelle 2 werden die gemessenen Daten für einen Bleiabsorber dargestellt.

\begin{table}[H]
  \centering
  \caption{Zählrate und Aktivität in Abhängigkeit von der Zeit und der Dicke des Bleiabsorbers}
  \label{tab:Rechteckspannung}
  \begin{tabular}{c c c c}
    \toprule
    Zeit$/$s & $D/$mm & $N \pm \Delta N$ & $(A \pm \Delta A)/$s\\
    \midrule
    60 & 1 & $\num{7895(90)}$  &  $\num{131.6(15)}$ \\
    60 & 2 & $\num{7020(90)}$ &   $\num{117.0(15)}$ \\
    70 & 3 & $\num{7418(90)}$ &   $\num{106.0(13)}$ \\
    90 & 4 & $\num{8242(100)}$ &  $\num{91.6(11)}$ \\
    100 & 5 & $\num{8322(100)}$ & $\num{83.2(10)}$ \\
    130 & 10 & $\num{6843(90)}$ & $\num{52.6(7)}$ \\
    150 & 12 & $\num{6026(80)}$ & $\num{40.2(6)}$ \\
    170 & 15 & $\num{4932(80)}$ & $\num{29.0(5)}$ \\
    220 & 20 & $\num{4166(70)}$ & $\num{18.9(3)}$ \\
    350 & 30 & $\num{2569(60)}$ & $\num{7.3(2)}$ \\
    500 & 40 & $\num{1635(50)}$ & $\num{3.3(1)}$ \\
    700 & 50 & $\num{1176(40)}$ & $\num{1.7(1)}$ \\
    \bottomrule
  \end{tabular}
\end{table}

Es wird erneut eine lineare Regression durchgeführt. Die Gerade wird duch $mx + c$ beschrieben.

\begin{figure}[H]
  \centering
  \includegraphics{plotblei.pdf}
  \caption{Lineare Regression der Aktivität aufgetragen gegen die Dicke des Bleiabsorbers}
  \label{fig:plot}
\end{figure}


Für den Absorptionskoeffizienten des Bleiabsorbers ergibt sich dann:
\begin{align*}
  -m = \mu_{Pb} = \SI{0.091(2)}{\per\milli\meter} = \SI{91(2)}{\per\meter}
\end{align*}

Die Aktivität bei einer Schichtdicke von Null beträgt:
\begin{align*}
  A_{Pb} = \exp(c) \cdot \frac{1}{\symup{s}} = \SI{129(6)}{\per\second} \\
  \text{mit} \: c = \SI{4.86(5)}{}
\end{align*}

\subsubsection{Absorptionskoeffizienten des Compton-Effekts}

Der Absorptionskoeffizient des Compton-Effekts $\mu_{com}$ wird mit Gleichung (4) und (5) bestimmt. Für die jeweiligen Absorber
folgt mit $\epsilon = 1,2954$:
\begin{align*}
\mu_{com_{Zn}} &= \SI{50,635}{\per\meter}  \\
\mu_{com_{Pb}} &= \SI{69,373}{\per\meter}
\end{align*}

Die Abweichung der experimentell bestimmten Absorptionskoeffizienten von $\mu_{com}$ beträgt:
\begin{align*}
  \frac{\mu_{Zn}- \mu_{com_{Zn}}}{\mu_{com_{Zn}}} = -5,2 \% \\
  \frac{\mu_{Pb}- \mu_{com_{Pb}}}{\mu_{com_{Pb}}} = 31,2 \%
\end{align*}


\subsection{Beta-Strahlung}

Ohne einen Absorber zu verwenden hat das Zählrohr in $900$ Sekunden $994$ Wechselwirkungen gemessen.

In Tabelle 2 werden wieder die Anzahl an Ereignissen $N$ und die Aktivität $A = N/t$ in Abhängigkeit von der Dicke $D$ des Absorbers und der
Dauer der Messung $t$ angegeben.

\begin{table}[H]
  \centering
  \caption{Zählrate und Aktivität in Abhängigkeit von der Zeit und der Dicke des Absorbers bei Beta-Strahlung}
  \label{tab:}
  \begin{tabular}{c c c c}
    \toprule
    Zeit$/$s & $D/\symup{\mu m}$ & $N \pm \Delta N$ & $(A \pm \Delta A)/$s\\
    \midrule
    100 & 100 & $\num{3873(70)}$  &  $\num{38.7(7)}$ \\
    200 & 125 & $\num{2021(50)}$ &   $\num{10.1(3)}$ \\
    200 & 153 & $\num{1978(50)}$ &   $\num{9.9(3)}$ \\
    300 & 160 & $\num{1685(50)}$ &  $\num{5.6(2)}$ \\
    400 & 200 & $\num{880(30)}$ & $\num{2.2(1)}$ \\
    700 & 253 & $\num{572(30)}$ & $\num{0.8(1)}$ \\
    700 & 302 & $\num{489(30)}$ & $\num{0.7(1)}$ \\
    700 & 338 & $\num{478(30)}$ & $\num{0.7(1)}$ \\
    700 & 400 & $\num{438(30)}$ & $\num{0.6(1)}$ \\
    700 & 444 & $\num{481(30)}$ & $\num{0.7(1)}$ \\
    700 & 482 & $\num{494(30)}$ & $\num{0.7(1)}$ \\
    \bottomrule
  \end{tabular}
\end{table}

Daraus ergibt sich das folgende Diagramm.

\begin{figure}[H]
  \centering
  \includegraphics{plot1.pdf}
  \caption{Lineare Regression der Aktivität aufgetragen gegen die Dicke.}
  \label{fig:plot1}
\end{figure}

Dabei ist R die Massenbelegung des Aluminiums und ergibt sich aus Gleichung 6.
Die Dichte von Aluminium beträgt 2,7 / $\symup{\frac{g}{cm^3}}$ \cite{sample1}.

Es werden zwei lineare Regressionen durchgeführt. Eine für den abfallenden Teil der Kurve (ersten fünf Messpunkte)
und eine für den nahezu konstant verlaufenden Teil der Kurve (letzten 6 Messpunkte):
\begin{equation*}
  y = a\cdot x + b
\end{equation*}
Es ergibt sich für die Parameter der  Anpassungsfunktionen:
\begin{align*}
  a_1 &= (-9,82 \pm 1,66) \: \symup{\frac{m^3}{kg\cdot s}} \\
  a_2 &= (-0,20 \pm 0,17) \: \symup{\frac{m^3}{kg\cdot s}} \\
  b_1 &= (6,07 \pm 0,68) \\
  b_2 &= (-0,16 \pm 0,17)
\end{align*}

Die x-Koordinate des Schnittpunkts der beiden Ausgleichsgeraden ist der Wert $R_{max}$ und ergibt
sich folglich aus
\begin{equation*}
  R_{max} = \frac{b_2 - b_1}{a_1 - a_2}
\end{equation*}
Daraus ergibt sich ein Wert von
\begin{equation*}
  R_{max} = (0,65 \pm 0,13) \: \symup{\frac{kg}{m^3}}
\end{equation*}
Aus Gleichung 7 folgt dann für die maximale Energie $E_{max}$
\begin{equation*}
  E_{max} = (1,44 \pm 0,26) \: \symup{J}
\end{equation*}
