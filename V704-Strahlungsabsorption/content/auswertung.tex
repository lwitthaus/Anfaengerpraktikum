\section{Auswertung}
\label{sec:Auswertung}

\subsection{Gamma-Strahlung}
\subsubsection{Bestimmung des Absorptionskoeffizienten eines Zinkabsorbers}
Ohne einen Absorber zu verwenden hat das Zählrohr in $100$ Sekunden $994$ Wechselwirkungen gemessen.

In Tabelle 1 werden die Anzahl an Wechselwirkungen $N$ und die Aktivität $A = N/t$ in Abhängigkeit von der Dicke $D$ des Absorbers und der
Dauer der Messung $t$ angeben. Der Fehler $\Delta N$ von $N$ ist dabei $\sqrt{N}$, da es sich bei dem Vorliegendem Problem um eine
Poissonverteilung handelt. Für die Aktivität wird ebenfalls der Fehler angegeben.

\begin{table}[H]
  \centering
  \caption{Zählrate und Aktivität in Abhängigkeit von der Zeit und der Dicke des Zinkabsorbers}
  \label{tab:Rechteckspannung}
  \begin{tabular}{c c c c}
    \toprule
    Zeit$/$s & $D/$mm & $N \pm \Delta N$ & $(A \pm \Delta A)/$s\\
    \midrule
    80 & 2 & $\num{15280(130)}$  &  $\num{149.4(17)}$ \\
    80 & 4 & $\num{11950(110)}$ &   $\num{132.0(14)}$ \\
    90 & 6 & $\num{10236(110)}$ &   $\num{113.7(13)}$ \\
    100 & 8 & $\num{10944(110)}$ &  $\num{109.4(11)}$ \\
    120 & 10 & $\num{11544(110)}$ & $\num{96.2(10)}$ \\
    140 & 12 & $\num{12317(120)}$ & $\num{88.0(9)}$ \\
    150 & 14 & $\num{11972(110)}$ & $\num{79.8(8)}$ \\
    160 & 16 & $\num{11734(110)}$ & $\num{73.3(7)}$ \\
    170 & 18 & $\num{11780(110)}$ & $\num{69.3(7)}$ \\
    180 & 20 & $\num{11035(110)}$ & $\num{61.3(7)}$ \\
    \bottomrule
  \end{tabular}
\end{table}

Es wird eine lineare Regression durchgeführt .Die Anpassungsfunktion wird durch die Gerade $y = ax + b$  dargestellt und die Parameter $a$ und $b$
mit Python berechnet.

\begin{figure}[H]
  \centering
  \includegraphics{plot.pdf}
  \caption{Lineare Regression der Aktivität aufgetragen gegen die Dicke des Zinkabsorbers}
  \label{fig:plot}
\end{figure}

Dabei gilt $A_0 =$ 1/s.

Der Absorptionskoeffizient $\mu$ entspricht dann der negativen Steigung der Geraden.
\begin{align*}
  a = \mu_{Zn} = \SI{0.0478(15)}{\per\milli\meter} = \SI{47.8(15)e2}{\per\meter}
\end{align*}

\subsubsection{Bestimmung des Absorptionskoeffizienten eines Bleiabsorber}

In Tabelle 2 werden die gemessenen Daten für einen Bleiabsorber dargestellt.

\begin{table}[H]
  \centering
  \caption{Zählrate und Aktivität in Abhängigkeit von der Zeit und der Dicke des Bleiabsorbers}
  \label{tab:Rechteckspannung}
  \begin{tabular}{c c c c}
    \toprule
    Zeit$/$s & $D/$mm & $N \pm \Delta N$ & $(A \pm \Delta A)/$s\\
    \midrule
    60 & 1 & $\num{7895(90)}$  &  $\num{131.6(15)}$ \\
    60 & 2 & $\num{7020(90)}$ &   $\num{117.0(15)}$ \\
    70 & 3 & $\num{7418(90)}$ &   $\num{106.0(13)}$ \\
    90 & 4 & $\num{8242(100)}$ &  $\num{91.6(11)}$ \\
    100 & 5 & $\num{8322(100)}$ & $\num{83.2(10)}$ \\
    130 & 10 & $\num{6843(90)}$ & $\num{52.6(7)}$ \\
    150 & 12 & $\num{6026(80)}$ & $\num{40.2(6)}$ \\
    170 & 15 & $\num{4932(80)}$ & $\num{29.0(5)}$ \\
    220 & 20 & $\num{4166(70)}$ & $\num{18.9(3)}$ \\
    350 & 30 & $\num{2569(60)}$ & $\num{7.3(2)}$ \\
    500 & 40 & $\num{1635(50)}$ & $\num{3.3(1)}$ \\
    700 & 50 & $\num{1176(40)}$ & $\num{1.7(1)}$ \\
    \bottomrule
  \end{tabular}
\end{table}

Es wird erneut eine lineare Regression durchgeführt.

\begin{figure}[H]
  \centering
  \includegraphics{plotblei.pdf}
  \caption{Lineare Regression der Aktivität aufgetragen gegen die Dicke des Bleiabsorbers}
  \label{fig:plot}
\end{figure}
