\section{Diskussion}
\label{sec:Diskussion}

Der Absorptionskoeffizient des Compton-Effekts von Blei weicht deutlich von dem experimentell bestimmten Absorptionskoeffizienten ab.
Daraus wird geschlossen, dass neben dem Compton-Effekt auch noch der Photoeffekt eine wichtige Rolle bei diesem Absorber spielt. Für
den Zinkabsorber weicht der experimentell bestimmte Absorptionskoeffizient nur geringfügig von $\mu_{com}$ ab, daraus
lässt sich schließen, dass bei dem Zinkabsorber hauptsächlich der Compton-Effekt in Erscheinung tritt.

Der lineare Zusammenhang zwischen der Dicke des Absorbers und dem Logarithmus der Aktivität ist klar zu erkennen.
Abweichungen können auf die zufällige Natur des Teilchenzerfalles zurückgeführt werden und der endlichen Messdauer.

Im Diagramm der Betastrahl-Messung ist deutlich der erwartete Kurvenverlauf zu erkennen. Zu Beginn ist ein Abfall der
Werte sichtbar, jedoch stellt sich bei größerer Dicke ein nahezu konstanter Wert ein. Es dominiert dort also, wie vermutet, die Bremsstrahlung, da
sie wesentlich durchdringender ist. Es können folglich auch gut die Ausgleichsfunktionen konstruiert werden, aus dessen Schnittpunkt
dann die maximale Energie errechnet wird. Da jedoch kein entsprechender Vergleichswert vorhanden ist, ist der errechnete Wert schwierig
zu beurteilen. Es muss zudem auch erwähnt werden, dass die erhaltenen Fehler (zu erkennen an den Fehlerbalken im Diagramm) verhältnismäßig
groß sind und somit auch durchaus eine gewisse Abweichung möglich ist.
