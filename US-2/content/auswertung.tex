\section{Auswertung}
\label{sec:Auswertung}

\subsection{Bestimmung  der Tiefe der Fehlstellen}

Die Höhe des Acrylblockes beträgt $8,01\,$cm und die Schallgeschwindigkeit in Acryl ist $c_A=2730 \, \symup{\frac{m}{s}}$.
Die gemessene Zeit eines A-Scans ohne Fehlstelle beträgt $t =59,12\, \symup{\mu}$s.

Mit Gleichung (5) wird die Höhe $h$ des Acrylblockes bestimmt.
\begin{align*}
  h = \frac{1}{2} c_A t = 0.0807 \symup{m} = 8.07  \symup{cm}
\end{align*}

Die so berechnete Höhe ist größer als die abgemessene Höhe, da der Ultraschall nicht ausschließlich durch den Acryblock läuft, sonder auch durch die
Schutzschicht der Sonde. Aus den beiden Höhen wird die Differenz $\Delta s$ berechnet:
\begin{align*}
  \Delta s = h - 8,01 \symup{cm} = \SI{5.99e-4}{\meter}
\end{align*}


Die mit der Schieblehre abgemessenen Abstände der Fehlstellen zu der oberen $(s_o)$ und unteren $s_u)$ Kante werden in Tabelle 1 dargestellt. Die Nummerierung der
Fehlstellen entspricht dabei der gleichen wie in Abbildung 1. Zusätzlich werden auch die mit dem A-Scan bestimmten Laufzeiten $(t_o \text{und} t_u)$ für die
die beiden Strecken, sowie die daraus resultierenden Strecken $(s_o' \text{und} s_u')$, in dieser Tabelle dargestellt.
\begin{table}[H]
  \centering
  \caption{Gemessene Abstände der Fehlstellen in dem Acrylblock.}
  \label{tab:spannung1}
  \begin{tabular}{c c c c c c c}
    \toprule
  Fehlstelle & $s_o/$cm & $s_u/$cm & $t_o/\symup{\mu}$s & $t_u/\symup{\mu}$s & $s_o'/$cm & $s_u'/$cm  \\
    \midrule
    1  &  1,95 & 5,97 & 14,44 & 43,97 & 1,97 & 6,00 &   \\
    2  &  1,74 & 6,15 & 13,23 & 45,20 & 1,81 & 6,17 &   \\
    3  &  6,11 & 1,33 & 45,18 & 9,84  & 6,17 & 1,34 &   \\
    4  &  5,40 & 2,18 & 40,42 & 16,50 & 5,52 & 2,25 &   \\
    5  &  4,63 & 3,01 & 34,37 & 22,55 & 4,69 & 3,08 &   \\
    6  &  3,89 & 3,89 & 28,86 & 28,90 & 3,94 & 3,94 &   \\
    7  &  3,08 & 4,66 & 22,91 & 34,85 & 3,13 & 4,76 &   \\
    8  &  2,29 & 5,48 & 17,08 & 40,58 & 2,33 & 5,54 &   \\
    9  &  1,49 & 6,18 & 11,27 & 46,43 & 1,54 & 6,34 &   \\
    10 &  0,70 & 7,17 & 5,43  & -     & 0,74 & -    & \\
    11 &  5,55 & 1,58 & 40,99 & 11,57 & 5,59 & 1,58 &   \\
    \bottomrule
  \end{tabular}
\end{table}

Dabei wurde für die Strecken $s_o' \text{und} s_u'$ die zusätzliche Strecke $\Delta s$ bereits abgezogen. Für die zehnte Fehlstelle konnte
kein Wert für $t_u$ gemessen werden, da die elfte Fehlstelle diese überdeckt.

Die Größe $d'$ der Fehlstelle wir durch $d' = h - s_o' - s_u'$ bestimmt. In Tabelle 2 wird die Größe $d'$ der Fehlstellen und die
Größe $d$ der Fehstellenangegeben. Dabei gilt $d = h - s_o - s_u$. Außerdem wird die relative Abweichung $\Delta d$ dargestellt.

%\begin{table}[H]
%  \centering
%  \caption{Dicke der Fehlstellen.}
%  \label{tab:spannung1}
%  \begin{tabular}{c c c c}
%    \toprule
%  Fehlstelle & $d/$cm & $d'/cm &   \\
%    \midrule
%    1  &     \\
%    2  &     \\
%    3  &     \\
%    4  &     \\
%    5  &     \\
%    6  &     \\
%    7  &     \\
%    8  &     \\
%    9  &     \\
%    10 &   \\
%    11 &     \\
%    \bottomrule
%  \end{tabular}
%\end{table}


\begin{figure}
  \centering
  \includegraphics{plot.pdf}
  \caption{Plot.}
  \label{fig:plot}
\end{figure}
