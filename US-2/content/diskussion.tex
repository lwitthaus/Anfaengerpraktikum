\section{Diskussion}
\label{sec:Diskussion}

Mit dem A-Scan konnten die Tiefen der Fehlstelle zumeist Millimeter genau bestimmt werden. Bei dem
B-Scan waren die die Abweichungen der Orte ein wenig größer. Beide Verfahren weisen große Abweichungen bei der
Größe der Fehlstelle auf. Die Abweichungen bei dem B-Scan können an dem ungenauen Bestimmen der Laufzeiten aus
den Abbildungen 2 und 3 liegen. Für den B-Scan wird idealarweise das Koppelmittel vernachlässigt, welches jedoch
ebenfalls zu Abweichungen von den realen Werten führt.
Bei dem A-Scan war zudem nicht eindeutig zu erkennen, wo der Peak der reflektierten Schallwellen des Koppelmittels war. So konnten
Fehler bei der Bestimmung der Laufzeit auftreten.
