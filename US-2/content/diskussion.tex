\section{Diskussion}
\label{sec:Diskussion}

Mit dem A-Scan konnten die Tiefen der Fehlstelle zumeist Millimeter genau bestimmt werden. Bei dem
B-Scan waren die die Abweichungen der Orte ein wenig größer. Beide Verfahren weisen große Abweichungen bei der
Größe der Fehlstelle auf. Die Abweichungen bei dem B-Scan können an dem ungenauen Bestimmen der Laufzeiten aus
den Abbildungen 2 und 3 liegen. Für den B-Scan wird idealarweise das Koppelmittel vernachlässigt, welches jedoch
ebenfalls zu Abweichungen von den realen Werten führt.
Bei dem A-Scan war zudem nicht eindeutig zu erkennen, wo der Peak der reflektierten Schallwellen des Koppelmittels war. So konnten
Fehler bei der Bestimmung der Laufzeit auftreten.

Bei der Messung des Herzvolumens konnte der Herzschlag gut simuliert werden. Es ist an der Abbildung unschwer zu erkennen, dass über den
gesamten Zeitraum ein relativ konstantes Schalgvolumen erzeugt werden konnte und auch die Schlagfrequenz weicht an keiner stelle
sichtlich ab. Daher ergibt sich auch durch die Mittelung keine große Abweichung zu irgendeinem der gemessenen Werte. Soweit ist
die Berechnung des Herzvolumens also gut möglich. Jedoch ist anzumerken, dass trotzdem noch die Näherung von dem Kugelsegment
als verdrängendem Volumen durchgeführt wurde. Trotz guter Bestimmbarkeit der Größe $h$ sollten dadurch also trotzdem noch gewisse Abweichungen
auftreten. Der errechnete Wert wird also nicht genau dem tatsächlichen Wert entsprechen.
