\section{Theorie}
\label{sec:Theorie}

\subsection{Eigenschaften von Ultraschall}
Ultraschall ist eine Schallwelle, wlche sich in dem Frequenzbereich von 20 kHz und 1 GHz befindet. Somit sind diese
Wellen nicht mehr für den Menschen hörbar. Schall breitet sich über Druckschwankungen aus:
\begin{align}
  p(x,t) = p_0 + v_0 Z \cos{\omega t - k x}
\end{align}
Hierbei ist $Z= c \rho$ die akkustische Impedanz, $p(x, t)$ die räumlich und zeitliche Druckverteilung, $\omega$ die frequenz der Welle,
$k$ der Wellenvektor, $\rho$ die Dichte und $v_0$ die Anfangsgeschwindigkeit.

Die Ausbreitungsgeschwindigkeit hängt von dem Medium. In Festkörpern sind auch transversale Wellen möglich. Ein Teil der
Energie geht bei der Schallausbreitung verloren. Für die Intensität $I$ gilt:
\begin{align}
  I =I_0 e^{\alpha x}
\end{align}

Dabei ist $\alpha$ der Absorptionskoeffizient der Schallamplitude.

Trifft die Schallwelle auf eine Grenzfläche wird ein Teil transmitiert und ein gewisser Teil reflektiert.
Für den Reflexionskoeffizient $R$ und den Transmissionskoeffizient $T$ von Ultraschall gilt:
\begin{align}
  R = \left( \frac{Z_1 - Z_2}{Z_1 + Z_2} \right) \\
  T = 1 - R
\end{align}

Ultraschall kann mit piezoelektrischen Kristalle, zum Beispiel Quarze, erzeugt werden. In elektrischen Wechselfeldern werden
sie zu Schwingungen angeregt, wenn eine polare Achse des Kristalls in Richtung des elektrischen Feldes zeigt. Dadruch strahlen sie
Ultraschallwellen ab. Wenn Eigenfrequenz und Anregungsfrequenz übereinstimmen kommt es zu Resonanzeffekten, wodurch hohe Schallenergiedichten
genutzt werden können.


\subsection{Verschiedene Verfahren der Ultraschallmessung}
Es werden zwei Verfahren in der Ultraschalltechnik verwendet.
Bei dem Durchallungsverfahren, wird mit einem Sender ein kurzzeitiger Ultraschallimpuls ausgesendet, welcher
durch die Probe läuft und dahinter von einem Empfänger aufgefangen wird. Ist in der Probe eine Fehlstelle vorhanden, wird dementsprechend
eine geringere Intensität von dem Empfänger gemessen. Es ist keine Aussage über den Ort der Fehlstelle möglich.

Bei dem Impuls-Echo-Verfahren dient der Ultraschallsender ebenfalls als Empfänger. Hierbei werden die von der Probe reflektierten
Schallwellen gemessen. Daraus kann die größe der Fehlstelle berechnet werden. Ist die Schallgeschwindigkeit $c$ in dem Medium bekannt
so kann zusätzlich der Ort $s$ der Fehlstelle bestimmt werden.
\begin{align}
  s = \frac{1}{2}ct
\end{align}
