\section{Diskussion}
\label{sec:Diskussion}

Die berechneten Brechungsindices werden wie in der Theorie beschrieben mit größer werdenden Wellenlängen kleiner.
Aus dem Diagramm kann eine Dispersionskurve klar identifiziert werden und somit der Verlauf der anderen in dieser
Messreihe verworfen werden. Abweichungen der Messwerte von dem Fit können durch systematische Fehler bei der Durchführung des
Experimentes erklärt werden. Zusätzlich ist das Bestimmen der Wellenlängen ebenfalls nicht ohne Ungenauigkeiten möglich.
Die berechnete Abbesche-Zahl weist eine große Ungenauigkeit, durch die Fortpflanzung des Fehlers der Brechungsindices, auf. Dies lässt die
Aussagekraft des berechneten Wertes fragwürdig erscheinen.
Da sich bei der Berechnung des Auflösungsvermögens sowie der Absorptionsstelle die oben genannten Fehler wieder einwirken, sollte es
auch dort zu gewissen Abweichungen kommen. Trotzdem hat die berechnete Absorptionsstelle einen den Erwartungen entsprechenden Wert.
Dieser liegt nämlich im UV-Bereich. Also ist die Annahme, dass sich keine Absorptionsstelle im Bereich des sichtbaren Lichts befindet
gerechtfertigt.
