\section{Auswertung}
\label{sec:Auswertung}

\subsection{Fehlerrechnung}
Der Mittelwert eines Datensatzes mit $N$ Werten ist definiert durch:
\begin{equation}
  \bar{x} = \frac{1}{N} \sum_{i=1}^N x_i
\end{equation}
Die Standardabweichung eines Datensatzes von seinem Mittelwert durch:
\begin{equation}
  \sigma = \sqrt{\frac{1}{N(N-1)} \sum_{i=1}^N (x_i - \bar{x})}
\end{equation}
Pflanzen sich Unsicherheiten fort, wird der Fehler mit der gaußschen
Fehlerfortpflanzung berechnet:
\begin{equation}
  \sigma_f = \sqrt{
      \sum\limits_{i = 1}^N
       \left( \frac{\partial f}{\partial x_i} \sigma_i \right)^{\!\! 2}
     }
\end{equation}


\subsection{Bestimmung der Brechungsindices}

In Tabelle 1 werden die gemessenen Winkel $\phi_r$ und $\phi_l$ dargestellt, sowie der reultierende Winkel $\phi=\frac{1}{2}(\phi_l-\phi_r)$.

\begin{table}[H]
  \centering
  \caption{Gemessene und berechnete $\phi$-Winkel.}
  \label{tab:spannung1}
  \begin{tabular}{c c c}
    \toprule
    $\phi_l/$° & $\phi_r/$° & $\phi/$° \\
    \midrule
    337,5  &  216,6 & 60,45     \\
    339,8  &  219,6 & 60,10      \\
    338,0  &  218,0 & 60,00      \\
    336,6  &  216,6 & 60,00      \\
    338,6  &  218,7 & 59,95      \\
    338,1  &  218,2 & 59,95      \\
    339,7  &  219,6 & 60,05      \\
    \bottomrule
  \end{tabular}
\end{table}

Der Mittelwert von den Winkeln $\phi$ und dessen Standardabweichung beträgt $\phi = (60,07 \pm 0,17)°$. Die Standardabweichung wird hierbei
mit Python berechnet.


Die Winkel $\Omega_l$ und $\Omega_r$, sowie der daraus resultierende Winkel $\eta = 180° - (\Omega_r-\Omega_l)$ werden in Tabelle 2 in Abhängigkeit von der
Wellenlänge $\lambda$ dargestellt.



\begin{table}[H]
  \centering
  \caption{Gemessene $\Omega_l$ und $\Omega_l$ und berechnete $\eta$-Winkel.}
  \label{tab:spannung1}
  \begin{tabular}{c c c c}
    \toprule
  $\lambda/$m &  $\Omega_l/$° & $\Omega_r/$° & $\eta/$° \\
    \midrule
    ?  & 104,7  &  214,5 & 70,2     \\
    ?  & 103,7  &  214,6 & 69,1     \\
    ?  & 103,0  &  226,6 & 66,4      \\
    ?  & 102,8  &  216,8 & 66,0      \\
    ?  & 101,8  &  217,0 & 64,8      \\
    ?  & 101,7  &  217,8 & 63,9      \\
    ?  & 101,5  &  218,1 & 63,4      \\
    ?  & 101,0  &  218,8 & 62,2      \\
    \bottomrule
  \end{tabular}
\end{table}

Mit Gleichung (10) wird aus $\phi$ und $\eta$ die Brechungsindices berechnet. Dabei errechnet sich der Fehler von $n$ über die Gauß'sche Fehlerfortpflanzung:
\begin{align*}
  \sigma_n = \sqrt{
   \sigma_{\phi}^{2} \left(\frac{\cos{\left (\frac{\eta}{2} + \frac{\phi}{2} \right )}}{2 \sin{\left (\frac{\phi}{2} \right )}} - \frac{\sin{\left (\frac{\eta}{2}
  + \frac{\phi}{2} \right )} \cos{\left (\frac{\phi}{2} \right )}}{2 \sin^{2}{\left (\frac{\phi}{2} \right )}}\right)^{2}}
\end{align*}


\begin{table}[H]
  \centering
  \caption{Brechungsindices in Abhängigkeit von der Wellenlänge.}
  \label{tab:spannung1}
  \begin{tabular}{c c}
    \toprule
  $\lambda/$m &  $n$ \\
    \midrule
    ?  & 1,578 \pm 0,002   \\
    ?  & 1,593 \pm 0,002    \\
    ?  & 1,631 \pm 0,002     \\
    ?  & 1,636 \pm 0,002     \\
    ?  & 1,654 \pm 0,002     \\
    ?  & 1,668 \pm 0,002     \\
    ?  & 1,676 \pm 0,002     \\
    ?  & 1,695 \pm 0,002     \\
    \bottomrule
  \end{tabular}
\end{table}



\begin{figure}
  \centering
  \includegraphics{plot.pdf}
  \caption{Plot.}
  \label{fig:plot}
\end{figure}
