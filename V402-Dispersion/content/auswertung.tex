\section{Auswertung}
\label{sec:Auswertung}

\subsection{Fehlerrechnung}
Der Mittelwert eines Datensatzes mit $N$ Werten ist definiert durch:
\begin{equation}
  \bar{x} = \frac{1}{N} \sum_{i=1}^N x_i
\end{equation}
Die Standardabweichung eines Datensatzes von seinem Mittelwert durch:
\begin{equation}
  \sigma = \sqrt{\frac{1}{N(N-1)} \sum_{i=1}^N (x_i - \bar{x})}
\end{equation}
Pflanzen sich Unsicherheiten fort, wird der Fehler mit der Gauß'schen
Fehlerfortpflanzung berechnet:
\begin{equation}
  \sigma_f = \sqrt{
      \sum\limits_{i = 1}^N
       \left( \frac{\partial f}{\partial x_i} \sigma_i \right)^{\!\! 2}
     }
\end{equation}


\subsection{Bestimmung der Brechungsindices}

In Tabelle 1 werden die gemessenen Winkel $\phi_r$ und $\phi_l$ dargestellt, sowie der resultierende Winkel $\phi=\frac{1}{2}(\phi_l-\phi_r)$.

\begin{table}[H]
  \centering
  \caption{Gemessene und berechnete $\phi$-Winkel.}
  \label{tab:spannung1}
  \begin{tabular}{c c c}
    \toprule
    $\phi_l/$° & $\phi_r/$° & $\phi/$° \\
    \midrule
    337,5  &  216,6 & 60,45     \\
    339,8  &  219,6 & 60,10      \\
    338,0  &  218,0 & 60,00      \\
    336,6  &  216,6 & 60,00      \\
    338,6  &  218,7 & 59,95      \\
    338,1  &  218,2 & 59,95      \\
    339,7  &  219,6 & 60,05      \\
    \bottomrule
  \end{tabular}
\end{table}

Der Mittelwert von den Winkeln $\phi$ und dessen Standardabweichung beträgt $\phi = (60,07 \pm 0,17)°$. Die Standardabweichung wird hierbei
mit Python berechnet.


Die Winkel $\Omega_l$ und $\Omega_r$, sowie der daraus resultierende Winkel $\eta = 180° - (\Omega_r-\Omega_l)$ werden in Tabelle 2 in Abhängigkeit von der
Wellenlänge $\lambda$ dargestellt.



\begin{table}[H]
  \centering
  \caption{Gemessene $\Omega_l$ und $\Omega_l$ und berechnete $\eta$-Winkel.}
  \label{tab:spannung1}
  \begin{tabular}{c c c c}
    \toprule
  $\lambda/$nm &  $\Omega_l/$° & $\Omega_r/$° & $\eta/$° \\
    \midrule
    404,6  & 104,7  &  214,5 & 70,2     \\
    439,8  & 103,7  &  214,6 & 69,1     \\
    480,0  & 103,0  &  226,6 & 66,4      \\
    502,5  & 102,8  &  216,8 & 66,0      \\
    546,1  & 101,8  &  217,0 & 64,8      \\
    577,0  & 101,7  &  217,8 & 63,9      \\
    615,8  & 101,5  &  218,1 & 63,4      \\
    623,4  & 101,0  &  218,8 & 62,2      \\
    \bottomrule
  \end{tabular}
\end{table}

Die Wellenlängen sind dabei Literaturwerte \cite{sample2}.

Mit Gleichung (10) wird aus $\phi$ und $\eta$ die Brechungsindices berechnet. Dabei errechnet sich der Fehler von $n$ über die Gauß'sche Fehlerfortpflanzung:
\begin{align*}
  \sigma_n = \sqrt{
   \sigma_{\phi}^{2} \left(\frac{\cos{\left (\frac{\eta}{2} + \frac{\phi}{2} \right )}}{2 \sin{\left (\frac{\phi}{2} \right )}} - \frac{\sin{\left (\frac{\eta}{2}
  + \frac{\phi}{2} \right )} \cos{\left (\frac{\phi}{2} \right )}}{2 \sin^{2}{\left (\frac{\phi}{2} \right )}}\right)^{2}}
\end{align*}



\begin{table}[H]
  \centering
  \caption{Brechungsindices in Abhängigkeit von der Wellenlänge.}
  \label{tab:spannung1}
  \begin{tabular}{c c}
    \toprule
  $\lambda/$nm &  $n$ \\
    \midrule
    404,6  & 1,813 \pm 0,004   \\
    439,8  & 1,805 \pm 0,004    \\
    480,0  & 1,784 \pm 0,004     \\
    502,5  & 1,781 \pm 0,004     \\
    546,1  & 1,771 \pm 0,004     \\
    577,0  & 1,764 \pm 0,004     \\
    615,8  & 1,760 \pm 0,004     \\
    623,4  & 1,750 \pm 0,004     \\
    \bottomrule
  \end{tabular}
\end{table}

\subsection{Ausgleichsrechnung für die Dispersionskurve}
Wird $n^2$ gegen die entsprechende Wellenlänge aufgetragen, so ergibt sich das folgende Diagramm.

\begin{figure}
  \centering
  \includegraphics{plot.pdf}
  \caption{Brechungsindices aufgetragen gegen die Wellenlänge.}
  \label{fig:plot}
\end{figure}

Dabei muss für die Ausgleichsrechnung zwischen den Gleichungen (7) und (9) entschieden werden.
Mithilfe von Python wird die Methode der kleinsten Quadrate angewandt und für die beiden
Gleichungen die folgenden Parameter ermittelt:

Gleichung (7):
\begin{align*}
  &A_0 = 2,93 \pm 0,01 \\
  &A_2 = (5,9 \pm 0,3)\cdot 10^4 \: \symup{nm^2}
\end{align*}
Gleichung (9):
\begin{align*}
  &A_0' = 3,40 \pm 0,03 \\
  &A_2' = (8,1 \pm 0,9)\cdot 10^{-7} \: \symup{\frac{1}{nm^2}}
\end{align*}

Daraus ergibt sich für die jeweiligen Abweichungsquadrate
\begin{align*}
  &s_7 = 0,00015 \\
  &s_9 = 0,0005
\end{align*}

Da das Abweichungsquadrat der Funktion aus Gleichung (7) wesentlich kleiner ist,
wird also eine Ausgleichsrechnung mit dieser Funktion durchgeführt.



\subsection{Berechnung der Abbeschen-Zahl}

Für die Abbesche-Zahl wird:
\begin{align}
  \nu = \frac{n_D-1}{n_F-n_C}
\end{align}

Hierbei sind $n_D$, $n_F$ und $n_C$ die Brechungsindices des Glasprismas bei den Fraunhoferschen Linien $\lambda_D =589$nm, $\lambda_F = 486$nm
und $\lambda_C = 656$nm. Für die Berechnung der Brechungsindices wird Gleichung (7) verwendet.

\begin{table}[H]
  \centering
  \caption{Brechungsindices für drei Fraunhofer Linien.}
  \label{tab:spannung1}
  \begin{tabular}{c c}
    \toprule
  $\lambda/$nm &  $n$ \\
    \midrule
    $\lambda_D$ & 1,761  \pm 0,004  \\
    $\lambda_F$ & 1,783   \pm 0,005  \\
    $\lambda_C$ & 1,751  \pm 0,004   \\
    \bottomrule
  \end{tabular}
\end{table}

Für die Abbesche-Zahl folgt:
\begin{align*}
  \nu = 24 \pm 5
\end{align*}

Der Fehler von $n$ und $\nu$ wird wieder mit der 'Gauß'schen Fehlerfortpflanzung berechnet.

\begin{align*}
  \sigma_n = \sqrt{\frac{\sigma_{A_0}^{2}}{4 A_0 + \frac{4 A_2}{\lambda^{2}}} + \frac{\sigma_{A_2}^{2}}{4 \lambda^{4} \left(A_0 + \frac{A_2}{\lambda^{2}}\right)}} \\
  \sigma_{\nu}= \sqrt{\frac{\sigma_{n_C}^{2} \left(n_D - 1\right)^{2}}{\left(- n_C + n_F\right)^{4}}
  + \frac{\sigma_{n_D}^{2}}{\left(- n_C + n_F\right)^{2}} + \frac{\sigma_{n_F}^{2} \left(n_D - 1\right)^{2}}{\left(- n_C + n_F\right)^{4}}}
\end{align*}


\subsection{Bestimmung des Auflösungsvermögens}
Das Auflösungsvermögen A ist gegeben durch
\begin{equation*}
  A = \frac{\lambda}{\Delta \lambda} = b\frac{dn}{d\lambda}
\end{equation*}
mit
\begin{equation*}
  \frac{dn}{d\lambda} = \frac{A_2}{\lambda^3\sqrt{A_0 + \frac{A_2}{\lambda^2}}}
\end{equation*}
und $b$ ist die Basislänge mit $b = 3$ cm.
Daraus ergibt sich für die beiden Wellenlängen $\lambda_C$ und $\lambda_F$:
\begin{align*}
  &A_C = 3580 \pm 180 \\
  &A_F = 8600 \pm 400
\end{align*}


Die Fehler von $A_C$ und $A_F$ werden mit der Gauß'schen Fehlerfortpflanzung berechnet.
\begin{align*}
  \sigma_A = \sqrt{\frac{A_2^{2} \sigma_{A_0}^{2}}{4 \lambda^{6} \left(A_0 + \frac{A_2}{\lambda^{2}}\right)^{3}}
  + \sigma_{A_2}^{2} \left(- \frac{A_2}{2 \lambda^{5} \left(A_0 + \frac{A_2}{\lambda^{2}}\right)^{\frac{3}{2}}}
  + \frac{1}{\lambda^{3} \sqrt{A_0 + \frac{A_2}{\lambda^{2}}}}\right)^{2}}
\end{align*}

\subsection{Bestimmung der Absorptionsstelle}
Aus einem Koeffizientenvergleich von Gleichung (6) und (7) ergibts ich für die gesuchte Absorptionsstelle:
\begin{equation*}
  \lambda_1 = \sqrt{\frac{A_2}{A_0 - 1}}
\end{equation*}
Daraus ergibt sich ein Wert von
\begin{equation*}
  \lambda_1 = (175 \pm 4) \: \symup{nm} .
\end{equation*}

Für den berechneten Fehler von $\lambda_1$ gilt:
\begin{align*}
  \sigma_{\lambda_1} = \sqrt{\frac{A_2^{2} \sigma_{A_0}^{2}}{\left(A_0 - 1\right)^{4}} + \frac{\sigma_{A_2}^{2}}{\left(A_0 - 1\right)^{2}}}
\end{align*}
