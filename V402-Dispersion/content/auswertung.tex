\section{Auswertung}
\label{sec:Auswertung}

\subsection{Fehlerrechnung}
Der Mittelwert eines Datensatzes mit $N$ Werten ist definiert durch:
\begin{equation}
  \bar{x} = \frac{1}{N} \sum_{i=1}^N x_i
\end{equation}
Die Standardabweichung eines Datensatzes von seinem Mittelwert durch:
\begin{equation}
  \sigma = \sqrt{\frac{1}{N(N-1)} \sum_{i=1}^N (x_i - \bar{x})}
\end{equation}
Pflanzen sich Unsicherheiten fort, wird der Fehler mit der gaußschen
Fehlerfortpflanzung berechnet:
\begin{equation}
  \sigma_f = \sqrt{
      \sum\limits_{i = 1}^N
       \left( \frac{\partial f}{\partial x_i} \sigma_i \right)^{\!\! 2}
     }
\end{equation}


\subsection{Bestimmung der Brechungsindices}

In Tabelle 1 werden die gemessenen Winkel $\phi_r$ und $\phi_l$ dargestellt, sowie der reultierende Winkel $\phi=\frac{1}{2}(\phi_l-\phi_r)$.

\begin{table}[H]
  \centering
  \caption{Gemessene und berechnete $\phi$-Winkel.}
  \label{tab:spannung1}
  \begin{tabular}{c c c}
    \toprule
    $\phi_l/$° & $\phi_r/$° & $\phi/$° \\
    \midrule
    337,5  &  216,6 & 60,45     \\
    339,8  &  219,6 & 60,10      \\
    338,0  &  218,0 & 60,00      \\
    336,6  &  216,6 & 60,00      \\
    338,6  &  218,7 & 59,95      \\
    338,1  &  218,2 & 59,95      \\
    339,7  &  219,6 & 60,05      \\
    \bottomrule
  \end{tabular}
\end{table}

Der Mittelwert von den Winkeln $\phi$ und dessen Standardabweichung beträgt $\phi = (60,07 \pm 0,17)°$. Die Standardabweichung wird hierbei
mit Python berechnet.


Die Winkel $\Omega_l$ und $\Omega_r$, sowie der daraus resultierende Winkel $\eta = 180° - (\Omega_r-\Omega_l)$ werden in Tabelle 2 in Abhängigkeit von der
Wellenlänge $\lambda$ dargestellt.



\begin{table}[H]
  \centering
  \caption{Gemessene $\Omega_l$ und $\Omega_l$ und berechnete $\eta$-Winkel.}
  \label{tab:spannung1}
  \begin{tabular}{c c c c}
    \toprule
  $\lambda/$nm &  $\Omega_l/$° & $\Omega_r/$° & $\eta/$° \\
    \midrule
    404,6  & 104,7  &  214,5 & 70,2     \\
    439,8  & 103,7  &  214,6 & 69,1     \\
    480,0  & 103,0  &  226,6 & 66,4      \\
    502,5  & 102,8  &  216,8 & 66,0      \\
    546,1  & 101,8  &  217,0 & 64,8      \\
    577,0  & 101,7  &  217,8 & 63,9      \\
    615,8  & 101,5  &  218,1 & 63,4      \\
    623,4  & 101,0  &  218,8 & 62,2      \\
    \bottomrule
  \end{tabular}
\end{table}

Mit Gleichung (10) wird aus $\phi$ und $\eta$ die Brechungsindices berechnet. Dabei errechnet sich der Fehler von $n$ über die Gauß'sche Fehlerfortpflanzung:
\begin{align*}
  \sigma_n = \sqrt{
   \sigma_{\phi}^{2} \left(\frac{\cos{\left (\frac{\eta}{2} + \frac{\phi}{2} \right )}}{2 \sin{\left (\frac{\phi}{2} \right )}} - \frac{\sin{\left (\frac{\eta}{2}
  + \frac{\phi}{2} \right )} \cos{\left (\frac{\phi}{2} \right )}}{2 \sin^{2}{\left (\frac{\phi}{2} \right )}}\right)^{2}}
\end{align*}



\begin{table}[H]
  \centering
  \caption{Brechungsindices in Abhängigkeit von der Wellenlänge.}
  \label{tab:spannung1}
  \begin{tabular}{c c}
    \toprule
  $\lambda/$nm &  $n$ \\
    \midrule
    404,6  & 1,813 \pm 0,004   \\
    439,8  & 1,805 \pm 0,004    \\
    480,0  & 1,784 \pm 0,004     \\
    502,5  & 1,781 \pm 0,004     \\
    546,1  & 1,771 \pm 0,004     \\
    577,0  & 1,764 \pm 0,004     \\
    615,8  & 1,760 \pm 0,004     \\
    623,4  & 1,750 \pm 0,004     \\
    \bottomrule
  \end{tabular}
\end{table}

Wird $n^2$ gegen die entsprechende Wellenlänge aufgetragen, so ergibt sich das folgende Diagramm.

\begin{figure}
  \centering
  \includegraphics{plot.pdf}
  \caption{Brechungsindices aufgetragen gegen die Wellenlänge.}
  \label{fig:plot}
\end{figure}

Dabei muss für die Ausgleichsrechnung zwischen den Gleichungen (7) und (9) entschieden werden.
Mithilfe von Python wird die Methode der kleinsten Quadrate angewandt und für die beiden
Gleichungen die folgenden Parameter ermittelt:

Gleichung (7):
\begin{align*}
  &A_0 = 2,93 \pm 0,01 \\
  &A_1 = (5,9 \pm 0,3)\cdot 10^4 \: \symup{m^2}
\end{align*}
Gleichung (9):
\begin{align*}
  &A_0' = 3,40 \pm 0,03 \\
  &A_1' = (8,1 \pm 0,9)\cdot 10^{-7} \: \symup{m^2}
\end{align*}
